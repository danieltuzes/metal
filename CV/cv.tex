%% start of file `template.tex'.
%% Copyright 2006-2015 Xavier Danaux (xdanaux@gmail.com).
%
% This work may be distributed and/or modified under the
% conditions of the LaTeX Project Public License version 1.3c,
% available at http://www.latex-project.org/lppl/.


\documentclass[11pt,a4paper,sans]{moderncv}        % possible options include font size ('10pt', '11pt' and '12pt'), paper size ('a4paper', 'letterpaper', 'a5paper', 'legalpaper', 'executivepaper' and 'landscape') and font family ('sans' and 'roman')

% moderncv themes
\moderncvstyle{casual}                             % style options are 'casual' (default), 'classic', 'banking', 'oldstyle' and 'fancy'
\moderncvcolor{red}                               % color options 'black', 'blue' (default), 'burgundy', 'green', 'grey', 'orange', 'purple' and 'red'
%\renewcommand{\familydefault}{\sfdefault}         % to set the default font; use '\sfdefault' for the default sans serif font, '\rmdefault' for the default roman one, or any tex font name
%\nopagenumbers{}                                  % uncomment to suppress automatic page numbering for CVs longer than one page

% character encoding
%\usepackage[utf8]{inputenc}                       % if you are not using xelatex ou lualatex, replace by the encoding you are using
%\usepackage{CJKutf8}                              % if you need to use CJK to typeset your resume in Chinese, Japanese or Korean

% adjust the page margins
\usepackage[scale=0.75]{geometry}
%\setlength{\hintscolumnwidth}{3cm}                % if you want to change the width of the column with the dates
%\setlength{\makecvtitlenamewidth}{10cm}           % for the 'classic' style, if you want to force the width allocated to your name and avoid line breaks. be careful though, the length is normally calculated to avoid any overlap with your personal info; use this at your own typographical risks...

% personal data
\name{Dániel}{Tüzes}
%\title{Resumé}                               % optional, remove / comment the line if not wanted
\address{Eötvös University, Pázmány Péter sétány 1/A}{1117 Budapest}{Hungary}% optional, remove / comment the line if not wanted; the "postcode city" and "country" arguments can be omitted or provided empty
\phone[mobile]{+36 70 335 8043}                   % optional, remove / comment the line if not wanted; the optional "type" of the phone can be "mobile" (default), "fixed" or "fax"
\phone[fixed]{+36 1 372 2805}
\email{danieltuzes@gmail.com}                               % optional, remove / comment the line if not wanted
\homepage{metal.elte.hu/\textasciitilde tuzes/personal}                         % optional, remove / comment the line if not wanted
\social[linkedin]{daniel-tuzes}                        % optional, remove / comment the line if not wanted
\social[github]{danieltuzes}                              % optional, remove / comment the line if not wanted
\extrainfo{\href{https://www.researchgate.net/profile/Daniel_Tuezes/}{ResearchGate}, \href{https://scholar.google.hu/citations?hl=en&user=CCrwBwMAAAAJ}{Google Scholar}}                 % optional, remove / comment the line if not wanted
\photo[64pt][0.4pt]{picture}                       % optional, remove / comment the line if not wanted; '64pt' is the height the picture must be resized to, 0.4pt is the thickness of the frame around it (put it to 0pt for no frame) and 'picture' is the name of the picture file
%\quote{Some quote}                                 % optional, remove / comment the line if not wanted

% bibliography adjustements (only useful if you make citations in your resume, or print a list of publications using BibTeX)
%   to show numerical labels in the bibliography (default is to show no labels)
\makeatletter\renewcommand*{\bibliographyitemlabel}{\@biblabel{\arabic{enumiv}}}\makeatother
%   to redefine the bibliography heading string ("Publications")
%\renewcommand{\refname}{Articles}

% bibliography with mutiple entries
%\usepackage{multibib}
%\newcites{book,misc}{{Books},{Others}}
%----------------------------------------------------------------------------------
%            content
%----------------------------------------------------------------------------------
\begin{document}
%-----       resume       ---------------------------------------------------------
\makecvtitle
Dániel Tüzes is a post doctoral student in physics, earned his Ph.D. and M.Sc. in physics and did his research at FAU Nürnberg-Erlangen and Eötvös University, Budapest. His research interests include stochastic simulations and modelling of non-conservative systems, and verifying the predictions by experimental investigation. He exploits the efficiency of C++ and the compute power of GPGPU in his stochastic simulations.\vspace{0.5em}

\textit{Theoretic aspects}: describing dislocation systems as a non-conservative N-body problem; using mean-field or local density approximation and finding the corresponding equations of motion. Derive predictions of the constructed models investigable by experiments.\vspace{0.5em}

\textit{Numerical aspects}: solving the partial differential equations describing the equation of motion in a numeric way requires effective coding achieved with C++. Dániel Tüzes is enthusiastic in exploiting the advantages of GPU programming in simulations.\vspace{0.5em}

\textit{Experimental aspects}: the predictions of the models can be verified by experiments using in-situ nanoindentation of single crystals with coupled acoustic emission detection.

\section{Education}
\cventry{2012--2019}{Ph.D.}{}{Eötvös University, Budapest and Friedrich Alexander University Nürnberg-Erlangen}{\textit{Summa cum laude}}{Materials science and condensed matter physics training program\\Thesis title: Stochastic properties of dislocation motion and rearrangement}  % arguments 3 to 6 can be left empty
\cventry{2010--2012}{M.Sc.}{Eötvös University, Budapest}{}{}{Research physicist training program; Atomic and Molecular Physics module\\Thesis title: Determining polarization of dislocation ensembles}
\cventry{2007--2010}{B.Sc.}{Eötvös University, Budapest}{Physicist training program}{}{}

\section{Computer skills}
\cvdoubleitem{C++}{Intermediate}{up to c++14}{preferred programming language}
\cvdoubleitem{Gnuplot}{Intermediate}{with scripting}{prefered data visualisation}

\cvdoubleitem{LaTeX, HTML, MD}{intermediate to\\ advanced level}{}{using pandoc to derive\\any markup language}\cvdoubleitem{Python}{Basic level}{}{passive knowledge}
\cvdoubleitem{PHP,JS}{Basic level}{}{former hobby}

\section{Languages}
\cvitemwithcomment{English}{Advanced level, C1}{Active}
\cvitemwithcomment{German}{Intermediate level, B2}{Active}
\cvitemwithcomment{French}{Basic level}{Passive}
\cvitemwithcomment{Hungarian}{Native}{Active}

\section{Experience}
\subsection{Vocational}
\cventry{2012--2019}{PhD Student}{Eötvös University}{Budapest}{}{Teacher assistant and lecturer in the field of Mechanics; International collaboration in research groups\newline{}%
Detailed achievements:%
\begin{itemize}%
\item wrote stochastic simulations to integrate non-deterministic PDEs
\item improved the efficiency of existing PDE solver by a factor 2 using mathematical identities
\item improved code efficiency by a factor from 2 to 5, by eliminating cache mismaches
\item implemented the stochastic model on GPU
\item developed a more impressive, detailed, sophisticated and sensitive way of analyzing patterns on a 2D scalar field
\item constructed study material from scratch; prepared setups for demonstrations and fast, live data visualization
\item initiated a new investigation topic in an international research collaboration group with an output of a PRB article
\end{itemize}
Achieved skills:
\begin{itemize}
\item advanced level programming in C++
\begin{itemize}
\item wrote code for different operating systems
\item used different compilers and options for different target systems and purposes
\item used STL intensively, along with industry-standard libraries, for example FFTW, LAPACK and CUDA
\item used classes, templates, lambdas
\end{itemize}
\item advanced level scripting in gnuplot, producing informative and impressive figures for data visualization
\item creating documentations in LaTeX, pandoc and different flavors of Markdown
\item high level of collaboration and cooperation, the ability to socialize and involve lonely researchers in the workgroup
\item intermediate level scipting in bash, regular expressions and git
\end{itemize}}
\cventry{2014--2016}{PhD Student}{Friedrich-Alexander Universität}{Nürnberg-Erlangen}{}{Invited researcher\newline{}Achievements and skills
\begin{itemize}
\item successful international cooperation resulting in two papers, one of them is published in PRB
\item working in an international and multicultural environment;
\item fluent use of English
\item basic Python programming
\end{itemize}}
\subsection{Miscellaneous}
\cventry{7.2019 -- 9.2019}{Production leader and manager}{Franz-Fertig -- DUNAPOL}{Buchen -- Budapest}{}{After a major internal and external attack on the company DUNAPOL, I reconstructed the information technology system at DUNAPOL, commissioned by Franz-Fertig. I investigated the used technology and methods how the company was supposedly working and measured the damages. I recreated the IT infrastructure, including the automated production management system, and by involving database specialist, I repaired the damaged files. With the assistance of the company owner, I coordinated the transportation between the parent company and the subsidiary, ordered and arranged work for the employees and helped managing the acquisitions and the invoicing.\newline{}\textbf{Achieved skills}:
\begin{itemize}
\item working in an employee - employer relationship
\item ordering and arranging tasks for workers below me
\item high workload: 12 hrs a day, 7 day a week for couple of weeks
\item using German and Hungarian side by side
\end{itemize}}

\section{Publications}
\subsection{Peer reviwed journals}
\begin{itemize}
\item Wu, RH; \textbf{Tüzes, D}; Ispanovity, PD; Groma, I; Hochrainer, T; Zaiser, M\\
	Instability of dislocation fluxes in a single slip:\\
	Deterministic and stochastic models of dislocation patterning\\
    \href{https://journals.aps.org/prb/abstract/10.1103/PhysRevB.98.054110}{PHYSICAL REVIEW B 98 : 5 Paper: 054110 , 15 p. (2018)};
    \href{https://arxiv.org/pdf/1708.05533}{arxiv PDF}
\item István, Hegyi Ádám; Dusán, Ispánovity Péter; Knapek, Michal; \textbf{Tüzes, Dániel}; Máthis, Krisztián; Chmelík, František; Dankházi, Zoltán; Varga, Gábor; Groma, István\\
    Micron-Scale Deformation: A Coupled In Situ Study of Strain Bursts and Acoustic Emission\\
    \href{https://www.cambridge.org/core/journals/microscopy-and-microanalysis/article/micronscale-deformation-a-coupled-in-situ-study-of-strain-bursts-and-acoustic-emission/DAF84F7E4CC7C2A211E39A1FFBCCB2D0}{MICROSCOPY AND MICROANALYSIS 23 : 6 pp. 1076-1081. , 6 p. (2017)};
    \href{https://arxiv.org/pdf/1604.01815}{arxiv PDF}
\item \textbf{Tüzes, D}; Ispánovity, PD; Zaiser, M\\
    Disorder is good for you: the influence of local disorder on strain localization and ductility of strain softening materials\\
    \href{https://link.springer.com/article/10.1007\%2Fs10704-017-0187-1}{INTERNATIONAL JOURNAL OF FRACTURE 205 : 2 pp. 139-150. , 12 p. (2017)};
    \href{https://arxiv.org/pdf/1604.01821}{arxiv PDF}
\item Ispánovity, PD; \textbf{Tüzes, D}; Szabó, P; Zaiser, M; Groma, I\\
    Role of weakest links and system-size scaling in multiscale modeling of stochastic plasticity\\
    \href{https://journals.aps.org/prb/abstract/10.1103/PhysRevB.95.054108}{PHYSICAL REVIEW B 95 : 5 Paper: 054108 , 13 p. (2017)};
    \href{https://arxiv.org/pdf/1604.01645}{arxiv PDF}
\item Groma, I; \textbf{Tüzes, D}; Ispánovity, PD\\
    Asymmetric X-ray line broadening caused by dislocation polarization induced by external load\\
    \href{https://www.sciencedirect.com/science/article/pii/S1359646213000110}{SCRIPTA MATERIALIA 68 : 9 pp. 755-758. , 4 p. (2013)}; \href{http://metal.elte.hu/~tuzes/docs/pre\%20AsymmetricX-ray\%20line\%20broadening\%20caused\%20by\%20dislocation\%20polarization\%20induced\%20by\%20external\%20load.pdf}{early version PDF}
\end{itemize}
    
\subsection{Books, lecture notes}
\begin{itemize}
\item Analysis I - III\\The official coursebook for physicists at Eötvös University (Hungarian)
\\Original title: "Analízis jegyzetek I-III". ISBN: (soon) \href{http://web.cs.elte.hu/~tarcsay/analizis-jegyzetek.pdf}{Download in PDF}

\item Mechanics practice\\
Short summary of the lecture notes and examples for calculation (Hungarian)\\
Original title: "Mechanika (emelt szint), gyakorlat". \href{https://metalog.elte.hu/nextcloud/index.php/s/AKqe8ZxjCswPrJE}{Download in PDF}

\item Continuum mechanics practice\\
Introduction to deformations and strains, and examples for calculation (Hungarian)\\
Original title: "Folytonos közegek mechanikája (emelt szint), gyakorlat". \href{https://metalog.elte.hu/nextcloud/index.php/s/QcPbMpB6cR6MsNE}{Download in PDF}

\item Quantum many body physics\\
A lecture note for MSc students. Lecturer: Gergely Szirmai and András Csordás (Hungarian)\\
Original title: "Soktestprobléma II".
\end{itemize}

\section{Talks and presentations}
\cvitem{Budapest\break 24th, May, 2018}{Dislocation patterns in a 2D stochastic model;\newline Original title: "Diszlokcáiómintázatok 2D-s sztochasztikus modellben"; \newline Presentation; report for the New National Excellence Program}

\cvitem{Budapest\break
27th, February, 2018}{Dislocation patterning in a 2D stochastic continuum model\newline  Original title: "Diszlokcáiómintázatok 2D-s sztochasztikus modellben";\newline  Presentation; seminar talk at the Eötvös University, Department of Materials Physic}

\cvitem{Nuremberg\break 
2nd, February, 2018}{Dislocation patterning in a 2D stochastic continuum model\newline Presentation; seminar talk at FAU, Institute of Materials Simulation}

\cvitem{Vienna\break
10-12th, November, 2017}{Dislocation patterning in a 2D stochastic continuum model
\newline  Presentation; 9th Seminar for Central European PhD Students – Research in Materials Science}

\cvitem{Lugano\break
1st, March, 2017}{A mesoscopic stochastic model for micron-scale plasticity
\newline  Presentation; CECAM, Challenges in crystal plasticity: from discrete dislocations to continuum models}

\cvitem{Budapest\break
18th, October, 2016}{The role of disorder in dislocation localization
\newline  Original title: "A rendezetlenség szerepe a deformáció lokalizációjában"
\newline  Presentation; seminar talk at Eötvös University, Department of Materials Physics}

\cvitem{Prague\break
19-22nd, October, 2014}{A mesoscopic stochastic model for micron-scale plasticity
\newline Presentation; 9th Seminar for Central European PhD Students\newline Research in Materials Science}

\cvitem{Muenchen\break 
15th, October, 2014}{Micropillar compressions and a mesoscopic stochastic model\newline  for micron-scale plasticity
\newline Presentation; BAYHOST Seminar}

\cvitem{Berkeley\break 
6-10th, October, 2014}{A mesoscopic stochastic model for micron-scale plasticity
\newline Poster section; Proceedings of the 7th international conference\newline on multiscale materials modeling, MMM-7}

\section{Scholarships and awards}
\cvitem{2018--2019}{two-year research scholarship\newline granted by the National Competitiveness and Excellence Program}
\cvitem{2017/2018}{one-year "New National Excellence Program scholarship"\newline granted by the Ministry of Human Capacities (sic)}
\cvitem{2016/2017}{1 year research scholarship granted by Prof. Michael Zaiser}
\cvitem{2015/2016}{1 year PhD scholarship granted by BayHost}
\cvitem{2012--2017}{PhD scholarship granted by the Hungarian State}
\cvitem{2011/2012}{Excellence of Faculty}
\cvitem{2011/2012}{Fellowship granted by the Republic}
\cvitem{2009/2010}{Fellowship granted by the Republic}
\cvitem{2008}{2nd place at \href{https://ortvay.elte.hu/main.html}{Rudolf Ortvay International Competition in Physics}}

\section{Community services and affiliations}
\cvitem{2008 -- }{Organizer in the Dürer Competition in Mathematics, Physics and Chemistry\newline
A comptetition for secondary school students}
\cvitem{2017 -- }{Mentor in the \href{http://hypt.elte.hu/}{Hungarian team} of the International Young Physicists Tournament}
\cvitem{2008 -- }{Founder and maintainer of \href{http://fizweb.elte.hu/}{FizWeb}\newline
The largest lecture note, example, exercise and book sharing platform for physicists at Eötvös University}
\cvitem{2013 -- 2019}{\href{https://talentcentrebudapest.eu/}{Content manager and maintainer of the website of the Talent Center Budapest; EUTCB is founded to foster and coordinate the joint European talent support activities}}
\cvitem{2016 -- 2018}{Content manager and maintainer of the webpage of the \href{http://echa.info/}{European Council for High Ability} and \href{http://etsn.eu/}{European Talent Support Network}}
\cvitem{2008 -- 2014}{\href{http://tehetsegpont.hu/}{Content manager and maintainer of the National Talent Support Council}}
\cvitem{2007 -- 2013}{\href{http://www.linkgroup.hu/}{Developer, content manager and maintainer of LinkGroup, the homepage of a biochemistry and network oriented research group} led by \href{https://en.wikipedia.org/wiki/Peter_Csermely}{Péter Csermely}}
\cvitem{2007 -- 2012}{Member of Bolyai College, the elite college of the Science Faculty}

\section{Hobbies}
\cvitem{sports}{Ultimate frisbee, table tennis}
\cvitem{dancing}{ballroom dancing, freestyle inline skating}
\cvitem{arts}{pottery, industrial arts}


\section{References}
\begin{cvcolumns}
  \cvcolumn{Professors}{\begin{itemize}
  \item István Groma, \href{mailto:groma@metal.elte.hu}{groma@metal.elte.hu}
  \item Michael Zaiser, \href{mailto:michael.zaiser@fau.de}{michael.zaiser@fau.de}\end{itemize}}
  \cvcolumn[0.5]{Company}{Franz Schwander, \href{https://www.linkedin.com/in/franz-schwander-249a63131/}{LinkedIN}\\(other contacts available per request)}
\end{cvcolumns}

\end{document}