%% start of file `template.tex'.
%% Copyright 2006-2013 Xavier Danaux (xdanaux@gmail.com).
%
% This work may be distributed and/or modified under the
% conditions of the LaTeX Project Public License version 1.3c,
% available at http://www.latex-project.org/lppl/.


\documentclass[11pt,a4paper,sans]{moderncv}        % possible options include font size ('10pt', '11pt' and '12pt'), paper size ('a4paper', 'letterpaper', 'a5paper', 'legalpaper', 'executivepaper' and 'landscape') and font family ('sans' and 'roman')

% moderncv themes
\moderncvstyle{banking}                            % style options are 'casual' (default), 'classic', 'oldstyle' and 'banking'
\moderncvcolor{blue}                                % color options 'blue' (default), 'orange', 'green', 'red', 'purple', 'grey' and 'black'
%\renewcommand{\familydefault}{\sfdefault}         % to set the default font; use '\sfdefault' for the default sans serif font, '\rmdefault' for the default roman one, or any tex font name
%\nopagenumbers{}                                  % uncomment to suppress automatic page numbering for CVs longer than one page

% character encoding
\usepackage[utf8]{inputenc}                       % if you are not using xelatex ou lualatex, replace by the encoding you are using
%\usepackage{CJKutf8}                              % if you need to use CJK to typeset your resume in Chinese, Japanese or Korean

% adjust the page margins
\usepackage[scale=0.8]{geometry}
%\setlength{\hintscolumnwidth}{3cm}                % if you want to change the width of the column with the dates
%\setlength{\makecvtitlenamewidth}{10cm}           % for the 'classic' style, if you want to force the width allocated to your name and avoid line breaks. be careful though, the length is normally calculated to avoid any overlap with your personal info; use this at your own typographical risks...

% personal data
\name{Daniel}{Tuzes}
\title{Avaloq cover letter}                               % optional, remove / comment the line if not wanted
\address{Pázmány Péter stny 1/A 4.71}{1117 Budapest}{Hungary}% optional, remove / comment the line if not wanted; the "postcode city" and and "country" arguments can be omitted or provided empty
\phone[mobile]{+36~70~335~8043}                   % optional, remove / comment the line if not wanted
\email{tuzes@metal.elte.hu}                               % optional, remove / comment the line if not wanted
\homepage{metal.elte.hu/\textasciitilde tuzes/}                         % optional, remove / comment the line if not wanted


%----------------------------------------------------------------------------------
%            content
%----------------------------------------------------------------------------------
\begin{document}
%-----       letter       ---------------------------------------------------------
% recipient data
\recipient{Avaloq Evolution AG}{Allmendstrasse 140\\8027 Zürich\\Switzerland}
\date{October 28, 2019}
\opening{Dear Michelle Hiestand,}
\closing{Yours faithfully,}
\makelettertitle

I was only four by then when I already knew, I want to be a scientist. This May, I successfully finished my Ph.D.\ in the field of physics, but during my training program I recognised, it is not the subject of investigation that makes a worker scientist, but the methodology and approach one uses, and there is place for critical thinking in the industry too.

I found your "Software Engineer - Taxes" open position and realised that I do related type of tasks as a researcher. I strongly believe that it is the business sector that lags most behind its possibilities by not using novel techniques from the fields of sciences. Therefore I suppose I could make the most of my knowledge if I capitalized it in the mentioned position.

I wrote numerous programs during my Ph.D.\ that not only helped the research but those projects were elementary based on computer simulations. In one of them, I reconstructed a simulation program based on an article describing a physical model. I checked the statistical behaviour of my program and compared it with the old results before I extended it with numerous new features and implemented new techniques to improve code efficiency. With the new code we were able to make fifty times larger simulations, which was a remarkable advancement on that field.

In another project I started to work on an existing code that seemed to be working, but the statistical data were not satisfactory. After weeks of investigation I found couple of bugs in the code preventing the program from producing the expected numbers. After debugging, the code was able to deliver the predicted theoretical expectations. Further efficiency improvements and interface development were required to be able to run statistical amount of simulations. With a new analysis technique I implemented, I was able to precisely evaluate the results and widely support the theory.

I used to be the "annoying kid" in the school with my never ending questions but my eagerness and lack of fear to ask became one of my strengths. My proactive thinking prevents many unnecessary mistakes not only in my tasks but in our projects too. I am always open to learn new tricks and techniques and I am also ready to face critics. I am a group player not only by accepting different roles and get on with colleagues but also by arising cohesive forces.  It often happens that I am the link between the project core and the remote, genius researcher.

I am willing to know more details on the projects you are working on, the working conditions and environment you are using and I am also happy to serve you further details beyond my résumé.

\makeletterclosing

\end{document}
