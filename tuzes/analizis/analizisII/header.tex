\usepackage{environ} % azért kell, mert egyes környezeteket el akarok rejteni
\usepackage{amsthm}
\usepackage{newunicodechar} % unicode-ban egyszerűbb tárolni, viszont a latex alapból nem ismeri ezeket
\newunicodechar{■}{\qed}
\newunicodechar{∎}{\qed}
\newunicodechar{↣}{\rightarrowtail}

\renewcommand{\contentsname}{Tartalomjegyzék}


\definecolor{sotetvoros}{HTML}{900000} % a színekre így hivatkozok
\definecolor{sotetzold}{HTML}{009900} % a színekre így hivatkozok
\usepackage{tcolorbox} % a def, tétel, bizonyítás, állítás, megjegyzés körüli színes kerethez.
\tcbuselibrary{most} % a def, tétel, bizonyítás, állítás, megjegyzés körüli színes kerethez.

%
% Saját környezetek definiálása, hogy a definíciók, tételek, bizonyítások és megjegyzések külön,
% de egységesen formázhatóak legyenek. A NewEnviron azért kell, hogy egyes környezetek
% elrejthetőek legyenek. Pl. azon extra definíciók, tételek, bizonyítások, amelyek a régi
% jegyzetben benne voltak, de most már nem szükségesek. Elrejthető továbbá az "ajánló" rovat,
% amely nem a tananyaghoz kapcsolódó vicces, közéleti, de a természettudományokhoz, vagy az
% emberi kultúrkincshez valahogy másképp hozzájárul.
%

\newenvironment{definicio}{}{}
\tcolorboxenvironment{definicio}{blanker,breakable,left=5mm, right=5mm,
before skip=10pt,after skip=10pt,borderline west={1.5mm}{0pt}{black},borderline east={1.5mm}{0pt}{black}}

\NewEnviron{def_extra}{
\begin{definicio} % ennek és az alábbi sornak a kikommentelésel eltűnik az apró oldalsó színes vonal a régi, már nem szükséges tartalomról
%\BODY
\end{definicio} % ennek és az alábbi sornak a kikommentelésel eltűnik az apró oldalsó színes vonal a régi, már nem szükséges tartalomról
}


\newenvironment{tetel}{}{}
\tcolorboxenvironment{tetel}{blanker,breakable,left=5mm,right=5mm,
before skip=10pt,after skip=10pt,borderline west={1.5mm}{0pt}{sotetvoros},borderline east={1.5mm}{0pt}{sotetvoros}}

\NewEnviron{tetel_extra}{
\begin{tetel} % ennek és az alábbi sornak a kikommentelésel eltűnik az apró oldalsó színes vonal a régi, már nem szükséges tartalomról
%\BODY
\end{tetel} % ennek és az alábbi sornak a kikommentelésel eltűnik az apró oldalsó színes vonal a régi, már nem szükséges tartalomról
}


\newenvironment{bizonyitas}{}{}
\tcolorboxenvironment{bizonyitas}{blanker,breakable,left=5mm,right=5mm,
before skip=10pt,after skip=10pt,borderline west={1.5mm}{0pt}{sotetvoros},borderline east={1.5mm}{0pt}{sotetvoros}}

\NewEnviron{biz_extra}{
\begin{bizonyitas} % ennek és az alábbi sornak a kikommentelésel eltűnik az apró oldalsó színes vonal a régi, már nem szükséges tartalomról
%\BODY
\end{bizonyitas} % ennek és az alábbi sornak a kikommentelésel eltűnik az apró oldalsó színes vonal a régi, már nem szükséges tartalomról
}

\newenvironment{allitas}{}{}
\tcolorboxenvironment{allitas}{blanker,breakable,left=5mm,right=5mm,
before skip=10pt,after skip=10pt,borderline west={0.75mm}{0pt}{sotetvoros},borderline east={0.75mm}{0pt}{sotetvoros}}

\NewEnviron{all_extra}{
\begin{allitas} % ennek és az alábbi sornak a kikommentelésel eltűnik az apró oldalsó színes vonal a régi, már nem szükséges tartalomról
%\BODY
\end{allitas} % ennek és az alábbi sornak a kikommentelésel eltűnik az apró oldalsó színes vonal a régi, már nem szükséges tartalomról
}

\newenvironment{megjegyzes}{}{}
\tcolorboxenvironment{megjegyzes}{blanker,breakable,left=5mm,right=5mm,
before skip=10pt,after skip=10pt,borderline west={0.75mm}{0pt}{lightgray},borderline east={0.75mm}{0pt}{lightgray}}

\NewEnviron{megj_extra}{
\begin{megjegyzes} % ennek és az alábbi sornak a kikommentelésel eltűnik az apró oldalsó színes vonal a régi, már nem szükséges tartalomról
%\BODY
\end{megjegyzes} % ennek és az alábbi sornak a kikommentelésel eltűnik az apró oldalsó színes vonal a régi, már nem szükséges tartalomról
}


\newenvironment{pelda}{}{}
\tcolorboxenvironment{pelda}{blanker,breakable,left=5mm,right=5mm,
before skip=10pt,after skip=10pt,borderline west={1.5mm}{	0pt}{sotetzold},borderline east={1.5mm}{0pt}{sotetzold}}


\newenvironment{ajanlofig}{\begin{figure}\begin{center}}{
\end{center}\end{figure}}

\NewEnviron{ajanlo}{
\begin{samepage} % az alábbi sorok kikommentelésével eltűnnek az ajánlók
\footnotesize
\BODY
\normalsize
\end{samepage} % eddig kell kikommentelni, hogy az ajánlók eltűnjenek
}