
\documentclass[12pt,a4paper]{scrartcl}
\usepackage{lmodern}
\usepackage{amssymb,amsmath}

\usepackage[T1]{fontenc}
\usepackage[utf8]{inputenc}
\usepackage[magyar]{babel}

\usepackage{upquote}
\usepackage[]{microtype}
\usepackage[dvipsnames,svgnames*,x11names*]{xcolor}
\usepackage[unicode=true]{hyperref}
\hypersetup{
            pdftitle={Analízis II},
            pdfauthor={Izsák Ferenc; Tarcsay Zsigmond; Tüzes Dániel},
            colorlinks=true,
            linkcolor=Maroon,
            citecolor=Blue,
            urlcolor=Blue,
            breaklinks=true}

\usepackage{graphicx,grffile}
\makeatletter
\def\fps@figure{htbp}
\renewcommand\paragraph{\@startsection{paragraph}{4}{\z@}%
                                    {3.25ex \@plus1ex \@minus.2ex}%
                                    {0.5em} % meg kellett növelni, mert amúgy -0.2em volt itt, és kicsinek tűnt
                                    {\normalfont\normalsize\bfseries}}
\makeatother

\usepackage[normalem]{ulem}

\providecommand{\tightlist}{%
  \setlength{\itemsep}{0pt}\setlength{\parskip}{0pt}}
\setcounter{secnumdepth}{4}


\usepackage{environ} % azért kell, mert egyes környezeteket el akarok rejteni
\usepackage{amsthm}
\usepackage{newunicodechar} % unicode-ban egyszerűbb tárolni, viszont a latex alapból nem ismeri ezeket
\newunicodechar{■}{\qed}
\newunicodechar{∎}{\qed}
\newunicodechar{↣}{\rightarrowtail}

\renewcommand{\contentsname}{Tartalomjegyzék}


\definecolor{sotetvoros}{HTML}{900000} % a színekre így hivatkozok
\definecolor{sotetzold}{HTML}{009900} % a színekre így hivatkozok
\usepackage{tcolorbox} % a def, tétel, bizonyítás, állítás, megjegyzés körüli színes kerethez.
\tcbuselibrary{most} % a def, tétel, bizonyítás, állítás, megjegyzés körüli színes kerethez.

%
% Saját környezetek definiálása, hogy a definíciók, tételek, bizonyítások és megjegyzések külön,
% de egységesen formázhatóak legyenek. A NewEnviron azért kell, hogy egyes környezetek
% elrejthetőek legyenek. Pl. azon extra definíciók, tételek, bizonyítások, amelyek a régi
% jegyzetben benne voltak, de most már nem szükségesek. Elrejthető továbbá az "ajánló" rovat,
% amely nem a tananyaghoz kapcsolódó vicces, közéleti, de a természettudományokhoz, vagy az
% emberi kultúrkincshez valahogy másképp hozzájárul.
%

\newenvironment{definicio}{}{}
\tcolorboxenvironment{definicio}{blanker,breakable,left=5mm, right=5mm,
before skip=10pt,after skip=10pt,borderline west={1.5mm}{0pt}{black},borderline east={1.5mm}{0pt}{black}}

\NewEnviron{def_extra}{
\begin{definicio} % ennek és az alábbi sornak a kikommentelésel eltűnik az apró oldalsó színes vonal a régi, már nem szükséges tartalomról
%\BODY
\end{definicio} % ennek és az alábbi sornak a kikommentelésel eltűnik az apró oldalsó színes vonal a régi, már nem szükséges tartalomról
}


\newenvironment{tetel}{}{}
\tcolorboxenvironment{tetel}{blanker,breakable,left=5mm,right=5mm,
before skip=10pt,after skip=10pt,borderline west={1.5mm}{0pt}{sotetvoros},borderline east={1.5mm}{0pt}{sotetvoros}}

\NewEnviron{tetel_extra}{
\begin{tetel} % ennek és az alábbi sornak a kikommentelésel eltűnik az apró oldalsó színes vonal a régi, már nem szükséges tartalomról
%\BODY
\end{tetel} % ennek és az alábbi sornak a kikommentelésel eltűnik az apró oldalsó színes vonal a régi, már nem szükséges tartalomról
}


\newenvironment{bizonyitas}{}{}
\tcolorboxenvironment{bizonyitas}{blanker,breakable,left=5mm,right=5mm,
before skip=10pt,after skip=10pt,borderline west={1.5mm}{0pt}{sotetvoros},borderline east={1.5mm}{0pt}{sotetvoros}}

\NewEnviron{biz_extra}{
\begin{bizonyitas} % ennek és az alábbi sornak a kikommentelésel eltűnik az apró oldalsó színes vonal a régi, már nem szükséges tartalomról
%\BODY
\end{bizonyitas} % ennek és az alábbi sornak a kikommentelésel eltűnik az apró oldalsó színes vonal a régi, már nem szükséges tartalomról
}

\newenvironment{allitas}{}{}
\tcolorboxenvironment{allitas}{blanker,breakable,left=5mm,right=5mm,
before skip=10pt,after skip=10pt,borderline west={0.75mm}{0pt}{sotetvoros},borderline east={0.75mm}{0pt}{sotetvoros}}

\NewEnviron{all_extra}{
\begin{allitas} % ennek és az alábbi sornak a kikommentelésel eltűnik az apró oldalsó színes vonal a régi, már nem szükséges tartalomról
%\BODY
\end{allitas} % ennek és az alábbi sornak a kikommentelésel eltűnik az apró oldalsó színes vonal a régi, már nem szükséges tartalomról
}

\newenvironment{megjegyzes}{}{}
\tcolorboxenvironment{megjegyzes}{blanker,breakable,left=5mm,right=5mm,
before skip=10pt,after skip=10pt,borderline west={0.75mm}{0pt}{lightgray},borderline east={0.75mm}{0pt}{lightgray}}

\NewEnviron{megj_extra}{
\begin{megjegyzes} % ennek és az alábbi sornak a kikommentelésel eltűnik az apró oldalsó színes vonal a régi, már nem szükséges tartalomról
%\BODY
\end{megjegyzes} % ennek és az alábbi sornak a kikommentelésel eltűnik az apró oldalsó színes vonal a régi, már nem szükséges tartalomról
}


\newenvironment{pelda}{}{}
\tcolorboxenvironment{pelda}{blanker,breakable,left=5mm,right=5mm,
before skip=10pt,after skip=10pt,borderline west={1.5mm}{	0pt}{sotetzold},borderline east={1.5mm}{0pt}{sotetzold}}


\newenvironment{ajanlofig}{\begin{figure}\begin{center}}{
\end{center}\end{figure}}

\NewEnviron{ajanlo}{
\begin{samepage} % az alábbi sorok kikommentelésével eltűnnek az ajánlók
\footnotesize
\BODY
\normalsize
\end{samepage} % eddig kell kikommentelni, hogy az ajánlók eltűnjenek
}


\title{Analízis II}
\providecommand{\subtitle}[1]{}
\subtitle{Előadásjegyzet fizikusoknak matematikusoktól}
\author{Izsák Ferenc \and Tarcsay Zsigmond \and Tüzes Dániel}
\providecommand{\institute}[1]{}
\institute{ELTE}
\date{}

\begin{document}
\maketitle


\setcounter{tocdepth}{6}
\tableofcontents

Ajánlott irodalom:
\href{http://www.typotex.hu/konyv/valos_fuggvenyek_es_fuggvenysorok}{Szőkefalvi
Nagy Béla: Valós függvények és függvénysorok}

\hypertarget{binomialis-sor}{%
\section{\texorpdfstring{\textbf{Binomiális
sor}}{Binomiális sor}}\label{binomialis-sor}}

Emlékeztető:
\(\left( {1 + x} \right)^{n} = {\sum\limits_{k = 0}^{n}{\begin{pmatrix} n \\ k \\ \end{pmatrix}x^{k}}}\),
ahol \(n \in {\mathbb{N}}\) (binomiális tétel alapján). Legyen
\(f\left( x \right): = \left( {1 + x} \right)^{\alpha},\alpha \in {\mathbb{R}},x > - 1\)!
Írjuk fel ennek az \(f\) függvénynek a 0 körüli Taylor-sorát!

\[\begin{array}{ll}
{f\left( x \right) = \left( {1 + x} \right)^{\alpha}} & {f\left( 0 \right) = 1} \\
{f^{\prime}\left( x \right) = \alpha\left( {1 + x} \right)^{\alpha - 1}} & {f^{\prime}\left( 0 \right) = \alpha} \\
{f^{''}\left( x \right) = \alpha\left( {\alpha - 1} \right)\left( {1 + x} \right)^{\alpha - 2}} & {f^{''}\left( 0 \right) = \alpha\left( {\alpha - 1} \right)} \\
 \vdots & \vdots \\
{f^{(j)}\left( x \right) = \alpha\left( {\alpha - 1} \right) \cdot \ldots \cdot \left( {\alpha - j + 1} \right)\left( {1 + x} \right)^{\alpha - j}} & {f^{(j)}\left( 0 \right) = \alpha\left( {\alpha - 1} \right) \cdot \ldots \cdot \left( {\alpha - j + 1} \right)} \\
\end{array}\]

Ekkor \(f\) függvény Taylor sora 0 körül:
\[\mathop \sum \limits_{j = 0}^\infty  \frac{{{f^{\left( j \right)}}\left( 0 \right)}}{{j!}}{x^j} = \mathop \sum \limits_{j = 0}^\infty  \frac{{\alpha \left( {\alpha  - 1} \right) \cdot  \ldots  \cdot \left( {\alpha  - j + 1} \right)}}{{j!}}{x^j}.\]
Jelölés: tetszőleges valós \(\alpha\) esetén
\(\begin{pmatrix} \alpha \\ j \\ \end{pmatrix}: = \frac{\alpha\left( {\alpha - 1} \right) \cdot \ldots \cdot \left( {\alpha - j + 1} \right)}{j!}\),
ezt használva
\({\sum\limits_{j = 0}^{\infty}\frac{f^{(j)}\left( 0 \right)}{j!}}x^{j} = {\sum\limits_{j = 0}^{\infty}{\begin{pmatrix} \alpha \\ j \\ \end{pmatrix}x^{j}}}\).
Most belátjuk, hogy a kapott sor konvergencia sugara 1. Ehhez célszerű
használni a hányados kritériumot.
\(a_{j}: = \begin{pmatrix} \alpha \\ j \\ \end{pmatrix}x^{j}\), ekkor
\[\frac{{{a_{j + 1}}}}{{{a_j}}} = \frac{{\frac{{\alpha (\alpha  - 1)\cdot \ldots \cdot(\alpha  - j)}}{{(j + 1)!}}}}{{\frac{{\alpha (\alpha  - 1)\cdot \ldots \cdot(\alpha  - j + 1)}}{{j!}}}}\frac{{{x^{j + 1}}}}{{{x^j}}} = \frac{{\alpha  - j}}{{j + 1}}x \Rightarrow \left| {\frac{{{a_{j + 1}}}}{{{a_j}}}} \right| = \underbrace {\left| {\frac{{\alpha  - j}}{{j + 1}}} \right|}_{ \to 1{\text{ha}}j \to \infty }\cdot\left| x \right|,\]
ezért \(\left| x \right| < 1\) esetén az abszolút értékekből álló sorra
valóban teljesül a hányados kritérium, így a sor konvergens, mert
abszolút konvergens is.

\begin{allitas}

Állítás:\\
\(\left| x \right| < 1\) esetén a Taylor sor előállítja \(f\)-et, vagyis
\(\left( {1 + x} \right)^{\alpha} = {\sum\limits_{j = 0}^{\infty}{\begin{pmatrix} \alpha \\ j \\ \end{pmatrix}x^{j}}}\).

\end{allitas}

\begin{bizonyitas}

Bizonyítás:\\
Legyen \(f\left( x \right): = \left( {1 + x} \right)^{\alpha}\), illetve
\(g\left( x \right): = {\sum\limits_{j = 0}^{\infty}{\begin{pmatrix} \alpha \\ j \\ \end{pmatrix}x^{j}}}\),
így
\(f^{\prime}\left( x \right) = \frac{\alpha}{1 + x}f\left( x \right)\),
sőt, mivel \(g\) hatványsorról előbb láttuk be, hogy konvergens
\(\left| x \right| < 1\) esetén, így hatványsorról lévén szó, a sor és a
tagonkénti deriválással nyert sor egyenletesen konvergens minden 1-nél
kisebb sugarú intervallumban, tehát a deriválást és az összegzést
felcserélhetjük, vagyis
\(g^{\prime}\left( x \right) = \frac{\alpha}{1 + x}g\left( x \right)\).
Látjuk, hogy \(f\left( 0 \right) = 1\) és \(g\left( 0 \right) = 1\),
ebből további átalakításokkal és tételek felhasználásával következik,
hogy \(f\left( x \right) = g\left( x \right)\) :
\(\frac{d}{dx}\left\lbrack {\ln g\left( x \right)} \right\rbrack = \frac{g'\left( x \right)}{g\left( x \right)} = \frac{\alpha}{1 + x} = \frac{d}{dx}\left\lbrack {\ln\left( {1 + x} \right)^{\alpha}} \right\rbrack\),
melyből következik, hogy
\(\ln g\left( x \right) = \ln\left( {1 + x} \right)^{\alpha} + c\).
\(c = 0\), mivel az egyenlőség \(x = 1\) esetén is fenn kell állnia.
Ezekből már következik, hogy
\(g\left( x \right) = \left( {1 + x} \right)^{\alpha}\).

\end{bizonyitas}

\begin{pelda}

Alkalmazás:

\begin{itemize}
\tightlist
\item
  \(\sqrt{1 + x} = \left( {1 + x} \right)^{1/2} = {\sum\limits_{j = 0}^{\infty}{\begin{pmatrix} {1/2} \\ j \\ \end{pmatrix}x^{j}}}\)
\item
  \(\arcsin^{\prime}\left( x \right) = \frac{1}{\sqrt{1 - x^{2}}} = \left( {1 - x^{2}} \right)^{- 1/2} = {\sum\limits_{j = 0}^{\infty}{\begin{pmatrix} {- 1/2} \\ j \\ \end{pmatrix}\left( {- x^{2}} \right)^{j}}}\)
\item
  \(g\left( x \right) = \ln\left( {1 + x} \right)\) sorfejtése 0 körül:
  \(g^{\prime}\left( x \right) = \frac{1}{1 + x} = {\sum\limits_{j = 0}^{\infty}{\left( {- 1} \right)^{j}x^{j}}}\).
  Mivel a hatványsor mindig egyenletesen konvergens a
  konvergencia-intervallumnál kisebb intervallumon, ezért integrálhajuk
  minkét oldalt 0-tól \(\xi\)-ig, vagyis \(\left| \xi \right| < 1\)
  esetén \[\begin{gathered}
    \mathop \smallint \limits_0^\xi  g'\left( x \right)dx = \mathop \smallint \limits_0^\xi  \mathop \sum \limits_{j = 0}^\infty  {\left( { - 1} \right)^j}{x^j}dx = \mathop \sum \limits_{j = 0}^\infty  {\left( { - 1} \right)^j}\mathop \smallint \limits_0^\xi  {x^j}dx \\ 
     \Downarrow  \\ 
    \ln \left( {1 + \xi } \right) = \mathop \sum \limits_{j = 0}^\infty  {\left( { - 1} \right)^j}\frac{{{\xi ^{j + 1}}}}{{j + 1}}. \\ 
  \end{gathered} \]
\end{itemize}

\end{pelda}

\hypertarget{taylor-formula-tobbvaltozos-fuggvenyekre}{%
\subsection{Taylor formula többváltozós
függvényekre}\label{taylor-formula-tobbvaltozos-fuggvenyekre}}

Legyen \(X\) normált tér \(\left. f:X\rightarrow{\mathbb{R}} \right.\)
és \(k\)+1-szer differenciálható az \(a \in X\) egy \(\rho\) sugarú
környezetében. Ekkor egy \(h \in X,\left\| h \right\| < \rho\) esetén
szeretnénk kifejezni az \(a\)+\(h\) helyen a függvényértéket az \(a\)
helyen felvettel:
\(f\left( {a + h} \right) = f\left( a \right) + \ldots\). Mi kerül ...
helyére?

Legyen \(\delta > 0,t \in \left( {- \delta,1 + \delta} \right)\),
\(\left. g:{\mathbb{R}}\rightarrow X,g\left( t \right) = a + t \cdot h \right.\),
valamint legyen olyan
\(\left. \phi:{\mathbb{R}}\rightarrow{\mathbb{R}}\operatorname{,\ hogy}\phi\left( t \right) = \left( {f \circ g} \right)\left( t \right) = f\left( {a + th} \right) \right.\)
és \(\phi\) függvény \(k\)+1-szer differenciálható
\(\left( {- \delta,\delta + 1} \right)\) intervallumon elég kis
\(\delta > 0\) esetén. Megjegyezzük, hogy ekkor
\(g'\left( t \right) \in \underbrace {L\left( {\mathbb{R},X} \right)}_{t \mapsto th{\text{ leképezés}}} \Leftrightarrow h \in X\),
sőt, ezt az azonosítást elhagyva: \(g^{\prime}\left( t \right): = h\).

Alkalmazzuk a Taylor formulát \(\phi\)-re:
\[\phi\left( 1 \right) = \phi\left( 0 \right) + \frac{\phi^{\prime}\left( 0 \right)}{1!}1 + \frac{\phi^{''}\left( 0 \right)}{2!}1^{2} + ... + \frac{\phi^{(k)}\left( 0 \right)}{k!}1^{k} + \frac{\phi^{({k + 1})}\left( \tau \right)}{\left( {k + 1} \right)!}1^{k + 1,}\]
ahol \(0 < \tau < 1\).

Ennek első tagjáról tudjuk, hogy
\(\phi\left( 0 \right) = f\left( a \right)\). Mi a többi?
\(\phi\left( t \right) = f\left( {a + th} \right)\), illetve
\(\phi = f \circ g\). Ekkor a deriváltja:
\(\phi '\left( t \right) = \underbrace {f'\left( {g\left( t \right)} \right)}_{ \in L\left( {X,\mathbb{R}} \right)}\underbrace {g'\left( t \right)}_{ \in L\left( {\mathbb{R},X} \right)} \in L\left( {\mathbb{R},\mathbb{R}} \right)\),
\(g^{\prime}\left( t \right) = h\), így
\[\phi '\left( t \right) = f'\left( {g\left( t \right)} \right)g'\left( t \right) = \underbrace {f'\left( {a + th} \right)}_{ \in L\left( {X,\mathbb{R}} \right)}\underbrace h_{ \in X} \in \mathbb{R},\]
ezért
\(\phi^{\prime}\left( 0 \right) = f^{\prime}\left( a \right)h \in {\mathbb{R}}\).

Továbbá
\[\phi ''\left( t \right) = \underbrace {\left[ {f''\left( {a + th} \right)h} \right]}_{ \in L\left( {X,\mathbb{R}} \right)}h = f''\left( {a + th} \right)\left( {h,h} \right),\]
így
\(\phi ''\left( 0 \right) = f''\left( a \right)\underbrace {\left( {h,h} \right)}_{ \in X \times X} \in \mathbb{R}\)
.

Tovább folytatva:
\[{\phi ^{\left( k \right)}}\left( t \right) = {f^{\left( k \right)}}\left( {a + th} \right)\left( {h,h,...,h} \right) \in \mathbb{R},\]ezért
\(\phi^{(k)}\left( 0 \right) = f^{(k)}\left( 0 \right)\left( {h,h,...,h} \right)\).
\[\begin{array}{*{20}{c}}
  {\phi \left( 1 \right) = \phi \left( 0 \right) + \frac{{\phi '\left( 0 \right)}}{{1!}}1 + \frac{{\phi ''\left( 0 \right)}}{{2!}}{1^2} + ... + \frac{{{\phi ^{\left( k \right)}}\left( 0 \right)}}{{k!}}{1^k} + \frac{{{\phi ^{\left( {k + 1} \right)}}\left( \tau  \right)}}{{\left( {k + 1} \right)!}}{1^{k + 1,}}} \\ 
   \Downarrow  \\ 
  \begin{aligned}
  f\left( {a + h} \right) =  & f\left( a \right) \\ 
   &  + \frac{{f'\left( a \right)}}{{1!}}h \\ 
   &  + \frac{{f''\left( a \right)\left( {h,h} \right)}}{{2!}} \\ 
   &  + ... \\ 
   &  + \frac{{{f^{\left( k \right)}}\left( a \right)\left( {h,h,...,h} \right)}}{{k!}} \\ 
   &  + \frac{{{f^{\left( {k + 1} \right)}}\left( {a + \tau h} \right)\left( {h,h,...,h} \right)}}{{\left( {k + 1} \right)}}, \\ 
\end{aligned}  
\end{array}\]ahol \(0 < \tau < 1\).

Speciális eset, mikor \(k = 1\), \(X = {\mathbb{R}}^{n}\),
\(\left. f:{\mathbb{R}}^{n}\rightarrow{\mathbb{R}} \right.\). Ekkor a
deriváltak: \[\begin{array}{l}
{f^{\prime}\left( a \right) = \left( {\left( {\partial_{1}f} \right)\left( a \right),\left( {\partial_{2}f} \right)\left( a \right),...,\left( {\partial_{n}f} \right)\left( a \right)} \right)} \\
{f^{''}\left( a \right) = \left( \begin{array}{llll}
{\left( {\partial_{1}^{2}f} \right)\left( a \right)} & {\left( {\partial_{2}\partial_{1}f} \right)\left( a \right)} & ... & {\left( {\partial_{n}\partial_{1}f} \right)\left( a \right)} \\
{\left( {\partial_{1}\partial_{2}f} \right)\left( a \right)} & {\left( {\partial_{2}^{2}f} \right)\left( a \right)} & ... & {\left( {\partial_{n}\partial_{2}f} \right)\left( a \right)} \\
\end{array} \right),} \\
\end{array}\]\\
így a függvény sorfejtése első rendig, a másodrendű maradéktaggal:
\(f\left( {a + h} \right) = f\left( a \right) + \frac{1}{1!}{\sum\limits_{j = 1}^{n}{\left( {\partial_{j}f} \right)\left( a \right)h_{j}}} + \frac{1}{2!}{\sum\limits_{j,k = 1}^{n}{\left( {\partial_{j}\partial_{k}f} \right)\left( {a + \tau h} \right)h_{j}}}h_{k}\),
ahol
\(h = \left( {h_{1},h_{2},...,h_{n}} \right) \in {\mathbb{R}}^{n}\).

\begin{megjegyzes}

Megjegyzés:\\
Legyenek \(X\), \(Y\) normált terek! Bebizonyítható, hogy ha
\(\left. f:X\rightarrow Y \right.\) és \(k\)-szor differenciálható
\(B_{\delta}\left( a \right)\)-n, akkor a Taylor formula
\(\left\| h \right\| < \delta\) esetén:
\(f\left( {a + h} \right) = f\left( a \right) + \frac{f^{\prime}\left( a \right)}{1!}h + ... + \frac{f^{(k)}\left( a \right)\left( {h,h,...,h} \right)}{k!} + R_{k}\),
ahol \(R_{k}\) a maradéktag, amelyre
\(\left\| R_{k} \right\| \leq \frac{\left\| h \right\|^{k}}{k!} \cdot \sup\limits_{\xi \in B_{\delta}{(a)}}\left\| {f^{(k)}\left( \xi \right) - f^{(k)}\left( a \right)} \right\|\).

\end{megjegyzes}

\hypertarget{tobbvaltozos-fuggvenyek-lokalis-szelsoerteke}{%
\subsection{Többváltozós függvények lokális
szélsőértéke}\label{tobbvaltozos-fuggvenyek-lokalis-szelsoerteke}}

\begin{definicio}

Definíció:\\
Legyen \(X\) normált tér, \(\left. f:X\rightarrow{\mathbb{R}} \right.\),
és értelmezve van az \(a \in X\) pont egy környezetében. Azt mondjuk,
hogy \(f\)-nek \(a\)-ban lokális minimuma van, ha
\(\left. \exists\delta > 0:x \in B_{\delta}\left( a \right)\Rightarrow f\left( x \right) \geq f\left( a \right) \right.\).
Ha
\(\left. x \in B_{\delta}\left( a \right)\backslash\left\{ a \right\}\Rightarrow f\left( x \right) > f\left( a \right) \right.\),
akkor szigorú lokális minimumról beszélünk.

\end{definicio}

\begin{tetel}

Tétel:\\
Tfh \(f\) differenciálható az \(a\)-ban és \(f\)-nek \(a\)-ban lokális
szélsőértéke van (minimuma vagy maximuma),
\(\left. \Rightarrow f^{\prime}\left( a \right) = 0 \right.\) (ahol
\(f^{\prime}\left( a \right) \in L\left( {X,{\mathbb{R}}} \right)\)).

\end{tetel}

\begin{bizonyitas}

Bizonyítás:\\
Legyen \(h \in X\) tetszőleges rögzített pont. Belátjuk, hogy
\(\underbrace {f'\left( a \right)}_{ \in L\left( {X,\mathbb{R}} \right)}h = 0 \in \mathbb{R}\).
Mivel \(f\) differenciálható \(a\)-ban, ezért elég kicsi \(t\) esetén
\(f\left( {a + th} \right) - f\left( a \right) = f^{\prime}\left( a \right)\left( {t \cdot h} \right) + \eta\left( {t \cdot h} \right)\),
ahol
\(\lim\limits_{t\rightarrow 0}\frac{\left| {\eta\left( {th} \right)} \right|}{\left\| {th} \right\|} = 0 = \lim\limits_{t\rightarrow 0}\frac{\left| {\eta\left( {th} \right)} \right|}{\left| t \right| \cdot \left\| h \right\|}\),
ahol \(\left\| h \right\| \neq 0\), ezért
\(\lim\limits_{t\rightarrow 0}\frac{\left| {\eta\left( {th} \right)} \right|}{\left| t \right|} = 0\).
Ha \(f^{\prime}\left( a \right)h \neq 0\) lenne, pl.
\(f^{\prime}\left( a \right)h > 0\) (\(h\) rögzített), akkor
\(\lim\limits_{t\rightarrow 0}\frac{f\left( {a + th} \right) - f\left( a \right)}{t} = f^{\prime}\left( a \right)h > 0\).
Ekkor
\(\left. \exists\delta_{1} > 0,0 < t < \delta_{1}\Rightarrow f\left( {a + th} \right) - f\left( a \right) > 0 \right.\),
illetve
\(\left. \exists\delta_{2} > 0, - \delta_{2} < t < 0\Rightarrow f\left( {a + th} \right) - f\left( a \right) < 0 \right.\),
ez utóbbi kettő pedig ellentmondás, merthogy szélsőérték esetén
\(f\left( {a + th} \right) - f\left( a \right)\) előjele ugyanaz kell,
hogy legyen.

\end{bizonyitas}

\begin{definicio}

Definíció:\\
Legyen \(X\) normált tér,
\(\left. g:X \times X\rightarrow{\mathbb{R}} \right.\), (folytonos)
bilineáris leképezés. Azt mondjuk, hogy

\begin{itemize}
\tightlist
\item
  \(g\) pozitív definit, ha
  \(g\left( {h,h} \right) > 0,\forall h \in X\text{\textbackslash}\left\{ 0 \right\}\),
\item
  negatív definit, ha
  \(g\left( {h,h} \right) < 0,\forall h \in X\text{\textbackslash}\left\{ 0 \right\}\),
\item
  pozitív szemidefinit, ha
  \(g\left( {h,h} \right) \geq 0,\forall h \in X\),
\item
  negatív szemidefinit, ha
  \(g\left( {h,h} \right) \leq 0,\forall h \in X\),
\item
  szigorúan pozitív definit, ha \(\exists c > 0\) állandó, hogy
  \(g\left( {h,h} \right) \geq c\left\| h \right\|^{2},\forall h \in X\).
\end{itemize}

\end{definicio}

\begin{megjegyzes}

Megjegyzés:\\
Ha \(X = {\mathbb{R}}^{n}\), ekkor abból, hogy \(g\) pozitív definit,
következik, hogy szigorúan pozitív definit. Végtelen dimenziós
vektorterekben általában ez nem igaz. Előbbi igazolása: legyen
\(X = {\mathbb{R}}^{n}\), ekkor tekintsük az
\(S_{1}: = \left\{ {x \in {\mathbb{R}}^{n}:\left| x \right| = 1} \right\}\)
halmazt, ekkor ez sorozatkompakt (mert korlátos és zárt). Legyen
\(G\left( h \right): = g\left( {h,h} \right)\)! Ekkor \(G\) függvény
folytonos. \(h \in S_{1}\), így
\(\left. G:S_{1}\rightarrow{\mathbb{R}} \right.\) folytonos, \(S_{1}\)
sorozatkompakt, ezért \(G\) felveszi az infinimumát (minimumát), vagyis
\(\exists h_{0} \in S_{1}:G\left( h \right) \geq G\left( h_{0} \right),h \in S_{1}\).
Mivel \(g\) pozitív definit, \(c: = G\left( h_{0} \right) > 0\), ahol
\(h_{0} \neq 0\). Ekkor
\(x \in X\text{\textbackslash}\left\{ 0 \right\}\) esetén
\[g\left( {x,x} \right) = g\left( {\frac{x}{{\left\| x \right\|}}\left\| x \right\|,\frac{x}{{\left\| x \right\|}}\left\| x \right\|} \right) = {\left\| x \right\|^2}\underbrace {g\left( {\frac{x}{{\left\| x \right\|}},\frac{x}{{\left\| x \right\|}}} \right)}_{ \geqslant c > 0} \geqslant c{\left\| x \right\|^2}.\]

\end{megjegyzes}

\begin{megjegyzes}

Megjegyzés:\\
\(X = {\mathbb{R}}^{n}\) esetén egy
\(\left. {\mathbb{R}}^{n} \times {\mathbb{R}}^{n}\rightarrow{\mathbb{R}} \right.\)
bilineáris leképezés egy négyzetes mátrixszal adható meg,
\(A: = \begin{pmatrix} a_{11} & a_{12} & \cdots & a_{1n} \\ a_{21} & a_{22} & \cdots & a_{2n} \\  \vdots & \vdots & \ddots & \vdots \\ a_{n1} & a_{n2} & \cdots & a_{nn} \\ \end{pmatrix}\).
Ha \(a_{jk} = a_{kj}\) -- vagyis ha a mátrix szimmetrikus --, akkor ha
\(A\) összes sajátértéke nagyobb mint 0, akkor \(A\) pozitív definit,
sőt, szigorúan pozitív definit.

\end{megjegyzes}

\begin{tetel}

Tétel:\\
Tfh \(f\) kétszer differenciálható az \(a \in X\) egy környezetében
(\(X\) normált tér) és \(f^{''} \in C\left( a \right)\).

\begin{enumerate}
\def\labelenumi{\arabic{enumi}.}
\tightlist
\item
  Ha \(f\)-nek \(a\)-ban lokális minimuma van
  \(\left. \Rightarrow f^{\prime}\left( a \right) = 0 \right.\), és
  \(f^{''}\left( a \right)\) pozitív szemidefinit.
\item
  Ha \(f^{\prime}\left( a \right) = 0\) és \(f^{''}\left( a \right)\)
  szigorú pozitív definit, akkor \(f\)-nek \(a\)-ban szigorú lokális
  minimuma van.
\end{enumerate}

\end{tetel}

\begin{bizonyitas}

Bizonyítás:\\
Alkalmazzuk a Taylor formulát az
\(\left. f:X\rightarrow{\mathbb{R}} \right.\) függvényre a 2.
deriváltig. Legyen \(h \in X,t \in {\mathbb{R}},\left| t \right|\) elég
kicsi, ekkor
\(f\left( {a + th} \right) = f\left( a \right) + \frac{f^{\prime}\left( a \right)}{1!}th + \frac{f^{''}\left( {a + \tau th} \right)}{2!}\left( {th,th} \right)\),
ahol \(\tau\) alkalmasan választott, valamilyen \(0 < \tau < 1\) szám.

\begin{enumerate}
\def\labelenumi{\arabic{enumi}.}
\tightlist
\item
  Tfh \(f\)-nek \(a\)-ban lokális minimuma van. Tudjuk, hogy ekkor
  \(f^{\prime}\left( a \right) = 0\), így \[\begin{aligned}
    0 \leqslant  & \frac{{f\left( {a + th} \right) - f\left( a \right)}}{{{t^2}}} \\ 
     =  & \frac{{f''\left( {a + \tau th} \right)}}{{2!}}\left( {h,h} \right) \\ 
     =  & \frac{{f''\left( a \right)}}{{2!}}\left( {h,h} \right) + \underbrace {\left[ {\frac{{f''\left( {a + \tau th} \right)}}{{2!}} - \frac{{f''\left( a \right)}}{{2!}}} \right]\left( {h,h} \right)}_{{\text{bizbe:\;}} \to 0{\text{\;ha\;}}t \to 0} \to \frac{{f''\left( a \right)}}{{2!}}\left( {h,h} \right), \\ 
  \end{aligned} \] ugyanis ekkor \(\left. t\rightarrow 0 \right.\)
  esetén \(\left. a + \tau th\rightarrow a \right.\),
  \(f^{''} \in C\left( a \right)\), és \[\begin{gathered}
    \left| {\frac{1}{2}\left[ {f''\left( {a + \tau th} \right) - f''\left( a \right)} \right]\left( {h,h} \right)} \right| \leqslant \frac{1}{2}\underbrace {\left\| {f''\left( {a + \tau th} \right) - f''\left( a \right)} \right\|}_{ \to 0{\text{\;ha\;}}t \to 0{\text{\;mert\;}}f'' \in C\left( a \right)} \cdot \underbrace {\left\| {h,h} \right\|}_{{\text{rögz}}} \\ 
     \Downarrow  \\ 
    \frac{{f''\left( a \right)}}{{2!}}\left( {h,h} \right) \geqslant 0. \\ 
  \end{gathered} \]
\item
  Felhasználva, hogy \(f^{\prime}\left( a \right) = 0\),
  \[\begin{aligned}
    \frac{{f\left( {a + th} \right) - f\left( a \right)}}{{{t^2}}} =  & \frac{1}{2}f''\left( {a + \tau th} \right)\left( {h,h} \right) \\ 
     =  & \frac{1}{2}f''\left( a \right)\left( {h,h} \right) + \frac{1}{2}\left[ {f''\left( {a + \tau th} \right) - f''\left( a \right)} \right]\left( {h,h} \right) .\\ 
  \end{aligned} \] Legyen \(h \in X,\left\| h \right\|: = c_{1} > 0\)!
  Egyrészt
  \(f^{''}\left( a \right)\left( {h,h} \right) \geq c_{2}\left\| h \right\|^{2} = c_{1}{}^{2}c_{2} > 0,\forall h \in X\)
  mert \(f^{''}\) szigorú pozitív definit, másrészt
  \[\left| {\left[ {f''\left( {a + \tau th} \right) - f''\left( a \right)} \right]\left( {h,h} \right)} \right| \leqslant \underbrace {\left\| {f''\left( {a + \tau th} \right) - f''\left( a \right)} \right\|}_{ \to 0{\text{\;ha\;}}t \to 0,{\text{\;mert\;}}f'' \in C\left( a \right)} \cdot \underbrace {{{\left\| h \right\|}^2}}_{{\text{rögz}}},\]
  ami tart 0-hoz ha \(t\) tart 0-hoz, így
  \[\frac{f\left( {a + th} \right) - f\left( a \right)}{t^{2}} = \frac{1}{2}f^{''}\left( a \right)\left( {h,h} \right) + \frac{1}{2}\left\lbrack {f^{''}\left( {a + \tau th} \right) - f^{''}\left( a \right)} \right\rbrack\left( {h,h} \right) > 0,\]
  ha \(t\) elég kicsi.
\end{enumerate}

\end{bizonyitas}

\begin{ajanlo}

\begin{ajanlofig}

\href{https://xkcd.com}{\includegraphics[width=5.20833in,height=2.82292in]{wikipedian_protester.png}}

\end{ajanlofig}

Text

\end{ajanlo}

\hypertarget{implicit-fuggvenytetel}{%
\subsection{Implicit függvénytétel}\label{implicit-fuggvenytetel}}

Probléma: adott egy
\(\left. \Phi:{\mathbb{R}}^{2}\rightarrow{\mathbb{R}} \right.\)
függvény. Egy \(\Phi\left( {x,y} \right) = 0\) egyenlet milyen
feltételek mellet határoz meg egy \(y = f\left( x \right)\) függvényt?
Tekintsük a következő példákat!

\begin{itemize}
\tightlist
\item
  \(\Phi\left( {x,y} \right): = x^{2} + y^{2} - 1 = 0\), ennek egy kör
  pontjai felelnek meg. Plusz feltétel lehet, hogy a megoldás valamelyik
  pont egy környezetében legyen, de még ekkor is lehet 2 megoldás
  (\(\left( 1,0 \right)\) és \(\left( {- 1,0} \right)\) környezetében).
\item
  \(\Phi\left( {x,y} \right): = x^{2} + y^{2} + 1 = 0\), ennek viszont
  nincs megoldása.
\item
  \(\Phi\left( {x,y} \right): = y^{2} - x^{2} = 0\), ennek két egyenes
  tesz eleget. Ha még le is szűkítjük az értelmezési tartományt úgy,
  hogy csak az egyik egyenes egy része legyen megoldás, akkor ugyan
  lokálisan függvényünk lesz (vagyis egyértelmű lesz \(y\)), de ilyet
  nem tudnánk csinálni az origó környezetében, mert ott metszik egymást
  az egyenesek. Ez azzal függ össze, hogy
  \(\left. \partial_{2}\Phi\left( {x,y} \right) = 2y\Rightarrow\partial_{2}\Phi\left( 0,0 \right) = 0 \right.\).
\end{itemize}

\begin{tetel}

Tétel:\\
Legyen
\(\left. \Phi:{\mathbb{R}}^{n} \times {\mathbb{R}}\rightarrow{\mathbb{R}} \right.\)-be
képező függvény, amely értelmezve van és folytonos valamilyen
\(\left( {a,b} \right) \in {\mathbb{R}}^{n} \times {\mathbb{R}}\) pont
egy környezetében és \(\Phi\left( {a,b} \right) = 0\), továbbá
\(\exists\partial_{n + 1}\Phi\) és folytonos is \(\left( {a,b} \right)\)
egy környezetében, és
\(\partial_{n + 1}\Phi\left( {a,b} \right) \neq 0\). Ekkor létezik az
\(a\) pontnak olyan \(B_{r}\left( a \right)\), az \(b\) pontnak olyan
\(\left( {b - d,b + d} \right)\) környezete és
\(\left. f:B_{r}\left( a \right)\rightarrow{\mathbb{R}} \right.\)
folytonos függvény, hogy
\(\left\{ {\left( {x,y} \right):\Phi\left( {x,y} \right) = 0,x \in B_{r}\left( a \right),y \in \left( {b - d,b + d} \right)} \right\} = \left\{ {\left( {x,f\left( x \right)} \right):x \in B_{r}\left( a \right)} \right\}\).

\end{tetel}

\begin{bizonyitas}

Bizonyítás:

\begin{itemize}
\tightlist
\item
  Tekintsük az \(n = 1\) esetet (könnyebb szemléletesen látni)! Pl tfh
  \(\partial_{n + 1}\Phi\left( {a,b} \right) > 0\)! Mivel
  \(\partial_{n + 1}\Phi\) folytonos
  \(\left. \Rightarrow\left( {a,b} \right) \right.\)-nek van olyan
  környezete, ahol
  \(\left. \partial_{n + 1}\Phi\left( {x,y} \right) > 0\Rightarrow\forall x \in B_{r_{1}}\left( a \right) \right.\)
  rögzített \(x\) esetén
  \(\left. y\mapsto\Phi\left( {x,y} \right) \right.\) szigorú monoton
  nő.
  \(\left. \Phi\left( {a,b} \right) = 0,y\mapsto\Phi\left( {x,y} \right) \right.\)
  szigorúan monoton nő
  \(\left. \Rightarrow\Phi\left( {a,b - d} \right) < 0 < \Phi\left( {a,b + d} \right) \right.\).
  Mivel \(\Phi\) folytonos \(\left( {a,b + d} \right)\) és
  \(\left( {a,b - d} \right)\) pontok között
  \(\left. \Rightarrow\exists r:0 < r \leq r_{1} \right.\), hogy
  \(x \in B_{r}\left( a \right)\) esetén és
  \(\Phi\left( {x,b - d} \right) < 0 < \Phi\left( {x,b + d} \right)\).\\
  Alkalmazzuk a Bolzano-tételt rögzített \(x \in B_{r}\left( a \right)\)
  esetén \(\left. y\mapsto\Phi\left( {x,y} \right) \right.\) függvényre!
  A tétel szerint ekkor \(\exists f\left( x \right)\), hogy
  \(b - d < f\left( x \right) < b + d\) esetén
  \(\Phi\left( {x,f\left( x \right)} \right) = 0\). Mivel
  \(\left. y\mapsto\Phi\left( {x,y} \right) \right.\) szigorú monoton
  nő, \(f\left( x \right)\) egyértelmű.
\item
  Be kell látnunk még, hogy \(f\) folytonos \(B_{r}\left( a \right)\)-n.
  Legyen \(x_{0} \in B_{r}\left( a \right)\), és
  \(\left( {a,b} \right)\) helyett tekintsük az
  \(\left( {x_{0},f\left( x_{0} \right)} \right)\) pontot! Ekkor
  \(b - d < f\left( x_{0} \right) < b + d\). Azt szeretnénk belátni,
  hogy \(f\) folytonos \(x_{0}\)-ban. Legyen \(\varepsilon > 0\)
  tetszőleges szám! Az előző állítás miatt
  \(\exists\varepsilon':0 < \varepsilon' < \varepsilon\), hogy
  \(\varepsilon'\) számot elég kicsire választva
  \(b - d \leq f\left( x_{0} \right) - \varepsilon' < f\left( x_{0} \right) + \varepsilon' \leq b + d\).
  Ez utóbbit másképp felírva:
  \(\left\lbrack {f\left( x_{0} \right) - \varepsilon',f\left( x_{0} \right) + \varepsilon'} \right\rbrack \subset \left\lbrack {b - d,b + d} \right\rbrack\).
  \(\Phi\left( {x_{0},f\left( x_{0} \right)} \right) = 0\), ezért mivel
  \(\Phi\) folytonos, \(\exists\rho:x \in B_{\rho}\left( x_{0} \right)\)
  esetén
  \(\Phi\left( {x,f\left( x_{0} \right) - \varepsilon'} \right) < 0 < \Phi\left( {x,f\left( x_{0} \right) + \varepsilon'} \right)\).
  Ekkor a Bolzano tétel segítségével
  \(\exists!y:\Phi\left( {x,y} \right) = 0,f\left( x_{0} \right) - \varepsilon' < y < f\left( x_{0} \right) + \varepsilon'\),
  node \(y = f\left( x \right)\) és \(\varepsilon' < \varepsilon\)
  miatt, minden \(x \in B_{\rho}\left( x_{0} \right)\) pontra
  \(f\left( x_{0} \right) - \varepsilon < f\left( x \right) < f\left( x_{0} \right) + \varepsilon\),
  tehát \(f\) folytonos \(x_{0}\) -ban.
\end{itemize}

\end{bizonyitas}

\hypertarget{a-tetel-altalanositasa-left.-phimathbbrn-times-mathbbrmrightarrowmathbbrm-right.-fuggvenyekre}{%
\subsubsection{\texorpdfstring{A tétel általánosítása
\(\left. \Phi:{\mathbb{R}}^{n} \times {\mathbb{R}}^{m}\rightarrow{\mathbb{R}}^{m} \right.\)
függvényekre:}{A tétel általánosítása \textbackslash{}left. \textbackslash{}Phi:\{\textbackslash{}mathbb\{R\}\}\^{}\{n\} \textbackslash{}times \{\textbackslash{}mathbb\{R\}\}\^{}\{m\}\textbackslash{}rightarrow\{\textbackslash{}mathbb\{R\}\}\^{}\{m\} \textbackslash{}right. függvényekre:}}\label{a-tetel-altalanositasa-left.-phimathbbrn-times-mathbbrmrightarrowmathbbrm-right.-fuggvenyekre}}

\begin{tetel}

Tétel:\\
Tfh
\(\left. \Phi:{\mathbb{R}}^{n} \times {\mathbb{R}}^{m}\rightarrow{\mathbb{R}}^{m} \right.\),
\(\Phi\left( {a,b} \right) = 0\), értelmezve van
\(\left( {a,b} \right)\) pont egy környezetében és itt folytonos is,
továbbá \(\left. y\mapsto\Phi\left( {x,y} \right) \right.\) folytonosan
differenciálható \(\left( {a,b} \right)\) valamilyen környezetében.
Továbbá legyen
\(\Phi: = \left( {\Phi_{1},\Phi_{2},...,\Phi_{m}} \right)\).
\(\begin{pmatrix} {\partial_{y_{1}}\Phi_{1}} & {\partial_{y_{2}}\Phi_{1}} & \cdots & {\partial_{y_{m}}\Phi_{1}} \\ {\partial_{y_{1}}\Phi_{2}} & {\partial_{y_{2}}\Phi_{2}} & \cdots & {\partial_{y_{m}}\Phi_{2}} \\  \vdots & \vdots & \ddots & \vdots \\ {\partial_{y_{1}}\Phi_{m}} & {\partial_{y_{2}}\Phi_{m}} & \cdots & {\partial_{y_{m}}\Phi_{m}} \\ \end{pmatrix} = :F\),
és tegyük fel, hogy
\(\det\left( {F\left( {a,b} \right)} \right) \neq 0\). Ekkor
\(\exists\left( {a,b} \right)\)-nek olyan
\(B_{r}\left( a \right) \times B_{\rho}\left( b \right)\) környezete, és
\(\left. f:B_{r}\left( a \right)\rightarrow B_{\rho}\left( b \right) \subset {\mathbb{R}}^{m} \right.\)
folytonos függvény, hogy
\(\left\{ {\left( {x,y} \right) \in B_{r}\left( a \right) \times B_{\rho}\left( b \right):\Phi\left( {x,y} \right) = 0} \right\} = \left\{ {\left( {x,f\left( x \right)} \right):x \in B_{r}\left( a \right)} \right\}\).

\end{tetel}

\begin{tetel}

Tétel:\\
Tfh teljesülnek az előbbi tétel feltételei és
\(\left. x\mapsto\Phi\left( {x,y} \right) \right.\) függvény is
folytonosan differenciálható az \(a\) környezetében, akkor
\(\left. f:B_{r}\left( a \right)\rightarrow{\mathbb{R}}^{m} \right.\)
differenciálható.

\end{tetel}

\begin{megjegyzes}

Megjegyzés:\\
Ha tudjuk, hogy \(f\) differencálható, akkor \(f\) deriváltja
kiszámolható. \(\Phi\left( {x,f\left( x \right)} \right) = 0\)-t
\(x \in B_{r}\left( a \right)\) szerint deriválva \[\begin{gathered}
  0 = {\partial _x}\Phi \left( {x,f\left( x \right)} \right) + {\partial _y}\Phi \left( {x,f\left( x \right)} \right)f'\left( x \right) \\ 
   \Downarrow  \\ 
  f'\left( x \right) =  - {\left[ {{\partial _y}\Phi \left( {x,f\left( x \right)} \right)} \right]^{ - 1}}\left[ {{\partial _x}\Phi \left( {x,f\left( x \right)} \right)} \right]. \\ 
\end{gathered} \]

\end{megjegyzes}

\hypertarget{inverz-fuggveny-tetel}{%
\subsubsection{Inverz függvény tétel}\label{inverz-fuggveny-tetel}}

\begin{tetel}

Tétel:\\
Legyen \(\left. g:{\mathbb{R}}^{n}\rightarrow{\mathbb{R}}^{n} \right.\),
mely értelmezve van és folytonosan differenciálható a
\(b \in {\mathbb{R}}^{n}\) egy környezetében, továbbá az alábbi mátrix
determinánsa nem 0 a \(b\)-ben.
\(g: = \left( {g_{1},g_{2},...,g_{n}} \right),g^{\prime} = \begin{pmatrix} {\partial_{1}g_{1}} & {\partial_{2}g_{1}} & \cdots & {\partial_{n}g_{1}} \\ {\partial_{1}g_{2}} & {\partial_{2}g_{2}} & \cdots & {\partial_{n}g_{2}} \\  \vdots & \vdots & \ddots & \vdots \\ {\partial_{1}g_{n}} & {\partial_{2}g_{n}} & \cdots & {\partial_{n}g_{n}} \\ \end{pmatrix}\).
Legyen \(a: = g\left( b \right)\). Ekkor
\(\left. \exists B_{r}\left( a \right),B_{r}\left( b \right),g^{- 1}:B_{r}\left( b \right)\rightarrow B_{r}\left( a \right) \right.\),
folytonosan differenciálható függvény, hogy
\(\left\{ {\left( {x,y} \right) \in B_{r}\left( a \right) \times B_{r}\left( b \right):x = g\left( y \right)} \right\} = \left\{ {\left( {x,g^{- 1}\left( x \right)} \right):x \in B_{r}\left( a \right)} \right\}\).

\end{tetel}

\begin{megjegyzes}

Megjegyzés:\\
A tétel szerint a \(g\) függvénynek létezik lokális inverze, azaz \(g\)
függvényt a \(b\) egy elég kis környezetére leszűkítve, létezik az
inverz.

\end{megjegyzes}

\begin{bizonyitas}

Bizonyítás:\\
Legyen \(\Phi\left( {x,y} \right): = x - g\left( y \right)\), ekkor
\(\Phi\left( {a,b} \right) = a - g\left( b \right) = 0\),
\(\partial_{y}\Phi\left( {x,y} \right) = - g^{\prime}\left( y \right)\)
és \(\det\left( {g^{\prime}\left( b \right)} \right) \neq 0\). Az előbbi
képlet szerint
\(\left\lbrack g^{- 1} \right\rbrack^{\prime}\left( x \right) = \left\lbrack {g^{\prime}\left( {g^{- 1}\left( x \right)} \right)} \right\rbrack^{- 1}\)
(mátrix inverz).

\end{bizonyitas}

\hypertarget{felteteles-szelsoertek}{%
\subsection{Feltételes szélsőérték}\label{felteteles-szelsoertek}}

\begin{definicio}

Definíció:\\
Legyen
\(\left. F:{\mathbb{R}}^{n} \times {\mathbb{R}}^{m}\rightarrow{\mathbb{R}} \right.\)-be
képező függvény,
\(\left. \Phi:{\mathbb{R}}^{n} \times {\mathbb{R}}^{m}\rightarrow{\mathbb{R}}^{m} \right.\)
és \(\Phi\left( {a,b} \right): = 0\). Azt mondjuk, hogy az \(F\)
függvénynek a \(\Phi\left( {x,y} \right): = 0\) feltétel mellett lokális
minimuma van az \(\left( {a,b} \right)\) pontban, ha
\(\exists\delta > 0:x \in B_{\delta}\left( a \right),y \in B_{\delta}\left( b \right),\Phi\left( {x,y} \right) = 0\)
esetén \(F\left( {x,y} \right) \geq F\left( {a,b} \right)\).

\end{definicio}

Kérdés: milyen szükséges feltétel adható a feltételes szélsőérték
létezéséhez? Az implicit függvénytétel segítségével a feltételes
szélsőérték visszavezethető egy szokásos szélsőértékre (feltétel
nélkülire). Feltesszük, hogy implicit függvény differenciálhatóságáról
szóló tétel feltételei teljesülnek. A tétel feltételei: \(\Phi\)
folytonosan differenciálható \(\left( {a,b} \right)\) egy környezetében
és \(\det\left( {\partial_{y}\Phi\left( {a,b} \right)} \right) \neq 0\).
Tfh \(\left( {a,b} \right)\)-n \(F\)-nek feltételes szélsőértéke van.
Tudjuk (az implicit függvénytételből), hogy ekkor
\[\exists\delta_{1} > 0:\left\{ {\left( {x,y} \right) \in B_{\delta_{1}}\left( a \right) \times B_{\delta_{1}}\left( b \right):\Phi\left( {x,y} \right) = 0} \right\} = {\left\{ {\left( {x,f\left( x \right)} \right) \in B_{\delta_{1}}\left( a \right)} \right\},}\]
ahol
\(\left. f:B_{\delta_{1}}\left( a \right)\rightarrow B_{\delta_{1}}\left( b \right) \right.\)
folytonosan differenciálható,
\(\delta_{2}: = \min\left\{ {\delta,\delta_{1}} \right\}\) jelöléssel
\(x \in B_{\delta_{2}}\left( a \right)\) esetén
\(F\left( {x,f\left( x \right)} \right) \geq F\left( {a,f\left( a \right)} \right)\),
tehát az
\(\left. x\mapsto F\left( {x,f\left( x \right)} \right) \right.\)
függvénynek \(a\)-ben lokális minimuma van.\\
\(g\left( x \right) = F\left( {x,f\left( x \right)} \right)\),
\(g^{\prime}\left( x \right) = \partial_{x}F\left( {x,f\left( x \right)} \right) + \partial_{y}F\left( {x,f\left( x \right)} \right) \cdot f^{\prime}\left( x \right)\),
a lokális minimumból következik, hogy
\(0 = g^{\prime}\left( a \right) = \partial_{x}F\left( {a,b} \right) + \partial_{y}F\left( {a,b} \right)f^{\prime}\left( a \right)\)
\(f^{\prime}\left( a \right) = - \left\lbrack {\partial_{y}\Phi\left( {a,b} \right)} \right\rbrack^{- 1} \cdot \left\lbrack {\partial_{x}\Phi\left( {a,b} \right)} \right\rbrack\),
ezt az előzőbe visszahelyettesítve
\[g'\left( a \right) = {\partial _x}F\left( {a,b} \right)\underbrace { - {\partial _y}F\left( {a,b} \right){{\left[ {{\partial _y}\Phi \left( {a,b} \right)} \right]}^{ - 1}}}_{: = \lambda }\left[ {{\partial _x}\Phi \left( {a,b} \right)} \right] = 0\]
Jelölés:
\(G\left( {x,y} \right) = F\left( {x,y} \right) + \lambda\Phi\left( {x,y} \right)\)

\begin{ajanlo}

\begin{ajanlofig}

\href{https://xkcd.com}{\includegraphics[width=5.20833in,height=2.82292in]{wikipedian_protester.png}}

\end{ajanlofig}

Text

\end{ajanlo}

Észrevétel: egyrészt
\(0 = g^{\prime}\left( a \right) = \partial_{x}F\left( {a,b} \right) + \lambda\partial_{x}\Phi\left( {a,b} \right) = \partial_{x}G\left( {a,b} \right)\),
másrészt
\(\lambda = - \left\lbrack {\partial_{y}F\left( {a,b} \right)} \right\rbrack\left\lbrack {\partial_{y}\Phi\left( {a,b} \right)} \right\rbrack^{- 1}\)
így írható:
\(\partial_{y}G\left( {a,b} \right) = \lambda\left\lbrack {\partial_{y}\Phi\left( {a,b} \right)} \right\rbrack + \left\lbrack {\partial_{y}F\left( {a,b} \right)} \right\rbrack = 0\).
Ezen kívül tudjuk, hogy \(\Phi\left( {a,b} \right) = 0\).

\begin{tetel}

Tétel:\\
Tfh \(F\) és \(\Phi\) folytonosan differenciálható
\(\left( {a,b} \right)\) egy környezetében, továbbá
\(\Phi\left( {a,b} \right) = 0,\,\partial_{y}\Phi\left( {a,b} \right)\)
mátrix determinánsa nem 0. Ha
\(\left. F:{\mathbb{R}}^{n} \times {\mathbb{R}}^{n}\rightarrow{\mathbb{R}} \right.\)
függvénynek \(\left( {a,b} \right)\) -ben lokális szélsőértéke van a
\(\Phi\left( {x,y} \right) = 0\) feltétel mellett, akkor a
\(G\left( {x,y} \right) = F\left( {x,y} \right) + \lambda\Phi\left( {x,y} \right)\)
függvényre
\(0 = \partial_{x}G\left( {a,b} \right) = \partial_{x}F\left( {a,b} \right) + \lambda\partial_{x}\Phi\left( {a,b} \right)\)
és
\(0 = \partial_{y}G\left( {a,b} \right) = \partial_{y}F\left( {a,b} \right) + \lambda\partial_{y}\Phi\left( {a,b} \right)\).
Itt
\(\lambda = \left( {\lambda_{1},\lambda_{2},...,\lambda_{m}} \right)\).

\end{tetel}

\begin{megjegyzes}

Megjegyzés:\\
A fentiek szerint az \(a \in {\mathbb{R}}^{n}\),
\(b \in {\mathbb{R}}^{m}\) és \(\lambda \in {\mathbb{R}}^{m}\)
ismeretlenekre \(n + 2m\) egyenletet nyertünk. Ennek egyértelmű
megoldására van esély.

\end{megjegyzes}

\hypertarget{vonalintegral}{%
\section{Vonalintegrál}\label{vonalintegral}}

Rövid összefoglalás a Reimann-integrálról.\\
Legyen
\(\left. f:\left\lbrack {a,b} \right\rbrack\rightarrow{\mathbb{R}} \right.\)
korlátos függvény! Tekintsük \(\left\lbrack {a,b} \right\rbrack\) egy
véges felosztását!
\(a: = x_{0} < x_{1} < x_{2} < ... < x_{k - 1} < x_{k} < ... < x_{n}: = b\).
Legyen \(x_{k - 1} < \xi_{k} < x_{k}\), ekkor definiáljuk:
\(t\left( \tau \right): = {\sum\limits_{k = 1}^{n}{f\left( \xi_{k} \right)}}\left( {x_{k} - x_{k - 1}} \right)\)
(ahol \(\tau\) jelöli a felosztást). Az \(f\) függvényt Reimann szerint
integrálhatónak nevezzük, ha
\(\left. t\left( \tau \right)\rightarrow I \right.\), ha a felosztást
minden határon túl finomítjuk. Ez azt jelenti, hogy
\(\forall\varepsilon > \exists\delta > 0\), hogy ha a felosztás
\(\delta\)-nál finomabb (minden részintervallum \(< \delta\)), akkor
\(\left| {t\left( \tau \right) - I} \right| < \varepsilon\). Cél:
röviden bizonyítjuk, hogy ha \(f\) folytonos, akkor \(f\) Reimann
integrálható.\\
Felső összeg:
\(S\left( \tau \right): = {\sum\limits_{k = 1}^{n}{M_{k}\left( {x_{k} - x_{k - 1}} \right)}}\),
ahol \(M_{k}: = \sup\limits_{\lbrack{x_{k - 1},x_{k}}\rbrack}f\), alsó
összeg:
\(s\left( \tau \right): = {\sum\limits_{k = 1}^{n}{m_{k}\left( {x_{k} - x_{k - 1}} \right)}}\),
ahol \(m_{k}: = \inf\limits_{\lbrack{x_{k - 1},x_{k}}\rbrack}f\).

\begin{allitas}

Állítás:\\
Bármilyen \(\tau\) felosztáshoz tartozó \(s\left( \tau \right)\) alsó
összeg \(\leq\) bármely \(\tau'\) felosztáshoz tartozó
\(S\left( {\tau'} \right)\) felső összeg.

\end{allitas}

Következmény: a felső összegek halmaza alulról korlátos, az alsó
összegek halmaza felülről korlátos. Ebből következik, hogy
\(\exists\inf\limits_{\tau}S\left( \tau \right) \geq \sup\limits_{\tau}s\left( \tau \right)\).

\begin{megjegyzes}

Megjegyzés:\\
\(\left. m_{k} \leq f\left( \xi_{k} \right) \leq M_{k}\Rightarrow s\left( \tau \right) \leq t\left( \tau \right) \leq S\left( \tau \right) \right.\).

\end{megjegyzes}

\begin{definicio}

Definíció:\\
Oszcillációs összeg:
\(O\left( \tau \right): = S\left( \tau \right) - s\left( \tau \right)\).

\end{definicio}

\begin{tetel}

Tétel:\\
Tfh
\(\left. f:\left\lbrack {a,b} \right\rbrack\rightarrow{\mathbb{R}}^{n} \right.\)
folytonos függvény. Ekkor az \(O\left( \tau \right)\) oszcillációs
összeg tart 0-hoz, ha a felosztást minden határon túl finomítjuk, azaz
\(\forall\varepsilon > 0\exists\delta > 0\), hogy ha a felosztást
\(\delta\)-nál finomabb, akkor
\(0 \leq O\left( \tau \right) \leq \varepsilon\).

\end{tetel}

\begin{bizonyitas}

Bizonyítás:\\
Az egyenletes folytonosság tétele (Heine) szerint
(\(\left\lbrack {a,b} \right\rbrack\) korlátos és zárt, tehát
sorozatkompakt) \(f\) egyenletesen folytonos. Ez pontosan azt jelenti,
hogy
\(\left. \forall\varepsilon > 0\exists\delta > 0:\left| {x_{1} - x_{2}} \right| \leq \delta\Rightarrow\left| {f\left( x_{1} \right) - f\left( x_{2} \right)} \right| \leq \varepsilon \right.\).
Ezért \(\delta\)-nál finomabb felosztást választva \[\begin{aligned}
  O\left( \tau  \right) = S\left( \tau  \right) - s\left( \tau  \right) &  = \mathop \sum \limits_{k = 1}^n \underbrace {\left( {{M_k} - {m_k}} \right)}_{ \leqslant \varepsilon }\left( {{x_k} - {x_{k - 1}}} \right) \\ 
   &  \leqslant \varepsilon \mathop \sum \limits_{k = 1}^n \left( {{x_k} - {x_{k - 1}}} \right) = \varepsilon  \cdot \left( {b - a} \right).\\ 
\end{aligned} \]

\end{bizonyitas}

Következmény: ha
\(\left. f:\left\lbrack {a,b} \right\rbrack\rightarrow{\mathbb{R}} \right.\)
folytonos, akkor Reimann integrálható, azaz ekkor
\(\left. t\left( \tau \right)\rightarrow I \right.\), ha a felosztást
finomítjuk. Ugyanis a tétel szerint tetszőleges \(\varepsilon > 0\)
számhoz \(\exists\delta > 0\), hogy ha a felosztást \(\delta\)-nál
finomabb, akkor
\[\left. 0 \leq S\left( \tau \right) - s\left( \tau \right) < \varepsilon\Rightarrow\inf\limits_{t}S\left( \tau \right) = \sup\limits_{\tau}s\left( \tau \right): = I. \right.\]
Ekkor
\(s\left( \tau \right) \leq t\left( \tau \right) \leq S\left( \tau \right)\),
\(\left| {t\left( \tau \right) - I} \right| < \varepsilon\), ha \(\tau\)
felosztás \(\delta\)-nál finomabb.

\begin{megjegyzes}

Megjegyzés:\\
\(S\left( \tau \right)\) és \(s\left( \tau \right)\) is tart \(I\)-hez.

\end{megjegyzes}

\hypertarget{folytonosan-differencialhato-ut-illetve-gorbe-ivhossz-kiszamitasa}{%
\subsection{Folytonosan differenciálható út, illetve görbe, ívhossz
kiszámítása}\label{folytonosan-differencialhato-ut-illetve-gorbe-ivhossz-kiszamitasa}}

Legyen egy
\(\left. \phi:\left\lbrack {\alpha,\beta} \right\rbrack\rightarrow{\mathbb{R}}^{n} \right.\)
folytonosan differenciálható! Ekkor azt mondjuk, hogy \(\phi\) egy
folytonosan differenciálható \(L\) utat határoz meg az
\({\mathbb{R}}^{n}\) térben.
\(t \in \left\lbrack {\alpha,\beta} \right\rbrack,\phi\left( t \right) \in {\mathbb{R}}^{n}\),
a mozgó pont a \(t\) időben a \(\phi\left( t \right)\) helyen van.

\begin{allitas}

Állítás:\\
A fönt értelmezett út hossza:
\(\int\limits_{\alpha}^{\beta}{\left| {\overset{.}{\phi}\left( t \right)} \right|dt}\).

\end{allitas}

\begin{bizonyitas}

Bizonyítás:\\
A legyen
\(\alpha = t_{0} < t_{1} < t_{2} < ... < t_{k - 1} < t_{k} < ... < t_{m} = \beta\)!
Az ívhossz definíció szerint a felosztáshoz tartozó törött vonal
hosszának limesze, miközben a felosztást finomítjuk. A törött vonal
hossza
\({\sum\limits_{k = 1}^{m}\left| {\phi\left( t_{k} \right) - \phi\left( t_{k - 1} \right)} \right|} = {\sum\limits_{k = 1}^{m}\left| \frac{\phi\left( t_{k} \right) - \phi\left( t_{k - 1} \right)}{t_{k} - t_{k - 1}} \right|}\left( {t_{k} - t_{k - 1}} \right) = {\sum\limits_{k = 1}^{m}\sqrt{\sum\limits_{j = 1}^{n}\left\lbrack \frac{\phi_{j}\left( t_{k} \right) - \phi_{j}\left( t_{k - 1} \right)}{t_{k} - t_{k - 1}} \right\rbrack^{2}}}\left( {t_{k} - t_{k - 1}} \right)\),
mely a Lagrange-féle középértéktétellel
\(= {\sum\limits_{k = 1}^{m}{\sqrt{\sum\limits_{j = 1}^{n}\left( {{\overset{.}{\phi}}_{j}\left( \tau_{jk} \right)} \right)^{2}}\left( {t_{k} - t_{k - 1}} \right)}}\),
ahol \(t_{k - 1} \leq \tau_{jk} \leq t_{k}\). Jó lenne, ha e helyett
ilyen alakú összeg lenne:
\({\sum\limits_{k = 1}^{m}\sqrt{\sum\limits_{j = 1}^{n}{{\overset{.}{\phi}}_{j}\left( \tau_{k} \right)^{2}}}}\left( {t_{k} - t_{k - 1}} \right) = {\sum\limits_{k = 1}^{m}{\left| {\overset{.}{\phi}\left( \tau_{k} \right)} \right|\left( {t_{k} - t_{k - 1}} \right)}}\),
ez nem mást, mint
\(\left. t\mapsto\left| {\overset{.}{\phi}\left( t \right)} \right| \right.\)
integrál közelítő összege. Mivel ez a függvény folytonos, az integrál
közelítő összeg tart az integrálhoz. Belátható, hogy a kétféle összeg
különbsége tart 0-hoz, ha a felosztást minden határon túl finomítjuk.

\end{bizonyitas}

\begin{definicio}

Definíció:\\
Legyen
\(\left. \phi:\left\lbrack {\alpha,\beta} \right\rbrack\rightarrow{\mathbb{R}}^{n} \right.\)
folytonosan differenciálható, \(\phi\) injektív,
\(\overset{.}{\phi}\left( t \right) \neq 0,\forall t \in \left\lbrack {\alpha,\beta} \right\rbrack\).
Ekkor azt mondjuk, hogy \(\phi\) egyszerű, folytonosan differenciálható
\(L\) utat határoz meg. Ekkor
\(\Gamma: = R_{\phi} \subset {\mathbb{R}}^{n}\) halmazt egyszerű,
folytonosan differenciálható görbének nevezzük.

\end{definicio}

\begin{tetel}

Tétel:\\
Ha \(\phi\) és \(\widetilde{\phi}\) olyan
\(\left. \left\lbrack {\alpha,\beta} \right\rbrack\rightarrow{\mathbb{R}}^{n} \right.\)
függvények, melyek egyszerű folytonosan differenciálható utat határoznak
meg és \(R_{\phi} = R_{\widetilde{\phi}}\), továbbá
\(\phi\left( \alpha \right) = \widetilde{\phi}\left( \alpha \right)\) és
\(\phi\left( \beta \right) = \widetilde{\phi}\left( \beta \right)\),
akkor
\({\int\limits_{\alpha}^{\beta}{\left| {\overset{.}{\phi}\left( t \right)} \right|dt}} = {\int\limits_{\alpha}^{\beta}{\left| {\overset{.}{\widetilde{\phi}}\left( \widetilde{t} \right)} \right|d\widetilde{t}}}\).
Ezt nevezzük \(\Gamma = R_{\phi}\) egyszerű folytonosan differenciálható
görbe ívhosszának.

\end{tetel}

\begin{megjegyzes}

Megjegyzés:\\
Ha \(\phi\left( \alpha \right) = \phi\left( \beta \right)\), akkor
egyszerű zárt folytonosan differenciálható útról (illetve görbéről)
beszélünk.

\end{megjegyzes}

\begin{definicio}

Definíció:\\
Legyen
\(\left. \phi:\left\lbrack {\alpha,\beta} \right\rbrack\rightarrow{\mathbb{R}}^{n} \right.\)
folytonosan differenciálható! Ez meghatároz egy \(L\) folytonosan
differenciálható utat. Legyen \(\Gamma: = R_{\phi}\) és
\(\left. f:\Gamma\rightarrow{\mathbb{R}} \right.\) folytonos függvény.
Értelmezni akarjuk az \(f\) függvénynek az \(x_{k}\) változó szerinti
vonalintegrálját. Tekintsük a következő közelítő összeget:
\(\sum\limits_{k = 1}^{m}{f\left( {\phi\left( \tau_{k} \right)} \right)\left\lbrack {\phi_{j}\left( t_{k} \right) - \phi_{j}\left( t_{k - 1} \right)} \right\rbrack}\),
ahol
\(\tau \in \left\lbrack {t_{k - 1},t_{k}} \right\rbrack \subset \left\lbrack {\alpha,\beta} \right\rbrack\).
Ha ez tart valamely \(I\) véges számhoz, miközben a felosztást
finomítjuk, akkor ezt nevezzük \(f\)-nek \(x_{j}\) szerinti
vonalintegráljának, s így jelöljük:
\(\int_{L}{f\left( x \right)dx_{j}}\). Számoljuk ki a limeszt!
\({\sum\limits_{k = 1}^{m}{f\left( {\phi\left( \tau_{k} \right)} \right)\left\lbrack {\phi_{j}\left( t_{k} \right) - \phi_{j}\left( t_{k - 1} \right)} \right\rbrack}} = {\sum\limits_{k = 1}^{m}{f\left( {\phi\left( \tau_{k} \right)} \right)\frac{\phi_{j}\left( t_{k} \right) - \phi_{j}\left( t_{k - 1} \right)}{t_{k} - t_{k - 1}}\left( {t_{k} - t_{k - 1}} \right)}}\),
mely a Lagrange-féle középértéktétel segítségével
\(= {\sum\limits_{k = 1}^{m}{f\left( {\phi\left( \tau_{k} \right)} \right){\overset{.}{\phi}}_{j}\left( \tau_{k}^{\ast} \right)}}\left( {t_{k} - t_{k - 1}} \right)\),
ahol \(t_{k - 1} < \tau_{k}^{\ast} < t\). Ha e helyett a következő
összeg lenne, az nagyon jó volna:
\[\mathop \sum \limits_{k = 1}^m f\left( {\phi \left( {{\tau _k}} \right)} \right){\dot \phi _j}\left( {{\tau _k}} \right)\left( {{t_k} - {t_{k - 1}}} \right) \to \underbrace {\mathop \smallint \limits_\alpha ^\beta  f\left( {\phi \left( t \right)} \right){{\dot \phi }_j}\left( t \right)dt}_{\mathop \smallint \limits_\alpha ^\beta  \left( {f \circ \phi } \right) \cdot {{\dot \phi }_j}}: = \mathop \smallint \limits_L f\left( x \right)d{x_j}.\]Belátható,
hogy a két összeg különbsége 0-hoz tart, ha a felosztást minden határon
túl finomítjuk.

\end{definicio}

\begin{allitas}

Állítás:\\
Ha \(\Gamma\) egyszerű folytonosan differenciálható görbe, amelyet egy
valamely \(L\) egyszerű folytonosan differenciálható úttal járunk be,
akkor a vonalintegrál értéke független a \(\phi\) paraméterezés
megválasztásától, ha rögzítettek a kezdő és végpontok (a bejárás iránya
is adott).

\end{allitas}

\begin{definicio}

Definíció:\\
\(\left. \phi:\left\lbrack {\alpha,\beta} \right\rbrack\rightarrow{\mathbb{R}}^{n} \right.\)
folytonosan differenciálható, \(L\) folytonosan differenciálható út,
\(\Gamma = R_{\phi}\),
\(\left. g:\Gamma\rightarrow{\mathbb{R}}^{n} \right.\) folytonos.
Tekintsük
\(\left. {\sum\limits_{k = 1}^{m}\left\langle {g\left( {\phi\left( \tau_{k} \right)} \right),\phi\left( t_{k} \right) - \phi\left( t_{k - 1} \right)} \right\rangle}\rightarrow? \right.\)
Ha a limesz létezik, akkor ezt így jelöljük: \(\int_{L}{gdx}\).

\end{definicio}

\begin{ajanlo}

\begin{ajanlofig}

\href{https://xkcd.com}{\includegraphics[width=5.20833in,height=2.82292in]{wikipedian_protester.png}}

\end{ajanlofig}

Text

\end{ajanlo}

\(\left. \phi:\left\lbrack {\alpha,\beta} \right\rbrack\rightarrow{\mathbb{R}}^{n} \right.\)
folytonosan differenciálható, ekkor ez egy \(L\) folytonosan
differenciálható utat határoz meg.
\(\Gamma: = R_{\phi} \subset {\mathbb{R}}^{n}\). Ha \(\phi\) injektív és
\(\overset{.}{\phi}\left( t \right) \neq 0\) minden \(t\)-re, akkor
egyszerű folytonosan differenciálható utat határoz meg, \(\Gamma\) -t
egyszerű folytonosan differenciálható görbének nevezzük. (Ekkor a fenti
integrált a \(\Gamma\) görbén vett integrálnak nevezzük.)

\begin{enumerate}
\def\labelenumi{\arabic{enumi}.}
\tightlist
\item
  Legyen \(\left. f:\Gamma\rightarrow{\mathbb{R}} \right.\) folytonosan
  differenciálható függvény. Ekkor nevezzük a
  \[\lim\limits_{\Delta t\rightarrow 0}{\sum\limits_{k = 1}^{m}{f\left( {\phi\left( \tau_{k} \right)} \right)\left\lbrack {\phi_{j}\left( t_{k} \right) - \phi_{j}\left( t_{k - 1} \right)} \right\rbrack}}: = {\int_{L}{f\left( x \right)dx_{j}}}\]
  mennyiséget \(f\)-nek \(j\)-edik változója szerinti
  vonalintegráljának. A felosztást finomítva a fenti közelítő összeg
  tart
  \({\int\limits_{\alpha}^{\beta}{f\left( {\phi\left( t \right)} \right){\overset{.}{\phi}}_{j}\left( t \right)dt}} = {\int\limits_{\alpha}^{\beta}{\left( {f \circ \phi} \right){\overset{.}{\phi}}_{j}}}\)
  integrálhoz.
\item
  Legyen \(\left. g:\Gamma\rightarrow{\mathbb{R}}^{n} \right.\)
  folytonos függvény. Tekintsük a következő mennyiséget:
  \[\sum\limits_{k = 1}^{m}{\left\langle {g\left( {\phi\left( \tau_{k} \right)} \right),\phi\left( t_{k} \right) - \phi\left( t_{k - 1} \right)} \right\rangle.}\]Kérdés:
  ennek van-e limesze, miközben a felosztást finomítjuk? A vizsgált
  mennyiséget átírva:
  \[\mathop \sum \limits_{k = 1}^m \left[ {\mathop \sum \limits_{j = 1}^n {g_j}\left( {\phi \left( {{\tau _k}} \right)} \right)\left( {{\phi _j}\left( {{t_k}} \right) - {\phi _j}{t_{k - 1}}} \right)} \right] = \mathop \sum \limits_{j = 1}^n \underbrace {\left[ {\mathop \sum \limits_{k = 1}^m {g_j}\left( {\phi \left( {{\tau _k}} \right)} \right)\left( {{\phi _j}\left( {{t_k}} \right) - {\phi _j}\left( {{t_{k - 1}}} \right)} \right)} \right]}_{ \to \mathop \smallint \limits_L {g_j}\left( x \right)d{x_j} = \mathop \smallint \limits_a^b \left( {{g_j} \circ \phi } \right){{\dot \phi }_j}},\]
  ahol \(g = \left( {g_{1},g_{2},...,g_{n}} \right)\). Felcserélve az
  összegzést az integrálással, a vizsgált mennyiség hatáértéke
  \({\int\limits_{\alpha}^{\beta}{\sum\limits_{j = 1}^{n}{\left( {g_{j} \circ \phi} \right){\overset{.}{\phi}}_{j}}}} = {\int\limits_{\alpha}^{\beta}{\left\langle {g\left( {\phi\left( t \right)} \right),\overset{.}{\phi}\left( t \right)} \right\rangle dt = :{\int_{L}{g\left( x \right)dx}}}}\).
  Ez mennyiség elég fontos fizikai alkalmazásokban, ezért mi is sokat
  fogunk vele foglalkozni. Szemléletes jelentést társíthatunk hozzá, ha
  \(g\left( x \right)\) az \(x\) pontban ható erő. Ekkor az integrál
  értéke a görbén végigmozogva az erőtér által végzett munka.
\item
  Ívhossz szerinti vonalintegrál. Az előzőhöz képest csak
  \(\left. f:\Gamma\rightarrow{\mathbb{R}} \right.\) a változás. Ekkor
  tekintsük a következő összeget:
  \(\sum\limits_{k = 1}^{m}{f\left( {\phi\left( \tau_{k} \right)} \right)\left| {\phi\left( t_{k} \right) - \phi\left( t_{k - 1} \right)} \right|}\).
  Ez mihez tart? \[\begin{aligned}
     & \sum\limits_{k = 1}^m {f\left( {\phi \left( {{\tau _k}} \right)} \right)\left| {\phi \left( {{t_k}} \right) - \phi \left( {{t_{k - 1}}} \right)} \right|}  =  \\ 
     =  & \sum\limits_{k = 1}^m {f\left( {\phi \left( {{\tau _k}} \right)} \right)\underbrace {\left| {\frac{{\phi \left( {{t_k}} \right) - \phi \left( {{t_{k - 1}}} \right)}}{{{t_k} - {t_{k - 1}}}}} \right|}_{{\text{Lagrange: }} = \dot \phi \left( {\tau _k^ * } \right)}} \left( {{t_k} - {t_{k - 1}}} \right) \to \int\limits_\alpha ^\beta  {f\left( {\phi \left( t \right)} \right)\left| {\dot \phi \left( t \right)} \right|dt} : = \int_L {fds}  \\ 
  \end{aligned}\]
\end{enumerate}

\begin{megjegyzes}

Megjegyzések:\\
Mind a 3 esetben ha \(\Gamma\) egyszerű folytonosan differenciálható
görbe, akkor a \(\Gamma\) -n vett integrál a paraméterezéstől
függetlenül mindig ugyanaz, ha rögzítjük a kezdő és végpontokat (vagyis
a bejárás irányát is megtartjuk). Célszerű értelmezni a szakaszonként
folytonosan differenciálható utat (egyszerű szakaszonként folytonosan
differenciálható görbét).

\end{megjegyzes}

\begin{definicio}

Definíció:\\
Legyen
\(\left. \phi:\left\lbrack {\alpha,\beta} \right\rbrack\rightarrow{\mathbb{R}}^{n} \right.\)
folytonos és \(\overset{.}{\phi}\) szakaszonként folytonos függvény,
azaz legyen olyan
\(\alpha < \alpha_{1} < ... < \alpha_{l} < \alpha_{l + 1} < ... < \alpha_{r} = \beta\)
felosztás, hogy létezzen
\(\left. \overset{.}{\phi} \right|_{({\alpha_{k - 1},\alpha_{k}})}\) és
folytonos is \(\forall k \in \left\{ {1,2,...,r} \right\}\) -ra, és a
végpontokban létezzen egyoldali határértéke. Ekkor azt mondjuk, hogy
\(\phi\) szakaszonként folytonosan differenciálható utat határoz meg.
Értelmezhető az egyszerű szakaszonként folytonosan differenciálható út.
A fenti 3 definíció és állítások átvihetők erre az esetre. Legyen
\(\Gamma: = R_{\phi}\) és
\(\left. g:\Gamma\rightarrow{\mathbb{R}}^{n} \right.\) folytonos
függvény.
\(\int_L {g\left( x \right)dx} = \mathop \smallint \limits_\alpha ^\beta \underbrace {\left\langle {g\left( {\phi \left( t \right)} \right),\dot \phi \left( t \right)} \right\rangle }_{{\text{szakaszonként folytonos}}}dt\).

\end{definicio}

\hypertarget{a-vonalintegral-alaptulajdonsagai}{%
\subsection{A vonalintegrál
alaptulajdonságai}\label{a-vonalintegral-alaptulajdonsagai}}

\begin{enumerate}
\def\labelenumi{\arabic{enumi}.}
\tightlist
\item
  Legyen
  \(\left. \phi:\left\lbrack {\alpha,\beta} \right\rbrack\rightarrow{\mathbb{R}}^{n} \right.\)
  (szakaszonként) folytonosan differenciálható függvény,
  \(\Gamma: = R_{\phi}\) és
  \(\left. g,h:\Gamma\rightarrow{\mathbb{R}}^{n} \right.\) folytonos
  függvények. Ekkor \(\lambda,\mu \in {\mathbb{R}}\) esetén
  \[{\int_{L}{\left( {\lambda g + \mu h} \right)\left( x \right)dx}} = \lambda{\int_{L}{g\left( x \right)dx}} + \mu{\int_{L}{h\left( x \right)dx.}}\]
\item
  Legyenek
  \(\left. \phi_{1}:\left\lbrack {\alpha,\beta} \right\rbrack\rightarrow{\mathbb{R}}^{n} \right.\)
  és
  \(\left. \phi_{2}:\left\lbrack {\beta,\gamma} \right\rbrack\rightarrow{\mathbb{R}}^{n} \right.\)
  szakaszonként folytonosan differenciálható függvények, és legyen
  \(\phi_{1}\left( \beta \right) = \phi_{2}\left( \beta \right)\). Ekkor
  legyen \[\phi\left( t \right): = \left\{ {\begin{matrix}
  {\phi_{1}\left( t \right)} & {t \in \left\lbrack {\alpha,\beta} \right\rbrack} \\
  {\phi_{2}\left( t \right)} & {t \in \left\lbrack {\beta,\gamma} \right\rbrack} \\
  \end{matrix},} \right.\] vagyis
  \(\left. \phi:\left\lbrack {\alpha,\gamma} \right\rbrack\rightarrow{\mathbb{R}}^{n} \right.\).
  Ekkor \(\phi\) is szakaszonként folytonosan differenciálható függvény
  lesz. \(\Gamma: = R_{\phi}\) illetve legyen
  \(\left. g:\Gamma\rightarrow{\mathbb{R}}^{n} \right.\) függvény
  folytonos! Ekkor
  \({\int_{L}{g\left( x \right)dx}} = {\int_{L_{1}}{g\left( x \right)dx}} + {\int_{L_{2}}{g\left( x \right)dx}}\)
\item
  Legyen
  \(\left. \phi:\left\lbrack {\alpha,\beta} \right\rbrack\rightarrow{\mathbb{R}}^{n} \right.\)
  (szakaszonként) folytonosan differenciálható függvény, ez meghatároz
  egy \(L\) szakaszonként folytonosan differenciálható utat,
  \(\Gamma: = R_{\phi} \subset {\mathbb{R}}^{n}\),
  \(\left. g:\Gamma\rightarrow{\mathbb{R}}^{n} \right.\). Ekkor
  \[\begin{aligned}
    \left| {\int_L {g\left( x \right)dx} } \right| &  = \left| {\int_L {\left\langle {g\left( {\phi \left( t \right)} \right),\dot \phi \left( t \right)} \right\rangle dt} } \right| \\ 
     &  \leqslant \mathop \smallint \limits_\alpha ^\beta  \left| {\left\langle {g\left( {\phi \left( t \right)} \right),\dot \phi \left( t \right)} \right\rangle } \right|dt \leqslant \mathop \smallint \limits_\alpha ^\beta  \underbrace {\left| {g\left( {\phi \left( t \right)} \right)} \right|}_{ \leqslant \mathop {\sup }\limits_\Gamma  \left| g \right|} \cdot \left| {\dot \phi \left( t \right)} \right|dt, \\ 
  \end{aligned} \] azaz
  \(\left| {\int_{L}{g\left( x \right)dx}} \right| \leq \sup\limits_{\Gamma}\left| g \right| \cdot {\int\limits_{\alpha}^{\beta}{\left| {\overset{.}{\phi}\left( t \right)} \right|dt}} = \sup\limits_{\Gamma}\left| g \right| \cdot \left\lbrack {L\text{~ívhossza}} \right\rbrack\).
\end{enumerate}

\hypertarget{a-vonalintegral-uttol-valo-fuggetlensege}{%
\subsection{A vonalintegrál úttól való
függetlensége}\label{a-vonalintegral-uttol-valo-fuggetlensege}}

Adott valamilyen \(\Omega \subset {\mathbb{R}}^{n}\) tartomány (nyílt és
összefüggő). Legyen
\(\left. f:\Omega\rightarrow{\mathbb{R}}^{n} \right.\) folytonos
függvény.

Kérdés: milyen feltételek mellet lesz igaz, hogy az \(\Omega\) két
tetszőleges pontját összekötő szakaszonként folytonosan differenciálható
út mentén vett \(\int_{L}{f\left( x \right)dx}\) integrálja \(f\)-nek
nem függ az úttól egészében, csak annak végpontjaitól?

\begin{tetel}

Tétel:\\
Tfh \(\left. f:\Omega\rightarrow{\mathbb{R}}^{n} \right.\) folytonos
függvény és \(\int_{L}{f\left( x \right)dx}\) értéke csak a kezdő és
végpontoktól függ bármely \(\Omega\)-ban haladó \(L\) út esetén. Legyen
\(a \in \Omega\) és \(\xi \in \Omega\) tetszőleges pontok, \(a\)
rögzített. Tekintsünk egy tetszőleges, olyan \(\Omega\)-ban haladó
szakaszonként folytonosan differenciálható utat, amely \(a\)-t összeköti
\(\xi\)-vel. (Mi az, hogy összeköti? Azt jelenti ez, hogy
\(\left. \exists\phi:\left\lbrack {\alpha,\beta} \right\rbrack\rightarrow{\mathbb{R}}^{n} \right.\),
mely szakaszonként folytonosan differenciálható és
\(\phi\left( t \right) \in \Omega,\forall t \in \left\lbrack {\alpha,\beta} \right\rbrack\),
\(\phi\left( \alpha \right) = a\) és
\(\phi\left( \beta \right) = \xi\).) Legyen
\(F\left( \xi \right): = {\int_{L}{f\left( x \right)dx}} = {\int\limits_{a}^{\xi}{f\left( x \right)dx}}\).
Ekkor \(\partial_{j}F\left( \xi \right) = f_{j}\left( \xi \right)\),
azaz \(F^{\prime}\left( \xi \right) = f\left( \xi \right)\).

\end{tetel}

\begin{bizonyitas}

Bizonyítás:\\
Legyen \(h \in {\mathbb{R}}\text{\textbackslash}\left\{ 0 \right\}\),
\(h^{(j)}: = \left( {0,0,...h,0,...,0} \right) \in {\mathbb{R}}^{n}\) (a
\(j\)-edik komponense a \(h\)). Az \(a\)-tól a \(\xi + h^{(j)}\)-ig
terjedő utat definiálja a következő függvény.
\[\psi \left( t \right): = \left\{ {\begin{array}{*{20}{c}}
  {\phi \left( t \right)}&{\alpha  \leqslant t \leqslant \beta } \\ 
  {\left( {{\xi _1},...,{\xi _j} + t - \beta ,...,{\xi _n}} \right)}&{\beta  \leqslant t \leqslant \beta  + h} 
\end{array}} \right.\] Ekkor \(\beta \leq t \leq \beta + h\) esetén
\(\overset{.}{\psi}\left( t \right) = \left( 0,...,1,...,0 \right)\).
Ekkor \[\begin{aligned}
  \frac{{F\left( {\xi  + {h^{\left( j \right)}}} \right) - F\left( \xi  \right)}}{h} &  = \frac{1}{h}\left[ {\mathop \smallint \limits_a^{\xi  + {h^{\left( j \right)}}} f\left( x \right)dx - \mathop \smallint \limits_a^\xi  f\left( x \right)dx} \right] \\ 
   &  = \frac{1}{h}\mathop \smallint \limits_\xi ^{\xi  + {h^{\left( j \right)}}} f\left( x \right)dx \\ 
   &  = \frac{1}{h}\mathop \smallint \limits_\beta ^{\beta  + h} \left\langle {f\left( {\psi \left( t \right)} \right),\dot \psi \left( t \right)} \right\rangle dt \\ 
   &  = \frac{1}{h}\mathop \smallint \limits_\beta ^{\beta  + h} {f_j}\left( {{\xi _1},...,{\xi _j} + t - \beta ,...,{\xi _n}} \right)dt \\ 
   &  = {f_j}\left( {{\xi _1},...,{\xi _j} + \tau  - \beta ,...,{\xi _n}} \right) \to {f_j}\left( {{\xi _1},...,{\xi _j},...,{\xi _n}} \right), \\ 
\end{aligned} \] ahol \(\tau\) valamilyen alkalmasan választott
\(\beta < \tau < \beta + h\) szám (az egyenlőség az integrálszámítás
középérték tételéből következik).

\end{bizonyitas}

\begin{definicio}

Definíció:\\
Legyen \(\left. f:\Omega\rightarrow{\mathbb{R}}^{n} \right.\)
(egyszerűség kedvéért) folytonos, \(\Omega \subset {\mathbb{R}}^{n}\)
tartomány. Ha \(\left. \Phi:\Omega\rightarrow{\mathbb{R}} \right.\)
függvényre \(\Phi^{\prime} = f\), akkor \(\Phi\)-t \(f\) primitív
függvényének nevezzük.

\end{definicio}

\begin{megjegyzes}

Megjegyzés:

\begin{enumerate}
\def\labelenumi{\arabic{enumi}.}
\tightlist
\item
  Ha \(f\) folytonos, akkor \(\Phi^{\prime} = f\) az \(\Omega\)-n azzal
  ekvivalens, hogy \(\partial_{j}\Phi = f_{j},\forall j\). Az állítás
  fordítottja is igaz, mert ha \(\partial_{j}\Phi\) létezik és folytonos
  \(\Omega\)-n \(\left. \Rightarrow\Phi \right.\) differenciálható
  \(\Omega\)-n,
  \(f = \left\lbrack {\partial_{1}\Phi,...,\partial_{n}\Phi} \right\rbrack\).
  (Ha \(\partial_{j}\Phi\) létezik \(\Omega\)-n \(\Rightarrow\Phi\)
  diffható, csak ha folytonos is \(\partial_{j}\Phi,\forall j\))
\item
  A fenti tétel úgy is fogalmazható, hogy ha
  \(\left. f:\Omega\rightarrow{\mathbb{R}}^{n} \right.\) folytonos
  függvényre \(\int_{L}{f\left( x \right)dx}\) csak \(L\) kezdő és
  végpontjától függ, akkor \(f\)-nek létezik primitív függvénye,
  mégpedig \(F\), amelyre
  \(F\left( \xi \right) = {\int\limits_{a}^{\xi}{f\left( x \right)dx}}\).
\item
  Ha \(\Phi\) függvény \(f\)-nek primitív függvénye
  \(\left. \Rightarrow\Phi + c \right.\) is primitív függvénye, ahol
  \(c \in {\mathbb{R}}\).
\end{enumerate}

\end{megjegyzes}

\begin{tetel}

Tétel:\\
Tfh \(\left. f:\Omega\rightarrow{\mathbb{R}}^{n} \right.\) folytonos és
\(\Phi^{\prime} = f\) (vagyis \(f\)-nek létezik primitív függvénye).
Ekkor \(\int_{L}{f\left( x \right)dx}\) értéke csak \(L\) kezdő és
végpontjától függ, minden \(\Omega\)-n haladó szakaszonként folytonosan
differenciálható \(L\) út esetén.

\end{tetel}

\begin{bizonyitas}

Bizonyítás:\\
Egyszerűség kedvéért először tfh \(L\) folytonosan differenciálható,
\(\left. \phi:\left\lbrack {\alpha,\beta} \right\rbrack\rightarrow\Omega \right.\)
folytonosan differenciálható, \(\phi\left( \alpha \right) = a\) és
\(\phi\left( \beta \right) = b\). \[\begin{aligned}
  \int_L {f\left( x \right)dx}  &  = \mathop \smallint \limits_\alpha ^\beta  \left\langle {f\left( {\phi \left( t \right)} \right),\dot \phi \left( t \right)} \right\rangle dt \\ 
   &  = \int\limits_\alpha ^\beta  {\sum\limits_{j = 1}^n {{f_j}\left( {\phi \left( t \right)} \right){{\dot \phi }_j}\left( t \right)} dt}  \\ 
   &  = \int\limits_\alpha ^\beta  {\sum\limits_{j = 1}^n {{\partial _j}\Phi \left( {\phi \left( t \right)} \right){{\dot \phi }_j}\left( t \right)dt} }  = \int\limits_\alpha ^\beta  {\frac{{d\Phi \left( {\phi \left( t \right)} \right)}}{{dt}}dt},  \\ 
\end{aligned} \] mely a Newton-Leibniz formula felhasználásával
\(= \Phi\left( {\phi\left( \beta \right)} \right) - \Phi\left( {\phi\left( \alpha \right)} \right) = \Phi\left( b \right) - \Phi\left( a \right)\).

\end{bizonyitas}

\begin{tetel}

Tétel:\\
\protect\hypertarget{zartgorbe}{}{} legyen
\(\Omega \subset {\mathbb{R}}^{n}\) tetszőleges tartomány,
\(\left. f:\Omega\rightarrow{\mathbb{R}}^{n} \right.\) folytonos
függvény. Ekkor \(\int_{L}{f\left( x \right)dx}\) értéke csak \(L\)-nek
kezdő és végpontjától függ bármely \(\Omega\)-ban haladó, szakaszonként
folytonosan differenciálható \(L\) út esetén
\(\left. \Leftrightarrow f \right.\)-nek létezik primitív függvénye.

\end{tetel}

\begin{megjegyzes}

Megjegyzés:\\
\(\int_{L}{f\left( x \right)dx}\) értéke csak a kezdő és végpontoktól
függ \(\Leftrightarrow\) bármely szakaszonként folytonosan
differenciálható zárt út mentén az integrál értéke 0.

\end{megjegyzes}

\begin{allitas}

Állítás:\\
Legyen \(\Phi\) az \(f\) folytonos függvény primitív függvénye és
\(F\left( \xi \right): = {\int\limits_{a}^{\xi}{f\left( x \right)dx}}\).
Ekkor \(\Phi - F = {all}\) az \(\Omega\)-n.

\end{allitas}

\begin{bizonyitas}

Bizonyítás:\\
Az előbbi bizonyítás szerint \(\Phi^{\prime} = f\) esetén ha
\(\phi\left( \beta \right) = \xi\), \(\phi\left( \alpha \right) = a\),
\(F\left( \xi \right) = {\int\limits_{a}^{\xi}{f\left( x \right)dx}} = {\int\limits_{\alpha}^{\beta}{\frac{d\Phi\left( {\phi\left( t \right)} \right)}{dt}dt = \Phi\left( \xi \right) - \Phi\left( a \right)}}\),
vagyis
\(\Phi\left( \xi \right) - F\left( \xi \right) = \Phi\left( a \right)\)
konstans.

\end{bizonyitas}

Kérdés: milyen jól használható feltételt tudunk mondani a primitív
függvény létezésére? Tfh
\(\left. f:\Omega\rightarrow{\mathbb{R}}^{n} \right.\) folytonosan
differenciálható (vagyis \(\partial_{k}f_{j}\) folytonos minden
\(k,j\)-re). Ekkor \(\exists\Phi\), hogy \(\Phi^{\prime} = f\), azaz
\(\left. f_{j} = \partial_{j}\Phi\Rightarrow\partial_{k}f_{j} = \partial_{k}\left( {\partial_{j}\Phi} \right) \right.\),
mely a Young tétel szerint
\(= \partial_{j}\left( {\partial_{k}\Phi} \right) = \partial_{j}f_{k}\).

\begin{tetel}

Tétel:\\
Tfh \(\left. f:\Omega\rightarrow{\mathbb{R}}^{n} \right.\) folytonosan
differenciálható függvény. Ha \(f\)-nek létezik primitív függvénye
\(\left. \Rightarrow\partial_{k}f_{j} = \partial_{j}f_{k} \right.\) az
\(\Omega\)-n.

\end{tetel}

Kérdés: a feltétel elegendő-e, azaz ha
\(\partial_{k}f_{j} = \partial_{j}f_{k}\), abból következik-e, hogy
létezik \(f\)-nek primitív függvénye? Általában nem. Tekintsük a
következő példát:
\(\Omega: = \left\{ {\left( {x_{1},x_{2}} \right) \in {\mathbb{R}}^{2}:0 < \left| x \right| < 2} \right\}\).
Legyen \(f: = \left( {f_{1},f_{2}} \right)\),
\(f_{1}\left( {x_{1},x_{2}} \right): = \frac{- x_{2}}{x_{1}^{2} + x_{2}^{2}}\)
és
\(f_{2}\left( {x_{1},x_{2}} \right) = \frac{x_{1}}{x_{1}^{2} + x_{2}^{2}}\).
Belátjuk, hogy \(\partial_{2}f_{1} = \partial_{1}f_{2}\), ugyanis
\(\partial_{2}f_{1}\left( {x_{1},x_{2}} \right) = \frac{- \left( {x_{1}^{2} + x_{2}^{2}} \right) + 2x_{2}^{2}}{\left( {x_{1}^{2} + x_{2}^{2}} \right)^{2}} = \frac{x_{2}^{2} - x_{1}^{2}}{\left( {x_{1}^{2} + x_{2}^{2}} \right)^{2}}\).
\(\partial_{1}f_{2}\left( {x_{1},x_{2}} \right) = \frac{x_{1}^{2} + x_{2}^{2} - 2x_{1}^{2}}{\left( {x_{1}^{2} + x_{2}^{2}} \right)^{2}} = \frac{x_{2}^{2} - x_{1}^{2}}{\left( {x_{1}^{2} + x_{2}^{2}} \right)^{2}}\),
de \({\int_{S_{1}}{f\left( x \right)dx}} \neq 0\), ahol \(S_{1}\) az
egységkör, ami egy
\(\left. \left\lbrack {0,2\pi} \right)\rightarrow{\mathbb{R}}^{2},t\mapsto\left( {\cos t,\sin t} \right) \right.\)
függvény által meghatározott út.

\begin{ajanlo}

\begin{ajanlofig}

\href{https://xkcd.com}{\includegraphics[width=5.20833in,height=2.82292in]{wikipedian_protester.png}}

\end{ajanlofig}

Text

\end{ajanlo}

Előző óráról:
\({\int_{L}{f\left( x \right)dx}} = {\int\limits_{\alpha}^{\beta}{\left\langle {f\left( {\phi\left( t \right)} \right),\overset{.}{\phi}\left( t \right)} \right\rangle dt}}\).
Azt vizsgáltuk, hogy mi volt a feltétele, hogy az integrál értéke csak a
kezdő és végpontoktól függjön, azaz \(a = \phi\left( \alpha \right)\) és
\(b = \phi\left( \beta \right)\) értékektől. Azt tudtuk mondani, hogy
akkor függ csak a kezdő és végpontoktól, hogyha
\(\left. \exists\Phi^{\prime} = f\Leftrightarrow\partial_{j}\Phi = f_{j} \right.\)
és \(\partial_{j}\Phi\) folytonos (minden \(j\)-re).

\begin{tetel}

Tétel:\\
Tfh \(\left. f:\Omega\rightarrow{\& reals;}^{n} \right.\) folytonosan
differenciálható függvény. Ha \(f\)-nek létezik primitív függvénye
\(\left. \Rightarrow\partial_{k}f_{j} = \partial_{j}f_{k} \right.\) az
\(\Omega\) -n.

\end{tetel}

Kérdés: abból, hogy \(\partial_{j}f_{k} = \partial_{k}f_{j}\) az
\(\Omega\)-n, következik-e, hogy \(f\)-nek van primitív függvénye (azaz
az integráljának értéke csak a kezdő és végpontoktól függ)? Általában
nem. Példa: legyen
\(\Omega: = \left\{ {x \in {\mathbb{R}}^{2}:0 < x < 2} \right\}\),
\(f_{1}\left( {x_{1},x_{2}} \right): = \frac{- x_{2}}{x_{1}^{2} + x_{2}^{2}}\),
\(f_{2}: = \frac{x_{1}}{x_{1}^{2} + x_{2}^{2}}\). Láttuk már, hogy
\(\partial_{2}f_{1} = \partial_{1}f_{2}\). Most belátjuk, hogy az
egységkörvonalon az integrál értéke nem 0 (ami ellentmond annak, hogy az
integrál értéke csak a kezdő és végpontokból függ, ami azt jelentené,
hogy létezik \(f\)-nek primitív függvénye).
\(\phi = \left( {\phi_{1},\phi_{2}} \right)\),
\(\phi_{1}\left( t \right): = \cos t,t \in \left\lbrack {0,2\pi} \right)\),
\(\phi_{2}\left( t \right): = \sin t,t \in \left\lbrack {0,2\pi} \right)\).
Ekkor \({\overset{.}{\phi}}_{1}\left( t \right) = - \sin t\) és
\({\overset{.}{\phi}}_{2}\left( t \right) = \cos t\).
\({\int_{S_{1}}{f\left( x \right)dx}} = {\int\limits_{0}^{2\pi}{\left\langle {f\left( {\phi\left( t \right)} \right),\overset{.}{\phi}\left( t \right)} \right\rangle dt}} = {\int\limits_{0}^{2\pi}{\left\lbrack {\frac{- \sin t}{\cos^{2}t + \sin^{2}t}\left( {- \sin t} \right) + \frac{\cos t}{\cos^{2}t + \sin^{2}t}\cos t} \right\rbrack dt}} = 2\pi \neq 0\).

Kvázi definíció: egy \(\Omega \subset {\mathbb{R}}^{2}\) tartományt
egyszeresen összefüggőnek nevezünk, ha tetszőleges, a tartományban levő
egyszerű zárt (szakaszosan) folytonos differenciálható görbét folytonos
mozgatással ponttá lehet húzni úgy, hogy végig a tartományban maradjunk.

\begin{definicio}

Definíció:\\
Egy \(\Omega \subset {\mathbb{R}}^{n}\) tartományt csillagszerűnek
nevezzük, ha \(\exists a \in \Omega\), hogy \(\forall x \in \Omega\)
esetén az \(a\)-t \(x\)-szel összekötő egyenes szakasz végig benne van
\(\Omega\)-ban. (Egyenes szakasz \(a\) és \(x\) pontok között:
\(L_{a,x} = \left\{ {a + t\left( {x - a} \right):t \in \left\lbrack 0,1 \right\rbrack} \right\}\))

\end{definicio}

\begin{tetel}

Tétel:\\
Legyen \(\Omega \subset {\mathbb{R}}^{n}\) csillagszerű tartomány,
\(\left. f:\Omega\rightarrow{\mathbb{R}}^{n} \right.\) folytonosan
differenciálható. Ha
\(\left. \partial_{j}f_{k} = \partial_{k}f_{j},\forall j,k\Rightarrow f \right.\)-nek
létezik primitív függvénye.

\end{tetel}

\begin{megjegyzes}

Megjegyzés:\\
A tétel kiterjeszthető egyszeresen összefüggő tartományokra is.

\end{megjegyzes}

\hypertarget{parameteres-integralok}{%
\subsection{Paraméteres integrálok}\label{parameteres-integralok}}

\begin{definicio}

Definíció:\\
Tfh
\(\left. f:\left\lbrack {a,b} \right\rbrack \times \left\lbrack {c,d} \right\rbrack\rightarrow{\mathbb{R}} \right.\)
adott legalább folytonos függvény. Értelmezzük a \(g\) függvényt:
\(g\left( x \right): = {\int\limits_{c}^{d}{f\left( {x,y} \right)dy}}\).
Ezt nevezzük \(f\) paraméteres integráljának. Miket tud ez?

\end{definicio}

\begin{tetel}

Tétel:\\
Ha
\(\left. f \in C\left( {\left\lbrack {a,b} \right\rbrack \times \left\lbrack {c,d} \right\rbrack} \right)\Rightarrow g \in C\left\lbrack {a,b} \right\rbrack \right.\).

\end{tetel}

\begin{bizonyitas}

Bizonyítás:\\
Legyen \(x_{0} \in \left\lbrack {a,b} \right\rbrack\) tetszőleges
rögzített! \[\begin{aligned}
  \left| {g\left( x \right) - g\left( {{x_0}} \right)} \right| &  = \left| {\mathop \smallint \limits_c^d f\left( {x,y} \right)dy - \mathop \smallint \limits_c^d f\left( {{x_0},y} \right)dy} \right| \\ 
   &  = \left| {\mathop \smallint \limits_c^d \left( {f\left( {x,y} \right) - f\left( {{x_0},y} \right)} \right)dy} \right| \leqslant \mathop \smallint \limits_c^d \left| {f\left( {x,y} \right) - f\left( {{x_0},y} \right)} \right|dy \\ 
\end{aligned} \] Mivel \(f\) folytonos,
\(D_{f} = \left\lbrack {a,b} \right\rbrack \times \left\lbrack {c,d} \right\rbrack\)
korlátos és zárt halmaz (ezért sorozatkompakt is), ezért \(f\)
egyenletesen folytonos (Heine tétel). Véve egy tetszőleges
\(\varepsilon > 0\) számot, ehhez
\(\left. \exists\delta > 0:\left| {\left( {x,y} \right) - \left( {x^{\ast},y^{\ast}} \right)} \right| < \delta\Rightarrow\left| {f\left( {x,y} \right) - f\left( {x^{\ast},y^{\ast}} \right)} \right| < \varepsilon \right.\).
Speciel \(x^{\ast} = x_{0}\), \(y^{\ast} = y\). Tehát
\[\left. \left| {x - x_{0}} \right| < \delta\Rightarrow\left| {f\left( {x,y} \right) - f\left( {x_{0},y} \right)} \right| < \varepsilon,\forall y \in \left\lbrack {c,d} \right\rbrack\Rightarrow\left| {g\left( x \right) - g\left( x_{0} \right)} \right| \leq \varepsilon\left( {d - c} \right) \right.\].

\end{bizonyitas}

\begin{tetel}

Tétel:\\
Ha \(f\) folytonos
\(\left\lbrack {a,b} \right\rbrack \times \left\lbrack {c,d} \right\rbrack\)-n
és \(\exists\partial_{1}f\) az
\(\left( {a,b} \right) \times \left( {c,d} \right)\)-n és létezik
folytonos kiterjesztése
\(\left\lbrack {a,b} \right\rbrack \times \left\lbrack {c,d} \right\rbrack\)-re,
akkor \(g\) függvény folytonosan differenciálható
\(\left\lbrack {a,b} \right\rbrack\)-n, és
\(g^{\prime}\left( x \right) = {\int\limits_{c}^{d}{\partial_{1}f\left( {x,y} \right)dy}}\)
(vagyis felcserélhetjük a deriválást és az integrálást).

\end{tetel}

\begin{bizonyitas}

Bizonyítás:\\
Legyen \(x_{0} \in \left( {a,b} \right) \ni x \neq x_{0}\)!
\(\frac{g\left( x \right) - g\left( x_{0} \right)}{x - x_{0}} = {\int\limits_{c}^{d}{\frac{f\left( {x,y} \right) - f\left( {x_{0},y} \right)}{x - x_{0}}dy}} =\)
a Lagrange-féle középérték-tétel felhasználásával
\(= {\int\limits_{c}^{d}{\partial_{1}f\left( {\xi_{y},y} \right)dy}}\),
ahol \(\xi_{y}\) \(x\) és \(x_{0}\) között van. \[\begin{aligned}
  \left| {\frac{{g\left( x \right) - g\left( {{x_0}} \right)}}{{x - {x_0}}} - \mathop \smallint \limits_c^d {\partial _1}f\left( {{x_0},y} \right)dy} \right| &  = \left| {\mathop \smallint \limits_c^d \left( {{\partial _1}f\left( {{\xi _y},y} \right) - {\partial _1}f\left( {{x_0},y} \right)} \right)dy} \right| \\ 
   &  \leqslant \mathop \smallint \limits_c^d \left| {{\partial _1}f\left( {{\xi _y},y} \right) - {\partial _1}f\left( {{x_0},y} \right)} \right|dy \to 0, \\ 
\end{aligned} \] mert \(\partial_{1}f\) egyenletesen folytonos.

\end{bizonyitas}

\hypertarget{a-vonalintegralrol-szolo-tetel-bizonyitasa}{%
\subsubsection{A vonalintegrálról szóló tétel
bizonyítása}\label{a-vonalintegralrol-szolo-tetel-bizonyitasa}}

Tfh \(\Omega \subset {\mathbb{R}}^{n}\) csillagszerű tartomány,
\(\left. f:\Omega\rightarrow{\mathbb{R}}^{n} \right.\) folytonosan
differenciálható, továbbá
\(\partial_{j}f_{k} = \partial_{k}f_{j},\forall j,k\)-ra az
\(\Omega\)-n. Belátjuk, hogy \(f\)-nek van primitív függvénye. Célszerű
feltétel, hogy legyen \(a = 0\), válasszunk egy \(x \in \Omega\)-t,
ekkor
\(L_{a,x} = \left\{ {t \cdot x:t \in \left\lbrack 0,1 \right\rbrack} \right\}\).
Az \(a\)-t \(x\)-szel összekötő, folytonosan differenciálható utat a
következő
\(\left. \phi:{\mathbb{R}}\rightarrow{\mathbb{R}}^{n} \right.\) függvény
határozhatja meg:
\(\phi\left( t \right): = t \cdot x,t \in \left\lbrack 0,1 \right\rbrack,x \in \Omega \subset {\mathbb{R}}^{n}\).
Legyen
\(F\left( x \right): = {\int_{L_{0,x}}{f\left( \xi \right)d\xi}} = {\int\limits_{0}^{1}{\left\langle {f\left( {\phi\left( t \right)} \right),\overset{.}{\phi}\left( t \right)} \right\rangle dt}} = {\int\limits_{0}^{1}{\left( {\sum\limits_{k = 1}^{n}{f_{k}\left( {\phi\left( t \right)} \right) \cdot x_{k}}} \right)dt}} = {\int\limits_{0}^{1}{\left( {\sum\limits_{k = 1}^{n}{f_{k}\left( {t \cdot x} \right) \cdot x_{k}}} \right)dt}}\).
Belátjuk, hogy
\(\partial_{j}F\left( x \right) = f_{j}\left( x \right)\). Most
\(F\left( x \right)\)-et differenciáljuk \(x_{j}\) paraméter szerint.
\(F\left( x \right) = {\sum\limits_{k = 1}^{n}{\int\limits_{0}^{1}{f_{k}\left( {t \cdot x} \right)x_{k}dt}}}\).\\
\(j \neq k\) esetén
\(\partial_{j}{\int\limits_{0}^{1}{f_{k}\left( {t \cdot x} \right)x_{k}dt}} = {\int\limits_{0}^{1}{\partial_{j}f_{k}\left( {t \cdot x} \right)t \cdot x_{k}dt}}\),\\
\(j = k\) esetén
\(\partial_{k}{\int\limits_{0}^{1}{f_{k}\left( {t \cdot x} \right)x_{k}dt}} = {\int\limits_{0}^{1}{\left\lbrack {\partial_{k}f_{k}\left( {t \cdot x} \right)t \cdot x_{k} + f_{k}\left( {t \cdot x} \right)} \right\rbrack dt}}\)\\
Ezért \[\begin{aligned}
  {\partial _j}F\left( x \right) &  = \mathop \smallint \limits_0^1 \left[ {\mathop \sum \limits_{k = 1}^n {\partial _j}{f_k}\left( {t \cdot x} \right)t \cdot {x_k} + {f_j}\left( {t \cdot x} \right)} \right]dt \\ 
   &  = \mathop \smallint \limits_0^1 \left[ {\mathop \sum \limits_{k = 1}^n {\partial _k}{f_j}\left( {t \cdot x} \right)t \cdot {x_k} + {f_j}\left( {t \cdot x} \right)} \right]dt. \\ 
\end{aligned} \] Legyen
\(g_{j}\left( t \right): = f_{j}\left( {t \cdot x} \right)t\), ekkor
\({\overset{.}{g}}_{j}\left( t \right) = {\sum\limits_{k = 1}^{n}{\partial_{k}f_{j}\left( {t \cdot x} \right)x_{k} \cdot t}} + f_{j}\left( {t \cdot x} \right)\),
így
\(\partial_{j}F\left( x \right) = {\int\limits_{0}^{1}{{\overset{.}{g}}_{j}\left( t \right)dt}} = g_{j}\left( 1 \right) - g_{j}\left( 0 \right) = f_{j}\left( x \right) - 0 = f_{j}\left( x \right)\).

\hypertarget{komplex-fuggvenytan}{%
\section{Komplex függvénytan}\label{komplex-fuggvenytan}}

\begin{definicio}

Definíció:\\
Tfh \(\left. f:{\mathbb{C}}\rightarrow{\mathbb{C}} \right.\) értelmezve
van egy \(z_{0} \in {\mathbb{C}}\) pont egy környezetében. Azt monjduk,
hogy \(f\) differenciálható \(z_{0}\)-ban, ha
\(\exists\lim\limits_{z\rightarrow z_{0}}\frac{f\left( z \right) - f\left( z_{0} \right)}{z - z_{0}}\)
és véges. A limeszt a valós függvények deriváláshoz hasonlóan így
jelöljük: \(f^{\prime}\left( z_{0} \right)\). A komplex
differenciálhatóságnak van geometriai szemléletes jelentése is.
\(f^{\prime}\left( z_{0} \right) \in {\mathbb{C}}\), és a komplex számok
trigonometrikus jelölésével legyen ez
\(f^{\prime}\left( z_{0} \right): = r\left( {\cos\phi + i\sin\phi} \right)\).
Tfh ez nem 0.
\(\lim\limits_{z\rightarrow z_{0}}\frac{f\left( z \right) - f\left( z_{0} \right)}{z - z_{0}} = r\left( {\cos\phi + i\sin\phi} \right)\),
így \(z_{0}\) környezetében
\(\left| \frac{f\left( z \right) - f\left( z_{0} \right)}{z - z_{0}} \right| \approx r\),
másképp:
\(\left| {f\left( z \right) - f\left( z_{0} \right)} \right| \approx r\left| {z - z_{0}} \right|\).
Erre azt mondjuk, hogy \(f\) leképezés limeszben körtartó. Másrészt
\(\arg\frac{f\left( z \right) - f\left( z_{0} \right)}{z - z_{0}} \approx \phi\),
\(\arg\left\lbrack {f\left( z \right) - f\left( z_{0} \right)} \right\rbrack - \arg\left( {z - z_{0}} \right) \approx \phi\),
\(\arg\left\lbrack {f\left( z \right) - f\left( z_{0} \right)} \right\rbrack \approx \phi + \arg\left( {z - z_{0}} \right)\),
ez egy \(z_{0}\) körüli \(\phi\) szögű forgatás.

\end{definicio}

\hypertarget{a-komplex-differencialhatosag-szukseges-feltetele}{%
\subsection{A komplex differenciálhatóság szükséges
feltétele}\label{a-komplex-differencialhatosag-szukseges-feltetele}}

Nézzük meg, hogy mit jelent az, hogy \(g\) komplex változós függvény
differenciálható egy \(z \in {\mathbb{C}}\) pontban! Így definiáltuk:
\(\lim\limits_{h\rightarrow 0}\frac{g\left( {z + h} \right) - g\left( z \right)}{h} = g^{\prime}\left( z \right)\).
Két esetet vizsgálunk,
\(h: = h_{1} \in {\mathbb{R}}\text{\textbackslash}\left\{ 0 \right\}\)
az első esetben illetve
\(h: = ih_{2},h_{2} \in {\mathbb{R}}\text{\textbackslash}\left\{ 0 \right\}\)
a második esetben. Tehát
\(\lim\limits_{h_{1}\rightarrow 0}\frac{g\left( {z + h_{1}} \right) - g\left( z \right)}{h_{1}} = \lim\limits_{h_{2}\rightarrow 0}\frac{g\left( {z + ih_{2}} \right) - g\left( z \right)}{ih_{2}}\).
Tudunk \(\mathbb{C}\) és \({\mathbb{R}}^{2}\) között egy bijekciót
létesíteni:
\(\left. J:{\mathbb{C}}\rightarrow{\mathbb{R}}^{2},x + iy\mapsto\left( {x,y} \right) \right.\)
ahol \(x,y \in {\mathbb{R}}\). Bizonyítható, hogy \(J\) lineáris
bijekció. Mint definiáltuk,
\(\left. g:{\mathbb{C}}\rightarrow{\mathbb{C}} \right.\), és legyen
\(z: = x_{1} + ix_{2}\) ahol \(x_{1}\) és \(x_{2} \in {\mathbb{R}}\),
továbbá
\(g\left( {x_{1} + ix_{2}} \right): = g_{1}\left( {x_{1} + ix_{2}} \right) + ig_{2}\left( {x_{1} + ix_{2}} \right)\),
ahol \(\left. g_{1}:{\mathbb{C}}\rightarrow{\mathbb{R}} \right.\) és
\(\left. g_{2}:{\mathbb{C}}\rightarrow{\mathbb{R}} \right.\). A korábbi
\(J\) bijekció alapján legyen
\({\widetilde{g}}_{1}\left( {x_{1},x_{2}} \right): = g_{1}\left( {x_{1} + ix_{2}} \right)\)
és
\({\widetilde{g}}_{2}\left( {x_{1},x_{2}} \right): = g_{2}\left( {x_{1} + ix_{2}} \right)\),
így
\(\left. {\widetilde{g}}_{1}:{\mathbb{R}}^{2}\rightarrow{\mathbb{R}} \right.\)
és
\(\left. {\widetilde{g}}_{2}:{\mathbb{R}}^{2}\rightarrow{\mathbb{R}} \right.\).
\[\frac{{g(z + {h_1}) - g(z)}}{{{h_1}}} = \underbrace {\frac{{{{\tilde g}_1}({x_1} + {h_1},{x_2}) - {{\tilde g}_1}({x_1},{x_2})}}{{{h_1}}}}_{ \to {\partial _1}{{\tilde g}_1}({x_1},{x_2})} + \underbrace {i\frac{{{{\tilde g}_2}({x_1} + {h_1},{x_2}) - {{\tilde g}_2}({x_1},{x_2})}}{{{h_1}}}}_{i{\partial _1}{{\tilde g}_2}({x_1},{x_2})}\]
\[\frac{{g(z + i{h_2}) - g(z)}}{{i{h_2}}} = \underbrace {\frac{{{{\tilde g}_1}({x_1},{x_2} + {h_2}) - {{\tilde g}_1}({x_1},{x_2})}}{{i{h_2}}}}_{ - i{\partial _2}{{\tilde g}_1}({x_1},{x_2})} + \underbrace {i\frac{{{{\tilde g}_2}({x_1},{x_2} + {h_2}) - {{\tilde g}_2}({x_1},{x_2})}}{{i{h_2}}}}_{{\partial _2}{{\tilde g}_2}({x_1},{x_2})}\]
Ezért kell, hogy
\(\partial_{1}{\widetilde{g}}_{1} = \partial_{2}{\widetilde{g}}_{2}, - \partial_{2}{\widetilde{g}}_{1} = \partial_{1}{\widetilde{g}}_{2}\)
legyen. Ezek a Cauchy-Riemann (parciális differenciál) egyenletek.

\begin{tetel}

Cauchy-alaptétel:\\
Tfh \(\Omega \subset {\mathbb{C}}\) egyszeresen összefüggő és
\(\left. g:\Omega\rightarrow{\mathbb{C}} \right.\) differenciálható
\(\Omega\)-n. Ekkor \(g\) integrálja bármely szakaszonként folytonosan
differenciálható, \(\Omega\)-ban haladó egyszerű zárt görbén 0.

\end{tetel}

\begin{bizonyitas}

Bizonyítás:\\
Legyen
\(\left. \phi:\left\lbrack {\alpha,\beta} \right\rbrack\rightarrow{\mathbb{C}} \right.\)
szakaszonként folytonosan differenciálható függvény, mely egy egyszerű
szakaszonként folytonosan differenciálható zárt \(\Gamma\) görbét
határoz meg, \(\phi\left( \alpha \right) = \phi\left( \beta \right)\),
\(\phi\left( t \right) \in \Omega,\forall t \in \left\lbrack {\alpha,\beta} \right\rbrack\).
Definíció szerint
\({\int_{\Gamma}{g\left( z \right)dz}}: = {\int\limits_{\alpha}^{\beta}{g\left( {\phi\left( t \right)} \right)\overset{.}{\phi}\left( t \right)dt}}\)
(itt az integrandus komplex értékű).
\(\phi\left( t \right) = \phi_{1}\left( t \right) + i\phi_{2}\left( t \right)\)
(ahol \(\phi_{1},\,\phi_{2}\) valós-valós függvények),
\(g\left( {x_{1} + ix_{2}} \right) = g_{1}\left( {x_{1} + ix_{2}} \right) + ig_{2}\left( {x_{1} + ix_{2}} \right)\).
Ezek alapján \[\begin{aligned}
   & \mathop \smallint \limits_\alpha ^\beta  g\left( {\phi \left( t \right)} \right)\dot \phi \left( t \right)dt =  \\ 
   =  & \mathop \smallint \limits_\alpha ^\beta  \left[ {{g_1}\left( {{\phi _1}\left( t \right) + i{\phi _2}\left( t \right)} \right) + i{g_2}\left( {{\phi _1}\left( t \right) + i{\phi _2}\left( t \right)} \right)} \right] \cdot \left[ {{{\dot \phi }_1}\left( t \right) + i{{\dot \phi }_2}\left( t \right)} \right]dt .\\ 
\end{aligned} \] . Definiáljuk \(\psi\) függvényt:
\(\left. \psi:{\mathbb{R}}\rightarrow{\mathbb{R}}^{2},\quad t\mapsto\left( {\phi_{1}\left( t \right),\phi_{2}\left( t \right)} \right) \right.\),
így az előző integrálban a szorzást elvégezve \[\begin{aligned}
  \int_\Gamma  {g\left( z \right)dz}  =  & \mathop \smallint \limits_\alpha ^\beta  \left[ {{g_1}\left( {\phi \left( t \right)} \right){{\dot \phi }_1}\left( t \right) - {g_2}\left( {\phi \left( t \right)} \right){{\dot \phi }_2}\left( t \right)} \right]dt +  \\ 
   &  + i\mathop \smallint \limits_\alpha ^\beta  \left[ {{g_1}\left( {\phi \left( t \right)} \right){{\dot \phi }_2}\left( t \right) + {g_2}\left( {\phi \left( t \right)} \right){{\dot \phi }_1}\left( t \right)} \right]dt \\ 
   =  & \mathop \smallint \limits_\alpha ^\beta  \left[ {{{\tilde g}_1}\left( {\psi \left( t \right)} \right){{\dot \phi }_1}\left( t \right) - {{\tilde g}_2}\left( {\psi \left( t \right)} \right){{\dot \phi }_2}\left( t \right)} \right]dt \\ 
   &  + i\mathop \smallint \limits_\alpha ^\beta  \left[ {{{\tilde g}_1}\left( {\psi \left( t \right)} \right){{\dot \phi }_2}\left( t \right) + {{\tilde g}_2}\left( {\psi \left( t \right)} \right){{\dot \phi }_1}\left( t \right)} \right]dt. \\ 
\end{aligned} \] (Ebben \({\widetilde{g}}_{1}\) és
\({\widetilde{g}}_{2}\) már
\(\left. {\mathbb{R}}^{2}\rightarrow{\mathbb{R}} \right.\) függvények.)
Belátjuk, hogy mindkét tag 0.

\begin{itemize}
\tightlist
\item
  \(\left. f: = \left( {f_{1},f_{2}} \right):{\mathbb{R}}^{2}\rightarrow{\mathbb{R}}^{2} \right.\).
  Először legyen \(f_{1}: = {\widetilde{g}}_{1}\),
  \(f_{2}: = - {\widetilde{g}}_{2}\). Ekkor
  \(\partial_{2}f_{1} = \partial_{2}{\widetilde{g}}_{1}\) és
  \(\partial_{1}f_{2} = - \partial_{1}{\widetilde{g}}_{2}\), (a
  Cauchy-Riemann egyenletekből pedig)
  \(\partial_{2}{\widetilde{g}}_{1} = - \partial_{1}{\widetilde{g}}_{2}\),
  azaz \(\partial_{2}f_{1} = \partial_{1}f_{2}\). Így a valós
  vonalintegrálokról szóló tétel szerint
  \(0 = {\int_{L}{f\left( x \right)dx}} = {\int\limits_{\alpha}^{\beta}{\left\lbrack {f_{1}\left( {\psi\left( t \right)} \right){\overset{.}{\phi}}_{1}\left( t \right) + f_{2}\left( {\psi\left( t \right)} \right){\overset{.}{\phi}}_{2}\left( t \right)} \right\rbrack dt}} = {\int\limits_{\alpha}^{\beta}{\left\lbrack {{\widetilde{g}}_{1}\left( {\psi\left( t \right)} \right){\overset{.}{\phi}}_{1}\left( t \right) - {\widetilde{g}}_{2}\left( {\psi\left( t \right)} \right)} \right\rbrack{\overset{.}{\phi}}_{2}dt}}\).
\item
  Másodszor \(f_{1}: = {\widetilde{g}}_{2}\),
  \(f_{2}: = {\widetilde{g}}_{1}\), ebből kapjuk, hogy a második
  integrál is 0. A Cauchy-Riemann egyenletekből most
  \(\partial_{1}{\widetilde{g}}_{1} = \partial_{2}{\widetilde{g}}_{2}\)
  illetve
  \(- \partial_{2}{\widetilde{g}}_{1} = \partial_{1}{\widetilde{g}}_{2}\).
  Ekkor \(\partial_{1}f_{2} = \partial_{2}f_{1}\), így
  \(0 = {\int_{L}{f\left( x \right)dx}} = {\int\limits_{\alpha}^{\beta}{\left\lbrack {f_{1}\left( {\psi\left( t \right)} \right){\overset{.}{\phi}}_{1}\left( t \right) + f_{2}\left( {\psi\left( t \right)} \right){\overset{.}{\phi}}_{2}} \right\rbrack dt}} = {\int\limits_{\alpha}^{\beta}{\left\lbrack {{\widetilde{g}}_{2}\left( {\psi\left( t \right)} \right){\overset{.}{\phi}}_{1}\left( t \right) + {\widetilde{g}}_{1}\left( {\psi\left( t \right)} \right){\overset{.}{\phi}}_{2}} \right\rbrack dt}}\).
\end{itemize}

\end{bizonyitas}

\begin{megjegyzes}

Megjegyzés:\\
A bizonyításokban felhasználtuk, hogy \(g\) folytonosan
differenciálható, így felhasználtuk a
\protect\hyperlink{zartgorbe}{valós vonalintegrálról szóló tételt és
megjegyzését.}

\end{megjegyzes}

\begin{ajanlo}

\begin{ajanlofig}

\href{https://xkcd.com}{\includegraphics[width=5.20833in,height=2.82292in]{wikipedian_protester.png}}

\end{ajanlofig}

Text

\end{ajanlo}

\hypertarget{a-cauchy-alaptetel-kozvetlen-kovetkezmenyei}{%
\subsection{A Cauchy alaptétel közvetlen
következményei}\label{a-cauchy-alaptetel-kozvetlen-kovetkezmenyei}}

Tfh
\(\left. \phi:\left\lbrack {\alpha,\beta} \right\rbrack\rightarrow{\mathbb{C}} \right.\)
szakaszonként folytonosan differenciálható függvény egyszerű zárt utat
határoz meg, \(\Gamma: = R_{\phi}\) egy egyszerű zárt szakaszonként
folytonosan differencálható görbe \(\mathbb{C}\)-n.

\begin{tetel}

Tétel:\\
Egy \({\mathbb{C}}\text{\textbackslash}\Gamma\) nyílt halmaz két
összefüggő komponensből (részből) áll, a két komponens közül az egyik
korlátos, a másik nem. A korlátos komponenst nevezzük \(\Gamma\)
belsejének, a nem korlátos komponenst \(\Gamma\) külsejének.

\end{tetel}

\begin{megjegyzes}

Megjegyzés:\\
A tétel állítása triviálisnak tűnhet, bizonyítása mégsem könnyű.

\end{megjegyzes}

\begin{tetel}

Tétel:\\
Legyen \(\Gamma,\Gamma_{1} \subset {\mathbb{C}}\) egyszerű zárt
szakaszonként folytonosan differenciálható görbe, legyen \(\Gamma_{1}\)
a \(\Gamma\) görbe belsejében. Tfh
\(\Omega \supset \left( {\Gamma \cup \Gamma_{1} \cup \left( {{belseje}\left( \Gamma \right)\text{\textbackslash}\left( {\Gamma_{1} \cup {belseje}\left( \Gamma_{1} \right)} \right)} \right)} \right)\),
vagyis hogy \(\Omega\) tartalmazza \(\Gamma,\Gamma_{1}\)-t és a kettejük
közötti tartományt. Legyen
\(\left. g:\Omega\rightarrow{\mathbb{C}} \right.\) differenciálható
függvény. Ekkor
\({\int_{\Gamma}{f\left( z \right)dz}} = {\int_{\Gamma_{1}}{f\left( z \right)dz}}\).

\end{tetel}

Ábra: ide kell betenni a gorbek\_gamma.eps filet.

\begin{bizonyitas}

Bizonyítás:\\
A Cauchy alaptételt alkalmazva a
\(\Gamma,\Gamma_{2}, - \Gamma_{1},\Gamma_{3}\) utakból álló
szakaszonként folytonosan differenciálható zárt görbére:
\(0 = {\int_{\Gamma}{f\left( z \right)dz}} + {\int_{\Gamma_{2}}{f\left( z \right)dz}} - {\int_{\Gamma_{1}}{f\left( z \right)dz}} + {\int_{\Gamma_{3}}{f\left( z \right)dz}}\),
mely limeszben \(\left. \Gamma_{2}\rightarrow\Gamma_{3} \right.\))
\(0 = {\int_{\Gamma}{f\left( z \right)dz}} - {\int_{\Gamma_{1}}{f\left( z \right)dz}}\).
(Az ábrán a körüljárást láthatjuk, \(\Gamma,\Gamma_{1}\) irányítása
azonos, óramutató járásával ellentétes irányú. Ez a bizonyítás csupán
vázlatos, szemléletes.)

\end{bizonyitas}

\begin{tetel}

Tétel:\\
Legyenek \(\Gamma,\Gamma_{1},\Gamma_{2},...,\Gamma_{k}\) egyszerű zárt
szakaszonként folytonosan differenciálható görbék,
\(\Gamma_{j} \subset {belseje}\left( \Gamma \right)\) és
\(\Gamma_{l} \subset kulseje\left( \Gamma_{j} \right)\) ha \(l \neq j\),
\(\forall j,l \in \left\{ {0,1,...,k} \right\}\). Legyen \(g\) függvény
differenciálható egy olyan tartományban, mely tartalmazza
\(\Gamma,\Gamma_{1},\Gamma_{2},...,\Gamma_{k}\)-t és
\({belseje}\left( \Gamma \right)\text{\textbackslash}\left( {\underset{j = 1}{\overset{k}{\cup}}{{belseje}\left( \Gamma_{j} \right)}} \right)\)-t
is. Ekkor
\({\int_{\Gamma}{f\left( z \right)dz}} = {\sum\limits_{j = 1}^{k}{\int_{\Gamma_{j}}{f\left( z \right)dz}}}\).

\end{tetel}

\hypertarget{cauchy-fele-integralformula}{%
\subsubsection{Cauchy-féle
integrálformula}\label{cauchy-fele-integralformula}}

\begin{tetel}

Tétel:\\
Legyen \(\Omega \subset {\mathbb{C}}\) egyszeresen összefüggő tartomány
és \(\Gamma \subset \Omega\) egyszerű zárt szakaszonként folytonosan
differenciálható görbe (ekkor
\({belseje}\left( \Gamma \right) \subset \Omega\)) és \(g\) diffható
\(\Omega\) -n. Ekkor \(\Gamma\) belsejében fekvő bármely \(z\) esetén
\(g\left( z \right) = \frac{1}{2\pi i}{\int_{\Gamma}{\frac{g\left( \zeta \right)}{\zeta - z}d\zeta}}\).

\end{tetel}

\begin{bizonyitas}

Bizonyítás:\\
Legyen
\(K_{\rho}: = \left\{ {\zeta \in {\mathbb{R}}:\left| {\zeta - z} \right| = \rho} \right\}\)
a \(z\) középpontú, \(\rho\) sugarú körvonal. \(\rho\)-t olyan kicsinek
választjuk, hogy \(K_{\rho} \subset \Gamma\) belseje lesz már. Ekkor a
Cauchy alaptétel közvetlen következménye szerint
\({\int_{\Gamma}{\frac{g\left( \zeta \right)}{\zeta - z}d\zeta}} = {\int_{K_{\rho}}{\frac{g\left( \zeta \right)}{\zeta - z}d\zeta}}\),
másrészt \({\int_{K_{\rho}}{\frac{1}{\zeta - z}d\zeta}} = 2\pi i\),
ugyanis
\(\zeta = z + \rho\cos\left( t \right) + i\rho\sin\left( t \right)\)
paraméterezés mellett
\(\zeta - z = \rho\cos\left( t \right) + i\rho\sin\left( t \right)\)--
erre valóban teljesül a
\(K_{\rho} = \left\{ {\zeta \in {\mathbb{R}}:\left| {\zeta - z} \right| = \rho} \right\}\)
kitétel --, így \(K_{\rho}\) előáll a
\(\left. \phi:\left\lbrack {0,2\pi} \right\rbrack\rightarrow{\mathbb{C}} \right.\),
\(\phi\left( t \right) = z + \rho\cos\left( t \right) + i\rho\sin\left( t \right)\)
függvény segítségével (ekkor
\(\phi^{\prime}\left( t \right) = - \rho\sin\left( t \right) + i\rho\cos\left( t \right)\)):
\(K_{\rho} = R_{\phi}\), továbbá
\[{\int_{K_{\rho}}{\frac{1}{\zeta - z}d\zeta}} = \int\limits_{0}^{2\pi}\frac{1}{\rho\cos\left( t \right) + i\rho\sin\left( t \right)}\left\lbrack {- \rho\sin\left( t \right) + i\rho\cos\left\lbrack t \right\rbrack} \right\rbrack dt = \int\limits_{0}^{2\pi}idt = 2\pi i.\]Ezek
szerint
\(g\left( z \right) = \frac{1}{2\pi i}g\left( z \right)2\pi i = \frac{1}{2\pi i}g\left( z \right){\int_{K_{\rho}}{\frac{1}{\zeta - z}d\zeta}} = \frac{1}{2\pi i}{\int_{K_{\rho}}{\frac{g\left( z \right)}{\zeta - z}d\zeta}}\).
Vizsgáljuk a következő mennyiséget:
\(\left| {\frac{1}{2\pi i}{\int_{K_{\rho}}{\frac{g\left( \zeta \right)}{\zeta - z}d\zeta}} - g\left( z \right)} \right| \leq \frac{1}{2\pi}\left| {{\int_{K_{\rho}}{\frac{g\left( \zeta \right)}{\zeta - z}d\zeta}} - {\int_{K_{\rho}}{\frac{g\left( z \right)}{\zeta - z}d\zeta}}} \right| = \frac{1}{2\pi}\left| {\int_{K_{\rho}}{\frac{g\left( \zeta \right) - g\left( z \right)}{\zeta - z}d\zeta}} \right| \leq\)
\[\begin{aligned}
  \left| {\frac{1}{{2\pi i}}\int_{{K_\rho }} {\frac{{g\left( \zeta  \right)}}{{\zeta  - z}}d\zeta }  - g\left( z \right)} \right| &  \leqslant \frac{1}{{2\pi }}\left| {\int_{{K_\rho }} {\frac{{g\left( \zeta  \right)}}{{\zeta  - z}}d\zeta }  - \int_{{K_\rho }} {\frac{{g\left( z \right)}}{{\zeta  - z}}d\zeta } } \right| \\ 
   &  = \frac{1}{{2\pi }}\left| {\int_{{K_\rho }} {\frac{{g\left( \zeta  \right) - g\left( z \right)}}{{\zeta  - z}}d\zeta } d\zeta } \right| \\ 
   &  \leqslant \frac{1}{{2\pi }}{\text{kerület}}\left( {{K_\rho }} \right) \cdot \mathop {\sup }\limits_{\zeta  \in {K_\rho }} \left| {\frac{{g\left( \zeta  \right) - g\left( z \right)}}{{\zeta  - z}}} \right| < \varepsilon,  \\ 
\end{aligned} \] ugyanis
\(\left| {g\left( \zeta \right) - g\left( z \right)} \right| < \varepsilon\)
ha \(\rho < \rho_{0}\),
\(\left| {\zeta - z} \right| = \rho = \text{állandó}\),
\(\begin{matrix} {\left| {\zeta - z} \right| = \rho = \text{állandó}} \\ {\text{kerület}\left( K_{\rho} \right) = 2\pi\rho} \\ \end{matrix}\),
tehát
\(\left. \left| {\frac{1}{2\pi i}{\int_{\Gamma}{\frac{g\left( \zeta \right)}{\zeta - z}d\zeta}} - g\left( z \right)} \right| < \varepsilon,\forall\varepsilon > 0\Rightarrow\frac{1}{2\pi i}{\int_{\Gamma}{\frac{g\left( \zeta \right)}{\zeta - z}d\zeta}} = g\left( z \right) \right.\).

\end{bizonyitas}

\hypertarget{cauchy-tipusu-integral}{%
\subsubsection{Cauchy-típusú integrál}\label{cauchy-tipusu-integral}}

\begin{definicio}

Definíció:\\
Legyen \(\Gamma\) egyszerű (nem feltételen zárt) szakaszonként
folytonosan differenciálható görbe. Legyen
\(\left. g:\Gamma\rightarrow{\mathbb{C}} \right.\) folytonos függvény!
Legyen
\(G(z): = \frac{1}{2\pi i}{\int_{\Gamma}{\frac{g\left( \zeta \right)}{\zeta - z}d\zeta}}\),
ezt nevezzük Cauchy-típusú integrálnak, ha
\(z \in {\mathbb{C}}\text{\textbackslash}\Gamma\).

\end{definicio}

\begin{tetel}

Tétel:\\
\(G\) függvény a \({\mathbb{C}}\text{\textbackslash}\Gamma\) nyílt
halmazon akárhányszor differenciálható és
\(G^{(k)}\left( z \right) = \frac{k!}{2\pi i}{\int_{\Gamma}{\frac{g\left( \zeta \right)}{\left( {\zeta - z} \right)^{k + 1}}d\zeta}}\).

\end{tetel}

\begin{bizonyitas}

Bizonyítás:\\
Csak a \(k = 1\) esetet látjuk be, teljes indukcióval a tétel
igazolható. Tehát ezt szeretnénk igazolni:
\(G^{\prime}\left( z \right) = \frac{1}{2\pi i}{\int_{\Gamma}{\frac{g\left( \zeta \right)}{\left( {\zeta - z} \right)^{2}}d\zeta}}\).
\[\begin{aligned}
  \frac{{G\left( {z + h} \right) - G\left( z \right)}}{h} &  = \frac{1}{h}\frac{1}{{2\pi i}}\left[ {\int_\Gamma  {\frac{{g\left( \zeta  \right)}}{{\zeta  - z - h}}d\zeta }  - \int_\Gamma  {\frac{{g\left( \zeta  \right)}}{{\zeta  - z}}d\zeta } } \right] \\ 
   &  = \frac{1}{{2\pi i}}\int_\Gamma  {\frac{1}{h}\left[ {\frac{{g\left( \zeta  \right)}}{{\zeta  - z - h}} - \frac{{g\left( \zeta  \right)}}{{\zeta  - z}}} \right]d\zeta }  \\ 
   &  = \frac{1}{{2\pi i}}\int_\Gamma  {g\left( \zeta  \right)\frac{{\left( {\zeta  - z} \right) - \left( {\zeta  - z - h} \right)}}{{h\left( {\zeta  - z - h} \right)\left( {\zeta  - z} \right)}}d\zeta }  \\ 
   &  = \frac{1}{{2\pi i}}\int_\Gamma  {\frac{{g\left( \zeta  \right)}}{{\left( {\zeta  - z - h} \right)\left( {\zeta  - z} \right)}}}  \\ 
\end{aligned} \] Vizsgáljuk: \[\begin{aligned}
  I &  = \left| {\frac{{G\left( {z + h} \right) - G\left( z \right)}}{h} - \frac{1}{{2\pi i}}\int_\Gamma  {\frac{{g\left( \zeta  \right)}}{{{{\left( {\zeta  - z} \right)}^2}}}d\zeta } } \right| \\ 
   &  = \frac{1}{{2\pi }}\left| {\int_\gamma  {g\left( \zeta  \right)\left[ {\frac{1}{{\left( {\zeta  - z - h} \right)\left( {\zeta  - z} \right)}} - \frac{1}{{{{\left( {\zeta  - z} \right)}^2}}}} \right]d\zeta } } \right| \\ 
   &  = \frac{1}{{2\pi }}\left| {\mathop \smallint \limits_\Gamma  g\left( \zeta  \right)\frac{{\left( {\zeta  - z} \right) - \left( {\zeta  - z - h} \right)}}{{\left( {\zeta  - z - h} \right){{\left( {\zeta  - z} \right)}^2}}}d\zeta } \right| \\ 
   &  = \frac{{\left| h \right|}}{{2\pi }}\left| {\mathop \smallint \limits_\Gamma  \frac{{g\left( \zeta  \right)}}{{\left( {\zeta  - z - h} \right){{\left( {\zeta  - z} \right)}^2}}}d\zeta } \right| \\ 
   &  \leqslant \frac{{\left| h \right|}}{{2\pi }} \cdot {\text{\'i vhossz}}\left( \Gamma  \right) \cdot \mathop {\sup }\limits_{\zeta  \in \Gamma } \frac{1}{{\left| {\zeta  - z - h} \right| \cdot {{\left| {\zeta  - z} \right|}^2}}} .\\ 
\end{aligned} \] \(\Gamma \subset {\mathbb{C}}\) korlátos és zárt, ezért
sorozatkompakt is, így a
\(\left. \zeta\mapsto\left| {\zeta - z} \right|,\zeta \in \Gamma \right.\)
folytonos függvényhez
\(\exists {\zeta _0} \in \Gamma :0 < \underbrace {\left| {{\zeta _0} - z} \right|}_{: = 2d} = \mathop {\inf }\limits_{\zeta \in \Gamma } \left| {\zeta - z} \right| \Rightarrow \left| {\zeta - z} \right| \geqslant 2d,\)
ha \(\zeta \in \Gamma\). Ha
\(\left. \left| h \right| \leq d\Rightarrow\left| {\zeta - z - h} \right| \geq d \right.\),
ugyanis \[\begin{gathered}
  \zeta  - z = \left( {\zeta  - z - h} \right) + h \\ 
   \Downarrow  \\ 
  \left| {\zeta  - z} \right| \leqslant \left| {\zeta  - z - h} \right| + \left| h \right| \\ 
   \Downarrow  \\ 
  \left| {\zeta  - z - h} \right| \geqslant \left| {\zeta  - z} \right| - \left| h \right| = d .\\ 
\end{gathered} \]
\(I \leqslant \frac{{\left| h \right|}}{{2\pi }} \cdot \text{ívhossz} \left( \Gamma \right)\frac{1}{{d\left( {2{d^2}} \right)}} \to 0,\)
ha \(\left. h\rightarrow 0 \right.\). Tehát
\[\left. \frac{G\left( {z + h} \right) - G\left( z \right)}{h}\rightarrow\frac{1}{2\pi i}{\int_{\Gamma}{\frac{g\left( \zeta \right)}{\left( {\zeta - z} \right)^{2}}d\zeta}} = G^{\prime}{\left( z \right).} \right.\]

\end{bizonyitas}

Spec eset: \(\Gamma \subset {\mathbb{C}}\) egyszerű zárt szakaszonként
folytonosan differenciálható görbe,
\(z \in \text{belseje}\left( \Gamma \right)\), \(g\) differenciálható
egy \(\Gamma\)-t tartalmazó egyszeresen összefüggő tartományon. Ekkor
\(g\left( z \right) = \frac{1}{2\pi i}{\int_{\Gamma}{\frac{g\left( \zeta \right)}{\zeta - z}d\zeta}}\),
az utóbbi tétel szerint pedig
\(\exists g^{(k)}\left( z \right) = \frac{k!}{2\pi i}{\int_{\Gamma}{\frac{g\left( \zeta \right)}{\left( {\zeta - z} \right)^{k + 1}}d\zeta}}\).
Eszerint ha egy komplex függvény egyszer differenciálható, akkor
akárhányszor differenciálható.

\hypertarget{a-primitiv-fuggveny-es-a-vonalintegral-kapcsolata}{%
\subsection{A primitív függvény és a vonalintegrál
kapcsolata}\label{a-primitiv-fuggveny-es-a-vonalintegral-kapcsolata}}

\begin{tetel}

Tétel:\\
Tfh \(g\) folytonos egy \(\Omega \subset {\mathbb{C}}\) tartományon,
továbbá \(\int_{\Gamma}{g\left( z \right)dz}\) vonalintegrál értéke
tetszőleges \(\Omega\)-ban haladó egyszerű szakaszonként folytonosan
differenciálható görbe esetén annak csak a kezdő és végpontjaitól függ.
Legyen \(a \in \Omega\) rögzített, \(z \in \Omega\) változó pont,
\(\Phi\left( z \right): = {\int\limits_{a}^{z}{g\left( \zeta \right)d\zeta}}\).
Ekkor \(\Phi^{\prime}\left( z \right) = g\left( z \right)\).

\end{tetel}

\begin{bizonyitas}

Bizonyítás:\\
\(\frac{\Phi\left( {z + h} \right) - \Phi\left( z \right)}{h} = \frac{1}{h}\left\lbrack {{\int\limits_{a}^{z + h}{g\left( \zeta \right)d\zeta}} - {\int\limits_{a}^{z}{g\left( \zeta \right)d\zeta}}} \right\rbrack = \frac{1}{h}{\int\limits_{z}^{z + h}{g\left( \zeta \right)d\zeta}}\).
Vegyük észre, hogy \[\begin{aligned}
  \left| {\frac{{\Phi (z + h) - \Phi (z)}}{h} - g(z)} \right| &  = \left| {\frac{1}{h}\mathop \smallint \limits_z^{z + h} g\left( \zeta  \right)d\zeta  - \frac{1}{h}\mathop \smallint \limits_z^{z + h} g\left( z \right)d\zeta } \right| \\ 
   &  = \frac{1}{{\left| h \right|}}\left| {\mathop \smallint \limits_z^{z + h} \left( {g\left( \zeta  \right) - g\left( z \right)} \right)d\zeta } \right| \\ 
   &  \leqslant \frac{1}{{\left| h \right|}}\left| h \right|\mathop {\sup }\limits_{\zeta  \in L\left( {z,z + h} \right)} \left| {g\left( \zeta  \right) - g\left( z \right)} \right|, \\ 
\end{aligned} \] mely 0-hoz tart, ha \(\left| h \right|\) is.

\end{bizonyitas}

\begin{tetel}

Következmény (Morera tétele):\\
Tfh \(g\) egy \(\Omega\) tartományon értelmezett folytonos függvény,
amelynek az \(\Omega\)-ban haladó egyszerű szakaszonként folytonosan
differenciálható görbéken vett integrálja csak a kezdő és végpontoktól
függ. Ekkor \(g\) differenciálható \(\Omega\)-n (vagyis akárhányszor
differenciálható). Ugyanis előbbi tétel szerint a
\(\Phi\left( z \right) = {\int\limits_{a}^{z}{g\left( \zeta \right)d\zeta}}\)
függvényre \(\Phi\) differenciálható és
\(\Phi^{\prime}\left( z \right) = g\left( z \right)\),tehát \(\Phi\)
egyszer differenciálható, ezért akárhányszor, így \(g\) is akárhányszor.

\end{tetel}

\begin{ajanlo}

\begin{ajanlofig}

\href{https://xkcd.com}{\includegraphics[width=5.20833in,height=2.82292in]{wikipedian_protester.png}}

\end{ajanlofig}

Text

\end{ajanlo}

\begin{definicio}

Definíció:\\
Ha \(f\) az \(\Omega \subset {\mathbb{C}}\) tartomány minden pontjában
differenciálható, akkor \(f\)-et holomorfnak nevezzük \(\Omega\) -n.

\end{definicio}

\hypertarget{taylor-sorfejtes-komplex-fuggvenyeken}{%
\subsection{Taylor-sorfejtés komplex
függvényeken}\label{taylor-sorfejtes-komplex-fuggvenyeken}}

\begin{allitas}

Lemma:\\
Legyen \(\Gamma \subset {\mathbb{C}}\) egyszerű, szakaszonként
folytonosan differenciálható görbe, s legyenek
\(\left. f_{k}:\Gamma\rightarrow{\mathbb{C}} \right.\) folytonos
függvények, \(k \in {\mathbb{N}}\). Tfh a
\({\sum\limits_{k = 1}^{\infty}f_{k}} = f\) sor egyenletesen konvergens.
Ekkor \(f\) is folytonos (valósban bizonyítottuk, de állítás, hogy
komplexben is így van). Ekkor
\({\int{f\left( z \right)dz}} = {\sum\limits_{k = 1}^{\infty}{\int{f_{k}\left( z \right)dz}}}\),
vagyis az integrálás és az összegzés felcserélhető.

\end{allitas}

\begin{bizonyitas}

Bizonyítás:\\
Legyen
\(\left. \phi:\left\lbrack {\alpha,\beta} \right\rbrack\rightarrow{\mathbb{C}} \right.\),
\(R_{\phi} = \Gamma\), \(\phi\) szakaszonként folytonosan
differenciálható. Ekkor
\({\int_{\Gamma}{f\left( z \right)dz}} = {\int\limits_{\alpha}^{\beta}{f\left( {\phi\left( t \right)} \right)\overset{.}{\phi}\left( t \right)}}\)
és
\({\int_{\Gamma}{f_{k}\left( z \right)dz}} = {\int\limits_{\alpha}^{\beta}{f_{k}\left( {\phi\left( t \right)} \right)\overset{.}{\phi}\left( t \right)}}\),
így \[\begin{aligned}
  \int_\Gamma  {f\left( z \right)dz}  &  = \mathop \smallint \limits_\alpha ^\beta  \left[ {f\left( {\phi \left( t \right)} \right)} \right]\dot \phi \left( t \right)dt \\ 
   &  = \mathop \smallint \limits_\alpha ^\beta  \underbrace {\left[ {\mathop \sum \limits_{k = 1}^\infty  {f_k}\left( {\phi \left( t \right)} \right)\dot \phi \left( t \right)} \right]}_{{\text{szakaszonk\'e nt folytonos}}}dt \\ 
   &  = \mathop \sum \limits_{k = 1}^\infty  \mathop \smallint \limits_\alpha ^\beta  {f_k}\left( {\phi \left( t \right)} \right)\dot \phi \left( t \right)dt \\ 
   &  = \mathop \sum \limits_{k = 1}^\infty  \int_\Gamma  {{f_k}\left( z \right)dz} . \\ 
\end{aligned} \]

\end{bizonyitas}

\hypertarget{egyenletes-konvergencia}{%
\subsection{Egyenletes konvergencia}\label{egyenletes-konvergencia}}

\hypertarget{weierstrass-tetele-komplex-fuggvenyekre}{%
\subsubsection{Weierstrass-tétele komplex
függvényekre}\label{weierstrass-tetele-komplex-fuggvenyekre}}

\begin{definicio}

Definíció:\\
Legyen
\(\left. f_{k}:\Omega\rightarrow{\mathbb{C}},f_{k} \in C\left( D_{f_{k}} \right),\Omega \subset {\mathbb{C}} \right.\)
tartomány. Azt mondjuk, hogy a
\({\sum\limits_{k = 1}^{\infty}f_{k}} = f\) az \(\Omega\) belsejében
egyenletesen konvergens, ha \(\forall K \subset \Omega\) sorozatkompakt
halmaz esetén a sor \(K\)-n egyenletesen konvergens.

\end{definicio}

\begin{tetel}

Weierstrass tétele:\\
Legyen \(\Omega \subset {\mathbb{C}}\) tartomány,
\(\left. f_{k}:\Omega\rightarrow{\mathbb{C}} \right.\) függvények
holomorfak, továbbá \(\sum\limits_{k = 1}^{\infty}f_{k}\) sor a
\(\Omega\) belsejében egyenletesen konvergens. Ekkor

\begin{enumerate}
\def\labelenumi{\arabic{enumi}.}
\tightlist
\item
  \(f: = {\sum\limits_{k = 1}^{\infty}f_{k}}\) is holomorf
\item
  \(f^{\prime} = {\sum\limits_{k = 1}^{\infty}{f_{k}{}^{\prime}}}\)
\item
  az utóbbi sor is egyenletesen konvergens \(\Omega\) belsejében
\end{enumerate}

\end{tetel}

Következmény: \(f^{(j)} = {\sum\limits_{k = 1}^{\infty}f_{k}^{(j)}}\) is
egyenletesen konvergens \(\Omega\) belsejében.

\begin{bizonyitas}

Bizonyítás:

\begin{enumerate}
\def\labelenumi{\arabic{enumi}.}
\tightlist
\item
  Egyrészt tudjuk, hogy \(f = {\sum\limits_{k = 1}^{\infty}f_{k}}\)
  folytonos \(\Omega\)-n (hiszen a sor \(\Omega\) belsejében
  egyenletesen konvergens). Legyen \(z_{0} \in \Omega\) rögzített
  pontja. Belátjuk, hogy \(f\) differenciálható \(z_{0}\) egy kis
  \(K_{r}\left( z_{0} \right)\) környezetében. Vegyünk egy
  \(K_{r}\left( z_{0} \right)\)-ban haladó, egyszerű szakaszonként
  folytonosan differenciálható zárt \(\Gamma\) görbét. Belátjuk, hogy
  \(\left. {\int_{\Gamma}{f\left( z \right)dz}} = 0\Rightarrow f \right.\)
  holomorf \(K_{r}\left( z_{0} \right)\)-n (Morera tétele miatt). Az
  előbbi lemma alapján
  \[\int_\Gamma  {f\left( z \right)dz}  = \int_\Gamma  {\underbrace {\mathop \sum \limits_{k = 1}^\infty  {f_k}\left( z \right)}_{\Gamma {\text{ - n\;egyenl}}{\text{.\;konv}}{\text{.}}}dz}  = \mathop \sum \limits_{k = 1}^\infty  \int_\Gamma  {{f_k}\left( z \right)dz}  = 0. \]
\item
  A Cauchy-féle integrálformula szerint ha \(z\) a
  \(K_{r}\left( z_{0} \right): = \left\{ {z:\left| {z - z_{0}} \right| = r} \right\}\)
  körvonal belsejében van, akkor
  \(f\left( z \right) = \frac{1}{2\pi}{\int_{K_{r}{(z_{0})}}{\frac{f\left( \zeta \right)}{\zeta - z}d\zeta}}\).
  Ekkor
  \(f^{\prime}\left( z \right) = \frac{1}{2\pi i}{\int_{K_{r}{(z_{0})}}{\frac{f\left( \zeta \right)}{\left( {\zeta - z} \right)^{2}}d\zeta}} = \frac{1}{2\pi i}{\int_{K_{r}{(z_{0})}}{\frac{\sum\limits_{k = 1}^{\infty}{f_{k}\left( \zeta \right)}}{\left( {\zeta - z} \right)^{2}}d\zeta}} = {\sum\limits_{k = 1}^{\infty}{\frac{1}{2\pi i}{\int_{K_{r}{(z_{0})}}{\frac{f_{k}\left( \zeta \right)}{\left( {\zeta - z} \right)^{2}}d\zeta}}}} = {\sum\limits_{k = 1}^{\infty}{f_{k}{}^{\prime}}}\left( z \right)\)
  (itt is felhasználtuk a sor egyenletes konvergenciáját
  \(K_{r}\left( z_{0} \right)\)-n).
\end{enumerate}

\end{bizonyitas}

További következmény: tekintsük a
\(\sum\limits_{k = 0}^{\infty}{c_{k}\left( {z - z_{0}} \right)^{k}}\)
hatványsort! Tfh ennek konvergencia sugara \(R > 0\). Tudjuk, hogy
\(\left| {z - z_{0}} \right| < R\) esetén a sor konvergens, ill minden
\(R\)-nél kisebb sugarú, \(z_{0}\) középpontú körben a hatványsor
egyenletesen konvergens. Mivel
\(f_{k}\left( z \right) = c_{k}\left( {z - z_{0}} \right)^{k}\) holomorf
függvény, és az \(f_{k}\) függvényekből álló sor a konvergencia sugár
belsejében egyenletesen konvergens, így a Weierstrass tételből
következően a sor összege is holomorf, és a sor tagonként akárhányszor
deriválható. Továbbá
\(f\left( z \right): = {\sum\limits_{k = 0}^{\infty}{c_{k}\left( {z - z_{0}} \right)^{k}}}\),
mivel a sor most is (komplex értelemben) tagonként differenciálható,
ezért egyszerű számolással kapjuk:
\(c_{k} = \frac{f^{(k)}\left( z_{0} \right)}{k!}\).

\begin{definicio}

Definíció:\\
Tfh \(f\) holomorf függvény \(z_{0}\) egy környezetében. Ekkor az \(f\)
függvény Taylor sorát így értelmezzük:
\(\sum\limits_{k = 1}^{\infty}{\frac{f^{(k)}\left( z_{0} \right)}{k!}\left( {z - z_{0}} \right)^{k}}\).

\end{definicio}

\begin{tetel}

Tétel:\\
Legyen \(\Omega \subset {\mathbb{C}}\) tartomány, tfh \(f\) holomorf
\(\Omega\)-n, \(z_{0} \in \Omega\). Tekintsük az \(f\) függvény
Taylor-sorát \(z_{0}\) körül! Ekkor
\(f\left( z \right) = {\sum\limits_{k = 0}^{\infty}{\frac{f^{(k)}\left( z_{0} \right)}{k!}\left( {z - z_{0}} \right)^{k}}}\),
ahol \(z \in B_{R}\left( z_{0} \right)\),
\(B_{R}\left( z_{0} \right): = \left\{ {z:\left| {z - z_{0}} \right| < R} \right\}\)
az a maximális sugarú \(z_{0}\) középpontú kör, amely
\(B_{R}\left( z_{0} \right) \subset \Omega\).

\end{tetel}

\begin{pelda}

Példa:\\
Legyen \(f\left( z \right): = \frac{1}{1 - z}\), ekkor \(f\) holomorf az
\({\mathbb{C}}\backslash\left\{ 1 \right\}\) tartományon. Fejtsük
Taylor-sorba \(f\)-t a \(z_{0} = 0\) körül! Ekkor
\(\frac{1}{1 - z} = {\sum\limits_{k = 1}^{\infty}z^{k}}\). A sor
\(\left| z \right| < 1\) esetén konvergens, \(\left| z \right| \geq 1\)
esetén divergens, tehát csak akkor igaz az előbbi egyenlőség, ha
\(\left| z \right| < 1\).

\end{pelda}

\begin{bizonyitas}

Bizonyítás:\\
Legyen \(z \in B_{R}\left( z_{0} \right)\), ekkor \(r\)-t úgy
választjuk, hogy \(\left| {z - z_{0}} \right| < r < R\). Jelöljük:
\(K_{r}\left( z_{0} \right): = \left\{ {z:\left| {z - z_{0}} \right| = r} \right\}\).
Alkalmazzuk a Cauchy- féle integrálformulát
\(K_{r}\left( z_{0} \right)\)-ra és \(z\)-re:
\(f\left( z \right) = \frac{1}{2\pi i}{\int_{K_{r}{(z_{0})}}{\frac{f\left( \zeta \right)}{\zeta - z}d\zeta}}\).
A nevező:
\(\frac{1}{\zeta - z} = \frac{1}{\left( {\zeta - z_{0}} \right) - \left( {z - z_{0}} \right)} = \frac{1}{\zeta - z_{0}}\frac{1}{1 - \frac{z - z_{0}}{\zeta - z_{0}}}\)
(ez azért jó, mert
\(\left| \frac{z - z_{0}}{\zeta - z_{0}} \right| < 1\), ugyanis
\(\left| {z - z_{0}} \right| < r = \left| {\zeta - z_{0}} \right|\)).
Tehát
\(\frac{1}{\zeta - z} = \frac{1}{\zeta - z_{0}}{\sum\limits_{k = 0}^{\infty}\left( \frac{z - z_{0}}{\zeta - z_{0}} \right)^{k}} = {\sum\limits_{k = 0}^{\infty}\frac{\left( {z - z_{0}} \right)^{k}}{\left( {\zeta - z_{0}} \right)^{k + 1}}}\).
A sor egyenletesen konvergens, ha
\(\zeta \in K_{r}\left( z_{0} \right)\) a Weierstrass kritérium szerint.
Így \[\begin{aligned}
  f\left( z \right) &  = \frac{1}{{2\pi i}}\int_{{K_r}({z_0})} {f\left( \zeta  \right)\sum\limits_{k = 0}^\infty  {\frac{{{{\left( {z - {z_0}} \right)}^k}}}{{{{\left( {\zeta  - {z_0}} \right)}^{k + 1}}}}} d\zeta }  \\ 
   &  = \sum\limits_{k = 0}^\infty  {{{\left( {z - {z_0}} \right)}^k}\underbrace {\frac{1}{{2\pi i}}\int_{{K_r}({z_0})} {\frac{{f\left( \zeta  \right)}}{{{{\left( {\zeta  - {z_0}} \right)}^{k + 1}}}}d\zeta } }_{: = {c_k}}}  \\ 
   &  = \sum\limits_{k = 0}^\infty  {{c_k}{{\left( {z - {z_0}} \right)}^k}}.  \\ 
\end{aligned} \] Ugyanis tudjuk, hogy
\(\left. f^{(k)}\left( z_{0} \right) = \frac{k!}{2\pi i}{\int_{K_{r}{(z_{0})}}{\frac{f\left( \zeta \right)}{\left( {\zeta - z_{0}} \right)^{k + 1}}d\zeta}}\Rightarrow c_{k} = \frac{f^{(k)}\left( z_{0} \right)}{k!} \right.\).

\end{bizonyitas}

Következmény: tfh \(f\), \(g\) holomorf függvények
\(\Omega \subset {\mathbb{C}}\) tartományon és
\(\exists z_{j},j \in {\mathbb{Z}}^{+},z_{j} \in \Omega\text{\textbackslash}\left\{ z_{0} \right\}:f\left( z_{j} \right) = g\left( z_{j} \right)\),
ahol \(\lim z_{j} = z_{0} \in \Omega\). Ekkor
\(f\left( z \right) = g\left( z \right),\forall z \in \Omega\).

\begin{bizonyitas}

Bizonyítás:

\begin{enumerate}
\def\labelenumi{\arabic{enumi}.}
\tightlist
\item
  Először belátjuk, hogy
  \(f\left( z \right) = g\left( z \right),z \in B_{R}\left( z_{0} \right) \subset \Omega\).
  Fejtsük Taylor-sorba mindkét függvény, \(f\)-t és \(g\)-t is \(z_{0}\)
  körül.
  \(f\left( z \right) = {\sum\limits_{k = 1}^{\infty}{c_{k}\left( {z - z_{0}} \right)^{k}}},g\left( z \right) = {\sum\limits_{k = 0}^{\infty}{d_{k}\left( {z - z_{0}} \right)^{k}}}\),
  \(f\left( z_{j} \right) = g\left( z_{j} \right)\), így
  \[\begin{aligned}
    f\left( {{z_j}} \right) &  = {c_0} + \underbrace {{c_1}\left( {{z_j} - {z_0}} \right) + {c_2}{{\left( {{z_j} - {z_0}} \right)}^2} + ...}_{ \to 0{\text{\;ha\;}}{z_j} \to {{\text{z}}_0}{\text{,\;mivel\;a\;}}f \in C\left( {{z_0}} \right)} \\ 
     &  = {d_0} + \underbrace {{d_1}\left( {{z_j} - {z_0}} \right) + {d_2}{{\left( {{z_j} - {z_0}} \right)}^2} + ...}_{ \to 0} \\ 
     &  = g\left( {{z_j}} \right) \Rightarrow {c_0} = {d_0}, \\ 
  \end{aligned} \] így
  \(c_{1}\left( {z_{j} - z_{0}} \right) + c_{2}\left( {z_{j} - z_{0}} \right)^{2} + .., = d_{1}\left( {z_{j} - z_{0}} \right) + d_{2}\left( {z_{j} - z_{0}} \right)^{2} + ...\)
  Mivel \(z_{j} \neq z_{0}\), ezért oszthatunk \(z_{j} - z_{0}\)-lal:
  \({c_1} + \underbrace {{c_2}\left( {{z_j} - {z_0}} \right) + ...}_{{\text{folytonos}}{\text{,\;}} \to {\text{0}}} = {d_1} + \underbrace {{d_2}\left( {{z_j} - {z_0}} \right) + ...}_{{\text{folytonos}}{\text{,\;}} \to 0} \Rightarrow {c_1} = {d_1}\),
  és így tovább, tehát
  \(\left. c_{k} = d_{k},\forall k \in {\mathbb{N}}\Rightarrow f\left( z \right) = g\left( z \right),\forall z \in B_{R}\left( z_{0} \right) \right.\).
\item
  Legyen \(z \in \Omega\) tetszőleges! Kössük össze \(z_{0}\)-t és
  \(z\)-t egy véges sok egyenes szakaszból álló \(\Gamma\) törött
  vonallal.
  \(\inf\left\{ {\rho\left( {\zeta,\partial\Omega} \right):\zeta \in \Gamma} \right\}: = \beta > 0\).
  Most \(z\)-t és \(z_{0}\)-t összekötjük egy \(\beta\) sugarú körökből
  álló körlánccal, ezeken \(f\left( z \right) = g\left( z \right)\), az
  egymás utáni körökön. Spec eset: \(g\left( z \right) \equiv 0\),
  \(\left. f\left( z_{j} \right) = 0,z_{j} \in \Omega\backslash\left\{ z_{0} \right\},j \in {\mathbb{Z}}^{+},\lim z_{j} = z_{0} \in \Omega\Rightarrow f\left( z \right) = 0,\forall z \right.\).
\end{enumerate}

\end{bizonyitas}

\begin{definicio}

Definíció:\\
Legyen \(f\) holomorf függvény \(\Omega\) tartományon,
\(z_{0} \in \Omega\) Azt mondjuk, hogy \(z_{0}\) az \(f\) függvénynek
\(n\)-szeres gyöke, ha
\(f\left( z_{0} \right) = f^{\prime}\left( z_{0} \right) = ... = f^{({n - 1})}\left( z_{0} \right) = 0,f^{(n)}\left( z_{0} \right) \neq 0\).

\end{definicio}

\begin{allitas}

Állítás:\\
\(z_{0}\) az \(f\)-nek \(n\)-szeres gyöke
\(\left. \Leftrightarrow f\left( z \right) = \left( {z - z_{0}} \right)^{n}g\left( z \right) \right.\),
ahol \(g\) holomorf \(z_{0}\) egy környezetében és
\(g\left( z_{0} \right) \neq 0\).

\end{allitas}

\begin{bizonyitas}

Bizonyítás:

\begin{itemize}
\tightlist
\item
  \(\Rightarrow\) irányban: tfh
  \(f\left( z_{0} \right) = f^{\prime}\left( z_{0} \right) = ... = f^{({n - 1})}\left( z_{0} \right) = 0,f^{(n)}\left( z_{0} \right) \neq 0\).
  Fejtsük Taylor-sorba \(z_{0}\) körül: \[\begin{aligned}
    f\left( z \right) &  = \mathop \sum \limits_{k = 1}^\infty  \frac{{{f^{\left( k \right)}}\left( {{z_0}} \right)}}{{k!}}{\left( {z - {z_0}} \right)^k} \\ 
     &  = \mathop \sum \limits_{k = n}^\infty  \frac{{{f^{\left( k \right)}}\left( {{z_0}} \right)}}{{k!}}{\left( {z - {z_0}} \right)^k} \\ 
     &  = {\left( {z - {z_0}} \right)^n}\underbrace {\mathop \sum \limits_{k = n}^\infty  \frac{{{f^{\left( k \right)}}\left( {{z_0}} \right)}}{{k!}}{{\left( {z - {z_0}} \right)}^{k - n}}}_{g\left( z \right)}. \\ 
  \end{aligned} \]\(g\) holomorf \(z_{0}\) környezetében:
  \(g\left( z_{0} \right) = \frac{f^{(n)}\left( z_{0} \right)}{n!} \neq 0\).
\item
  \(\Leftarrow\) irányban:
  \(f\left( z \right) = \left( {z - z_{0}} \right)^{n}g\left( z \right)\),
  \(g\) holomorf és \(g\left( z_{0} \right) \neq 0\). \(g\)-t sorba
  fejtjük \(z_{0}\) körül: \[\begin{gathered}
    g(z) = \sum\limits_{l = 0}^\infty  {{c_l}{{(z - {z_0})}^l}} ,{c_0} \ne 0 \\ 
     \Downarrow  \\ 
    f(z) = {(z - {z_0})^n}g(z) = \sum\limits_{l = 0}^\infty  {{c_l}{{(z - {z_0})}^{l + n}}}  = {c_0}{(z - {z_0})^n} + {c_1}{(z - {z_0})^{n + 1}} + ... \\ 
  \end{gathered} \] Leolvashatjuk, hogy \(f\) Taylor sorfejtésénél az
  első \(n\) db együttható 0.
  \(\left. \Rightarrow f\left( z_{0} \right) = 0,f^{\prime}\left( z_{0} \right) = 0,...,f^{({n - 1})}\left( z_{0} \right) = 0 \right.\),
  \(f^{(n)}\left( z_{0} \right) \neq 0\), mivel \(c_{0} \neq 0\).
\end{itemize}

\end{bizonyitas}

\hypertarget{egesz-fuggvenyek-liouville-tetele}{%
\subsubsection{Egész függvények, Liouville
tétele}\label{egesz-fuggvenyek-liouville-tetele}}

\begin{definicio}

Definíció:\\
Ha \(f\) függvény holmorf \(\mathbb{C}\)-n, akkor \(f\)-t egész
függvénynek nevezzük.

\end{definicio}

\begin{tetel}

Liouville tétele:\\
Ha \(f\) egész függvény korlátos \(\left. \Rightarrow f \right.\)
állandó.

\end{tetel}

\begin{bizonyitas}

Bizonyítás:\\
Tudjuk, hogy \(f\) holomorf \(\mathbb{C}\)-n. Fejtsük Taylor-sorba
\(z_{0} = 0\) körül! Legyen
\(K_{r}: = \left\{ {z \in {\mathbb{C}}:\left| z \right| = r} \right\},M_{r}: = \sup\left\{ {\left| {f\left( \zeta \right)} \right|,\left| \zeta \right| = r} \right\}\),
ekkor \(\forall z \in {\mathbb{C}}\)-re
\(f\left( z \right) = {\sum\limits_{k = 0}^{\infty}{c_{k}z^{k}}}\),
\(\left. c_{k} = \frac{f^{(k)}\left( 0 \right)}{k!} = \frac{1}{2\pi i}{\int_{K_{r}}{\frac{f\left( \zeta \right)}{\zeta^{k + 1}}d\zeta}}\Rightarrow\left| c_{k} \right| = \frac{1}{2\pi}\left| {\int_{K_{r}}{\frac{f\left( \zeta \right)}{z^{k + 1}}d\zeta}} \right| \leq \frac{1}{2\pi}2\pi r\frac{M_{r}}{r^{k + 1}} = \frac{M_{r}}{r^{k}} \right.\).
Ha speciel \(f\) korlátos, \(M_{r} \leq M\) (\(r\)-től függetlenül), így
\(\left. \left| c_{k} \right| \leq \frac{M}{r^{k}},\forall r\Rightarrow k \geq 1 \right.\)
esetén \(c_{k} = 0\), \(f\left( z \right) = c_{0}\).

\end{bizonyitas}

\begin{ajanlo}

\begin{ajanlofig}

\href{https://xkcd.com}{\includegraphics[width=5.20833in,height=2.82292in]{wikipedian_protester.png}}

\end{ajanlofig}

Text

\end{ajanlo}

\hypertarget{az-algebra-alaptetele}{%
\subsection{Az algebra alaptétele}\label{az-algebra-alaptetele}}

\begin{tetel}

Tétel:\\
Legyen \(P\) egy legalább elsőfokú, komplex együtthatós polinom! Ekkor
mindig \(\exists z_{0} \in {\mathbb{C}}:P\left( z_{0} \right) = 0\).

\end{tetel}

\begin{bizonyitas}

Bizonyítás:\\
Indirekt feltesszük, hogy
\(P\left( z \right) \neq 0,\forall z \in {\mathbb{C}}\). Ekkor
\(\frac{1}{P}\) holomorf függvény. Belátjuk, hogy korlátos is.
\(P\left( z \right) = a_{n}z^{n} + a_{n - 1}z^{n - 1} + ... + a_{1}z + a_{0},n \geq 1,a_{n} \neq 0\).
\(\left| {\frac{1}{{P\left( z \right)}}} \right| = \frac{1}{{\left| {P\left( z \right)} \right|}} = \frac{1}{{{z^n}\underbrace {\left| {{a_n} + {a_{n - 1}}{z^{ - 1}} + ... + {a_1}{z^{ - n + 1}} + {a_0}{z^{ - n}}} \right|}_{ \to \left| {{a_n}} \right|{\text{\;ha\;}}z \to \infty }}},\)
így \(\exists r\), hogy
\(\left| {a_{n} + \frac{a_{n - 1}}{z} + ... + \frac{a_{1}}{z^{n - 1}} + \frac{a_{0}}{z^{n}}} \right| \geq \frac{\left| a_{n} \right|}{2}\)
ha \(\left| z \right| \geq r\), tehát
\(\left. \exists\rho > 0:\left| z \right| > \rho\Rightarrow\frac{1}{\left| {P\left( z \right)} \right|} \leq 1 \right.\).
\(\left| z \right| \leq \rho\) esetén
\(\left| \frac{1}{P\left( z \right)} \right|\) korlátos a
Weierstrass-tétel miatt, hiszen \(\left| \frac{1}{P} \right|\)
folytonos, a \(\rho\) sugarú kör sorozatkompakt. \(\frac{1}{P}\)
korlátos, másrészt holomorf \(\Rightarrow\) (Liouvielle-tétel)
\(\left. \frac{1}{P} = {all.}\Rightarrow P = {all.} \right.\), de ez meg
ellentmond annak, hogy \(n \geq 1\).

\end{bizonyitas}

\hypertarget{az-exponencialis-a-szinusz-es-koszinusz-fuggvenyek-komplex-valtozokon}{%
\subsubsection{Az exponenciális, a szinusz és koszinusz függvények
komplex
változókon}\label{az-exponencialis-a-szinusz-es-koszinusz-fuggvenyek-komplex-valtozokon}}

(Kalkuluson szó volt az exponenciális, a szinusz és koszinusz
hatványsoráról a valósban. Ezeket kaptuk:
\(e^{x} = {\sum\limits_{k = 0}^{\infty}{\frac{x^{k}}{k!},\forall x \in {\mathbb{R}}}}\),
a konvergencia sugár végtelen.
\(\sin x = x - \frac{x^{3}}{3!} + \frac{x^{5}}{5!} - ... + , -\) illetve
\(\cos x = 1 - \frac{x^{2}}{2!} + \frac{x^{4}}{4!} - ... + , -\)
Akkoriban lehetett volna így is definiálni a függvényeket, és az akkori
definíciókat meg igazolni. Ha így tettük volna, a komplexes
általánosítás könnyebb volna.)

Legyen definíció szerint
\(e^{z}: = {\sum\limits_{k = 0}^{\infty}{\frac{z^{k}}{k!},\forall z \in {\mathbb{C}}}}\),
a konvergencia sugár ugyanaz, mint valósban, valamint
\(\sin z: = z - \frac{z^{3}}{3!} + \frac{z^{5}}{5!} - ... + , -\)
továbbá
\(\cos z: = 1 - \frac{z^{2}}{2!} + \frac{z^{4}}{4!} - ... + , -\). A
hatványsoros definíciós segítségével belátható, hogy
\(e^{z_{1} + z_{2}} = e^{z_{1}}e^{z_{2}} = {\sum\limits_{k = 0}^{\infty}\frac{z_{1}^{k}}{k!}} \cdot {\sum\limits_{l = 0}^{\infty}\frac{z_{2}^{l}}{l!}}\).
A sor abszolút konvergens, így szabadon cserélgethetők a
szorzótényezők:.
\(e^{z_{1} + z_{2}} = {\sum\limits_{n = 0}^{\infty}\frac{\left( {z_{1} + z_{2}} \right)^{n}}{n!}} = {\sum\limits_{n = 0}^{\infty}{\frac{1}{n!}{\sum\limits_{k = 0}^{n}\left\lbrack {\begin{pmatrix} n \\ k \\ \end{pmatrix}z_{1}^{k}z_{2}^{n - k}} \right\rbrack}}} = {\sum\limits_{n = 0}^{\infty}\left\lbrack {\sum\limits_{k = 0}^{n}{\frac{z_{1}^{k}}{k!}\frac{z_{2}^{n - k}}{\left( {n - k} \right)!}}} \right\rbrack}\),
ahol felhasználtuk a binomiális tételt.

Következmény: legyen
\(z \in {\mathbb{C}},\left( {x,y} \right) \in {\mathbb{R}}^{2},z: = x + iy\).
Ekkor \(e^{z} = e^{x + iy} = e^{x}e^{iy}\),
\(e^{iy} = 1 + \frac{iy}{1!} + \frac{\left( {iy} \right)^{2}}{2!} + \frac{\left( {iy} \right)^{3}}{3!} + ... = \left( {1 - \frac{y^{2}}{2!} + \frac{y^{4}}{4!} - ... + , -} \right) + i\left( {\frac{y}{1!} - \frac{y^{3}}{3!} + \frac{y^{5}}{5!} - ... + , -} \right) = \cos y + i\sin y\),
így \(e^{z} = e^{x}\left( {\cos y + i\sin y} \right)\), valamint
\(\left| e^{z} \right| = e^{x} = e^{\Re{(z)}}\),
\(\arg e^{z} = y = \Im\left( z \right)\),
\(e^{iz} = \left( {1 - \frac{z^{2}}{2!} + \frac{z^{4}}{4!} - ... + , -} \right) + i\left( {\frac{z}{1!} - \frac{z^{3}}{3!} + \frac{z^{5}}{5!} - ... + , -} \right)\),
\(e^{- iz} = \left( {1 - \frac{z^{2}}{2!} + \frac{z^{4}}{4!} - ... + , -} \right) + i\left( {- \frac{z}{1!} + \frac{z^{3}}{3!} - \frac{z^{5}}{5!} + ... - , +} \right)\),
így
\(\frac{e^{iz} + e^{- iz}}{2} = 1 - \frac{z^{2}}{2!} + \frac{z^{4}}{4!} - ... + , - = \cos z\),
valamint
\(\frac{e^{iz} - e^{- iz}}{2i} = \frac{z}{1!} - \frac{z^{3}}{3!} + \frac{z^{5}}{5!} - ... + , - = \sin z\).
Ezek igazak \(\forall z \in {\mathbb{C}}\). Komplexben is igaz, hogy
\(\sin^{2}z + \cos^{2}z = 1\), merthogy
\(\sin^{2}z + \cos^{2}z = \left\lbrack \frac{e^{iz} + e^{- iz}}{2} \right\rbrack^{2} + \left\lbrack \frac{e^{iz} - e^{- iz}}{2i} \right\rbrack^{2} = \frac{e^{2iz} + e^{- 2iz} + 2}{4} + \frac{e^{2iz} + e^{- 2iz} - 2}{- 4} = \frac{2 + 2}{4} = 1\),
valamint az összes többi formula, ami valósban is igaz volt
(periodicitás, paritás). \(e^{z}\) periodikus \(2\pi i\) szerint,
ugyanis
\(e^{z + 2\pi i} = e^{z}e^{2\pi i} = e^{z}\left( {\cos\left( {2\pi} \right) + i\sin\left( {2\pi} \right)} \right) = e^{z}\).
Ami másképp van: valósban
\(\left| {\sin x} \right| \leq 1 \geq \left| {\cos x} \right|\), mert
\(\sin^{2}x + \cos^{2}x = 1\) és \(\sin^{2}x \geq 0\),
\(\cos^{2}x \geq 0\), de komplexben ez így nem igaz. Megemlítendő, hogy
\(e^{z} \neq 0,\forall z \in {\mathbb{C}}\), mert
\({e^z} = \underbrace {{e^x}}_{ > 0}\underbrace {\left( {\cos \left( y \right) + i\sin \left( y \right)} \right)}_{{\text{abszolút értéke 1}}}\).

\hypertarget{izolalt-szingularis-pontok-laurent-sorfejtes}{%
\subsection{Izolált szinguláris pontok,
Laurent-sorfejtés}\label{izolalt-szingularis-pontok-laurent-sorfejtes}}

\begin{definicio}

Definíció:\\
Ha \(\Omega \subset {\mathbb{C}}\) tartomány,
\(\left. f:\Omega\rightarrow{\mathbb{C}} \right.\) holomorf egy
\(z_{0} \in \Omega\) pont kivételével, akkor \(z_{0} - t\) izolált
szinguláris pontnak nevezzük. Cél: \(f\)-et szeretnénk valamilyen sorba
fejteni \(z_{0}\) körül.

\end{definicio}

Tfh \(z_{0}\) egy izolált szinguláris pont, \(f\) holomorf a
\(\Omega: = \left\{ {z \in {\mathbb{C}}:0 < \left| {z_{0} - z} \right| < R} \right\}\)
tartományon, ahol \(0 < R \leq \infty\). Válasszuk \(r_{1},r_{2}\)
számokat: \(0 < r_{1} < r_{2} < R\). \(z \in \Omega\) esetén
\(r_{1},r_{2}\) megválasztható úgy, hogy
\(0 < r_{1} < \left| {z - z_{0}} \right| < r_{2} < R\). Nem nehéz
belátni, hogy a Cauchy-féle integrálformula szerint
\(f\left( z \right) = \frac{1}{2\pi i}{\int_{S_{r_{2}}}{\frac{f\left( \zeta \right)}{\zeta - z}d\zeta}} - \frac{1}{2\pi i}{\int_{S_{r_{1}}}{\frac{f\left( \zeta \right)}{\zeta - z}d\zeta}}\).
\(\zeta \in S_{r_{2}}\) esetén
\(\frac{1}{\zeta - z} = \frac{1}{\left( {\zeta - z_{0}} \right) - \left( {z - z_{0}} \right)} = \frac{1}{\zeta - z_{0}}\frac{1}{1 - \frac{z - z_{0}}{\zeta - z_{0}}} = \frac{1}{\zeta - z_{0}}{\sum\limits_{n = 0}^{\infty}\left( \frac{z - z_{0}}{\zeta - z_{0}} \right)^{n}} = {\sum\limits_{n = 0}^{\infty}\frac{\left( {z - z_{0}} \right)^{n}}{\left( {\zeta - z_{0}} \right)^{n + 1}}}\),
a sor egyenletesen konvergens \(\zeta \in S_{r_{2}}\) esetén. Ha
\(\zeta \in S_{r_{1}}\),
\(\frac{1}{\zeta - z} = \frac{1}{\left( {\zeta - z_{0}} \right) - \left( {z - z_{0}} \right)} = - \frac{1}{z - z_{0}}\frac{1}{1 - \frac{\zeta - z_{0}}{z - z_{0}}} = - \frac{1}{z - z_{0}}{\sum\limits_{m = 0}^{\infty}{\left( \frac{\zeta - z_{0}}{z - z_{0}} \right)^{m} = - {\sum\limits_{m = 0}^{\infty}\frac{\left( {\zeta - z_{0}} \right)^{m}}{\left( {z - z_{0}} \right)^{m + 1}}}}}\).
Vezessük be az
\(\left. m + 1 = : - n\Leftrightarrow m = : - n - 1 \right.\) új
indexváltozót, így
\(\frac{1}{\zeta - z} = - {\sum\limits_{n = - 1}^{- \infty}\frac{\left( {z - z_{0}} \right)^{n}}{\left( {\zeta - z_{0}} \right)^{n + 1}}}\).
Ekkor \[\begin{aligned}
  f\left( z \right) &  = \frac{1}{{2\pi i}}\int_{{S_{{r_2}}}} {f\left( \zeta  \right)\mathop \sum \limits_{n = 0}^\infty  \frac{{{{\left( {z - {z_0}} \right)}^n}}}{{{{\left( {\zeta  - {z_0}} \right)}^{n + 1}}}}d\zeta }  + \frac{1}{{2\pi i}}\int_{{S_{{r_1}}}} {f\left( \zeta  \right)\mathop \sum \limits_{n =  - 1}^{ - \infty } \frac{{{{\left( {z - {z_0}} \right)}^n}}}{{{{\left( {\zeta  - {z_0}} \right)}^{n + 1}}}}d\zeta }  \\ 
   &  = \mathop \sum \limits_{n = 0}^\infty  {\left( {z - {z_0}} \right)^n}\frac{1}{{2\pi i}}\int_{{S_{{r_2}}}} {\frac{{f\left( \zeta  \right)}}{{{{\left( {\zeta  - {z_0}} \right)}^{n + 1}}}}d\zeta }  + \mathop \sum \limits_{n =  - 1}^{ - \infty } {\left( {z - {z_0}} \right)^n}\frac{1}{{2\pi i}}\int_{{S_{{r_1}}}} {\frac{{f\left( \zeta  \right)}}{{{{\left( {\zeta  - {z_0}} \right)}^{n + 1}}}}d\zeta }  \\ 
   &  = \mathop \sum \limits_{n =  - \infty }^\infty  {\left( {z - {z_0}} \right)^n}\frac{1}{{2\pi i}}\int_{{S_r}} {\frac{{f\left( \zeta  \right)}}{{{{\left( {\zeta  - {z_0}} \right)}^{n + 1}}}}d\zeta }  \\ 
   &  = :\mathop \sum \limits_{n =  - \infty }^\infty  {c_n}{\left( {z - {z_0}} \right)^n} ,\\ 
\end{aligned} \] ahol
\(c_{n}: = \frac{1}{2\pi i}{\int_{S_{r}}{\frac{f\left( \zeta \right)}{\left( {\zeta - z_{0}} \right)^{n + 1}}d\zeta}},0 < r < R\).
(A Cauchy-alaptétel következménye miatt vehetünk \(S_{r_{1}}\),
\(S_{r_{2}}\) helyett \(S_{r}\)-et.)

\begin{tetel}

Tétel:\\
Tfh \(f\) holomorf az
\(\Omega: = \left\{ {z \in {\mathbb{C}}:0 < \left| {z - z_{0}} \right| < R} \right\}\)
tartományon, \(0 < R \leq \infty\)). Ekkor \(\forall z \in \Omega\)
esetén
\(f\left( z \right) = {\sum\limits_{n = - \infty}^{\infty}{c_{n}\left( {z - z_{0}} \right)^{n}}}\),
ahol
\(c_{n} = \frac{1}{2\pi i}{\int_{S_{r}}{\frac{f\left( \zeta \right)}{\left( {\zeta - z_{0}} \right)^{n + 1}}d\zeta}}\).
Ezt nevezzük \(f\) Laurent sorfejtésének.

\end{tetel}

\begin{megjegyzes}

Megjegyzés:\\
A Laurent sorfejtés egyértelmű. Ugyanis nem nehéz belátni, hogy ha
\[\left. f\left( z \right) = {\sum\limits_{n = - \infty}^{\infty}{c_{n}\left( {z - z_{0}} \right)^{n}}} = {\sum\limits_{n = - \infty}^{\infty}{d_{n}\left( {z - z_{0}} \right)^{n}}}\Rightarrow d_{n} = c_{n}. \right.\]A
Laurent sorfejtés egyenletesen konvergens
\(\text{belseje}\left( \Omega \right)\)-n, azaz minden \(\Omega\)-ban
fekvő sorozatkompakt halmazon.

\end{megjegyzes}

\hypertarget{az-izolalt-szingularis-pontok-osztalyozasa}{%
\subsubsection{Az izolált szinguláris pontok
osztályozása}\label{az-izolalt-szingularis-pontok-osztalyozasa}}

\[f\left( z \right) = \mathop \sum \limits_{n =  - \infty }^\infty  {c_n}{\left( {z - {z_0}} \right)^n} = \underbrace {\sum\limits_{n =  - \infty }^{ - 1} {{c_n}{{\left( {z - {z_0}} \right)}^n}} }_{{\text{főrész}}} + \underbrace {\sum\limits_{n = 0}^\infty  {{c_n}{{\left( {z - {z_0}} \right)}^n}} }_{{\text{reguláris rész}}}\]

\begin{enumerate}
\def\labelenumi{\arabic{enumi}.}
\tightlist
\item
  Ha \(c_{n} = 0,\forall n \in {\mathbb{Z}}^{-}\) esetén, akkor
  \(z_{0}\) megszüntethető szingularitás,
  \(f\left( z_{0} \right): = c_{0} < \infty\).
\item
  Ha véges sok negatív indexű együttható nem 0, akkor \(z_{0}\)-t
  pólusnak nevezzük (az ilyen együtthatók száma a pólus rendje).
\item
  Ha végtelen sok negatív indexre az együttható nem 0, akkor \(z_{0}\)-t
  lényeges szingularitásnak nevezzük.
\end{enumerate}

\begin{definicio}

Definíció:\\
A Laurent sorfejtésben a \(c_{- 1}\) együtthatót a függvény
\(z_{0}\)-beli reziduumának nevezzük.
\(\text{Rez}_{z_{0}}f: = c_{- 1} = \frac{1}{2\pi i}{\int_{S_{r}}{f\left( \zeta \right)d\zeta}}\)

\end{definicio}

\begin{megjegyzes}

Megjegyzés:\\
\({\int_{S_{r}}{f\left( \zeta \right)d\zeta}} = 2\pi i \cdot \text{Rez}_{z_{0}}f\).

\end{megjegyzes}

\begin{tetel}

Reziduum-tétel:\\
Tfh \(f\) holomorf az \(\Omega\) tartományon a \(z_{1},z_{2},...,z_{k}\)
izolált szinguláris pontok kivételével. Ekkor véve olyan egyszerű zárt
szakaszonként folytonosan differenciálható \(\Gamma\) görbét, amely
\(\Omega\)-ban van a belsejével együtt,
\[\left. z_{1},z_{2},...,z_{k} \in \text{belseje}\left( \Gamma \right)\Rightarrow{\int_{\Gamma}{f\left( \zeta \right)d\zeta}} = \frac{1}{2\pi i}\sum\limits_{j = 1}^{k}\text{Rez}_{z_{j}}f. \right.\]

\end{tetel}

\begin{bizonyitas}

Bizonyítás:\\
\({\int_{\Gamma}{f\left( \zeta \right)d\zeta}} = \sum\limits_{j = 1}^{k}{\int_{S_{j}}{f\left( \zeta \right)d\zeta}} = \sum\limits_{j = 1}^{k}2\pi i\text{Rez}_{z_{j}}f\).

\end{bizonyitas}

\hypertarget{reziduum-kiszamitasa-polus-eseten}{%
\subsubsection{Reziduum kiszámítása pólus
esetén}\label{reziduum-kiszamitasa-polus-eseten}}

Tfh \(f\) függvénynek \(z_{0}\)-ban \(m\)-edrendű pólusa van:
\[\begin{aligned}
  f\left( z \right) =  & \frac{{{c_{ - m}}}}{{{{\left( {z - {z_0}} \right)}^m}}} + \frac{{{c_{ - m + 1}}}}{{{{\left( {z - {z_0}} \right)}^{m - 1}}}} + ... + \frac{{{c_{ - 1}}}}{{z - {z_0}}} + {c_0} + {c_1}\left( {z - {z_0}} \right) + ... \\ 
   \Downarrow  &  \\ 
  f\left( z \right){\left( {z - {z_0}} \right)^m} =  & {c_{ - m}} + {c_{ - m + 1}}\left( {z - {z_0}} \right) + ... \\ 
   &  + {c_{ - 1}}{\left( {z - {z_0}} \right)^{m - 1}} + {c_0}{\left( {z - {z_0}} \right)^m} + {c_1}{\left( {z - {z_0}} \right)^{m + 1}} + ... \\ 
\end{aligned} \] Ez már hatványsor. \[\begin{gathered}
  {\left[ {\frac{{{d^{m - 1}}}}{{d{z^{m - 1}}}}\left( {f\left( z \right){{\left( {z - {z_0}} \right)}^m}} \right)} \right]_{z = {z_0}}} = {c_{ - 1}}\left( {m - 1} \right)! \\ 
   \Downarrow  \\ 
  {\text{Rez}}
  {{\text{Rez}}_{{z_0}}}f = {c_{ - 1}} = \frac{1}{{\left( {m - 1} \right)!}}{\left[ {\frac{{{d^{m - 1}}}}{{d{z^{m - 1}}}}\left( {f\left( z \right){{\left( {z - {z_0}} \right)}^m}} \right)} \right]_{z = {z_0}}} .\\ 
\end{gathered} \]

\begin{allitas}

Állítás:\\
Az \(f\) függvénynek \(z_{0}\)-ban \(m\)-edrendű pólusa van
\(\left. \Leftrightarrow g\left( z \right): = f\left( z \right)\left( {z - z_{0}} \right)^{m} \right.\)
holomorf és \(g\left( z_{0} \right) \neq 0\). Ugyanis
\(f\left( z \right){\left( {z - {z_0}} \right)^m} = \underbrace {{c_{ - m}}}_{ \ne 0} + {c_{ - m + 1}}\left( {z - {z_0}} \right) + ...\)

\end{allitas}

\begin{allitas}

Állítás:\\
Ha \(h\) holomorf függvény \(z_{0}\)-ban és \(h\)-nak \(z_{0}\)-ban
\(m\)-szeres gyöke van, akkor az \(f = \frac{1}{h}\) függvénynek a
\(z_{0}\)-ban \(m\)-edrendű pólusa van.

\end{allitas}

\begin{bizonyitas}

Bizonyítás:\\
\(h\left( z \right) = \left( {z - z_{0}} \right)^{m}h_{1}\left( z \right)\),
\(h_{1}\left( z_{0} \right) \neq 0\), \(h_{1}\) holomorf.
\(f\left( z \right) = \frac{1}{h\left( z \right)} = \frac{1}{\left( {z - z_{0}} \right)^{m}h_{1}\left( z \right)}\),
\(g\left( z \right): = f\left( z \right)\left( {z - z_{0}} \right)^{m} = \frac{1}{h_{1}\left( z \right)}\)
holomorf,
\(g\left( z_{0} \right) = \frac{1}{h_{1}\left( z_{0} \right)} \neq 0\).

\end{bizonyitas}

A reziduum kiszámításának két egyszerű esete:

\begin{enumerate}
\def\labelenumi{\arabic{enumi}.}
\tightlist
\item
  Tfh \(h\)-nak \(z_{0}\)-ban egyszeres gyöke van, vagyis
  \(h\left( z \right) = \left( {z - z_{0}} \right)h_{1}\left( z \right),h_{1}\left( z_{0} \right) \neq 0\).
  \(f = \frac{1}{h}\)-nak \(z_{0}\)-ban elsőrendű pólusa van. \(m = 1\)
  esetre \[\begin{aligned}
    {{\text{Rez}}_{{z_0}}}\left( f \right) &  = {\left[ {f\left( z \right)\left( {z - {z_0}} \right)} \right]_{z = {z_0}}} = {\left[ {\frac{{z - {z_0}}}{{h\left( z \right)}}} \right]_{z = {z_0}}} \\ 
     &  = \mathop {\lim }\limits_{z \to {z_0}} \frac{{z - {z_0}}}{{h\left( z \right)}} = \mathop {\lim }\limits_{z \to {z_0}} \frac{{z - {z_0}}}{{h\left( z \right) - \underbrace {h\left( {{z_0}} \right)}_0}} = \mathop {\lim }\limits_{z \to {z_0}} \frac{1}{{\frac{{h\left( z \right) - h\left( {{z_0}} \right)}}{{z - {z_0}}}}} = \frac{1}{{\underbrace {h'\left( {{z_0}} \right)}_{ \ne 0}}}. \\ 
  \end{aligned} \]
\item
  Tfh \(f = \phi\psi\), ahol \(\phi\) holomorf \(z_{0}\)-ban, viszont
  \(\psi\) -nek elsőrendű pólusa van itt. \(\text{Rez}_{z_{0}}f = ?\)
  \(\phi\left( z \right) = c_{0} + c_{1}\left( {z - z_{0}} \right) + c_{2}\left( {z - z_{0}} \right)^{2} + ...\),
  \(\psi = \frac{d_{- 1}}{z - z_{0}} + d_{0} + d_{1}\left( {z - z_{0}} \right) + d_{2}\left( {z - z_{0}} \right)^{2} + ...\),
  így
  \(f(z) = \phi(z) \cdot \psi(z) = \frac{c_{0}d_{- 1}}{z - z_{0}} + (c_{0}d_{0} + c_{1}d_{- 1}) + (c_{2}d_{- 1} + c_{1}d_{0} + c_{0}d_{1})(z - z_{0}) + ...\),
  \(\text{Rez}_{z_{0}}f = c_{0}d_{- 1} = \phi(z_{0}) \cdot \text{Rez}_{z_{0}}\psi\).
\end{enumerate}

Ábra: ide kell betenni a cauchy\_gamma\_r\_s\_r.eps filet.

\begin{pelda}

Alkalmazás:\\
A reziduum tétel alkalmazása a (valós) improprius integrálok
kiszámítására.
\({\int\limits_{- \infty}^{\infty}{\frac{\cos x}{1 + x^{2}}dx}} = {\int\limits_{- \infty}^{\infty}{\frac{\cos x + i\sin x}{1 + x^{2}}dx}} = {\int\limits_{- \infty}^{\infty}{\frac{e^{ix}}{1 + x^{2}}dx}}\).
\(\left. z\mapsto\frac{e^{iz}}{1 + z^{2}} \right.\) izolált
szingularitás \(z = \pm i\) -ben, máshol holomorf.

\end{pelda}

\begin{ajanlo}

\begin{ajanlofig}

\href{https://xkcd.com}{\includegraphics[width=5.20833in,height=2.82292in]{wikipedian_protester.png}}

\end{ajanlofig}

Text

\end{ajanlo}

\({\int\limits_{- \infty}^{\infty}{\frac{e^{ix}}{1 + x^{2}}dx}} = \lim\limits_{R\rightarrow\infty}{\int\limits_{- R}^{R}{\frac{e^{ix}}{1 + x^{2}}dx}}\).
Legyen \(\Gamma_{R}\) az \(S_{R}\) -rel jelölt félkörvonal és a
\(\left\lbrack {- R,R} \right\rbrack\) intervallum egymásutánja, ekkor
\[\begin{gathered}
  \int_{{\Gamma _R}} {\frac{{{e^{iz}}}}{{1 + {z^2}}}dz}  = \mathop \smallint \limits_{ - R}^R \frac{{{e^{iz}}}}{{1 + {z^2}}}dz + \int_{{S_R}} {\frac{{{e^{iz}}}}{{1 + {z^2}}}dz}  \\ 
   \Downarrow  \\ 
  \mathop {\lim }\limits_{R \to \infty } \mathop \smallint \limits_{ - R}^R \frac{{{e^{ix}}}}{{1 + {x^2}}}dx = \mathop {\lim }\limits_{R \to \infty } \left( {\int_{{\Gamma _R}} {\frac{{{e^{iz}}}}{{1 + {z^2}}}dz}  - \int_{{S_R}} {\frac{{{e^{iz}}}}{{1 + {z^2}}}dz} } \right). \\ 
\end{gathered} \] A reziduum tétel alapján
\({\int_{\Gamma_{R}}{\frac{e^{iz}}{1 + z^{2}}dz}} = 2\pi i \cdot \text{Rez}_{i}\left( z\mapsto\frac{e^{iz}}{1 + z^{2}} \right)\).
A reziduum kiszámításához vegyük észre, hogy
\[\underbrace {\frac{{{e^{iz}}}}{{1 + {z^2}}}}_f = \underbrace {{e^{iz}}}_\phi  \cdot \underbrace {\frac{1}{{1 + {z^2}}}}_\psi  \Rightarrow {\text{Rez}}_i\left( {\phi  \cdot \psi } \right) = \phi \left( i \right) \cdot {{\text{Rez}}_i}\left( \psi  \right) = \phi \left( i \right).\frac{1}{{h'\left( i \right)}},\]
ahol \(h\left( z \right) = 1 + z^{2}\), így
\[{{\text{Rez}}_i}\left( f \right) = {e^{ - 1}}\frac{1}{{2i}} \Rightarrow \int_{{\Gamma _R}} {\frac{{{e^{iz}}}}{{1 + {z^2}}}dz}  = 2\pi i\frac{1}{{e \cdot 2i}} = \frac{\pi }{e}.\]
Most belátjuk, hogy
\(\lim\limits_{R\rightarrow\infty}{\int_{S_{R}}\frac{e^{iz}}{1 + z^{2}}} = 0\).
A számítás során felhasználjuk, hogy
\(\left| e^{iz} \right| = e^{\Re{({iz})}} \leq 1\), és hogy
\(\left| {1 + z^{2}} \right| \geq \left| z^{2} \right| - 1 = \left| z \right|^{2} - 1\).
\[\left. \left| {\int_{S_{R}}{\frac{e^{iz}}{1 + z^{2}}dz}} \right| \leq \pi R \cdot \sup\limits_{z \in S_{R}}\left| \frac{e^{iz}}{1 + z^{2}} \right| \leq \pi R\frac{1}{\left| z \right|^{2} - 1} = \pi R\frac{1}{R^{2} - 1}\rightarrow 0, \right.\]
ha \(\left. R\rightarrow\infty \right.\), így
\({\int\limits_{- \infty}^{\infty}{\frac{e^{ix}}{1 + x^{2}}dx}} = \lim\limits_{R\rightarrow\infty}{\int\limits_{- R}^{R}{\frac{e^{iz}}{1 + z^{2}}dz}} = \frac{\pi}{e}\).

\hypertarget{komplex-fuggvenyek-inverze}{%
\subsubsection{Komplex függvények
inverze}\label{komplex-fuggvenyek-inverze}}

Először az ún. lokális inverz létezését vizsgáljuk.

\begin{allitas}

Állítás:\\
Tfh \(f\) holomorf a \(z_{0}\) pont egy környezetében.

\begin{itemize}
\tightlist
\item
  Ha \(f'\left( z_{0} \right) = 0\), akkor \(f\) -nek nincs lokális
  inverze, semmilyen kis környezetében.
\item
  Ha \(f'\left( z_{0} \right) \neq 0\), akkor \(f\)-t a \(z_{0}\) pont
  elég kis környezetére leszűkítve, \(f\)-nek létezik inverze, az inverz
  függvény értelmezve és holomorf a \(w_{0} = f\left( z_{0} \right)\)
  pont egy környezetében.
\end{itemize}

\end{allitas}

\begin{megjegyzes}

Megjegyzés:\\
Abból, hogy
\(\exists f'\left( z \right) \neq 0,\forall z \in \Omega\Rightarrow f\)
injektív. Például nézzük az \(f\left( z \right) = e^{z}\) függvényt,
mely \(2\pi i\) szerint periodikus, vagyis
\(f\left( {z + 2\pi i} \right) = f\left( z \right)\). Ez a függvény
tehát nem injektív, pedig
\(f'\left( z \right) = \left( e^{z} \right)' = e^{z} \neq 0,\forall z \in {\mathbb{C}}\).

\end{megjegyzes}

\begin{allitas}

Állítás:\\
Az \(f\left( z \right) = e^{z}\) függvény injektív az
\(\Omega: = \left\{ {{\mathbb{C}} \ni z = x + iy:x \in {\mathbb{R}},0 \leq y < 2\pi} \right\}\)
-n és
\(R_{f} = {\mathbb{C}}\text{\textbackslash}\left\{ \text{0} \right\}\).

\end{allitas}

\begin{bizonyitas}

Bizonyítás:\\
Legyen \(w \in {\mathbb{C}}\text{\textbackslash}\left\{ 0 \right\}\) és
\(e^{z} = w\).
\(\left. z = x + iy\Rightarrow e^{z} = e^{x + iy} = e^{x}\left( {\cos\left( y \right) + i\sin\left( y \right)} \right) \right.\),
valamint
\(w = r \cdot e^{i\phi} = r \cdot \left( {\cos\left( \phi \right) + i\sin\left( \phi \right)} \right)\),
ahol \(r = \left| w \right|\),
\(\left\lbrack {0,2\pi} \right) \ni \phi = \arg\left( w \right)\).
\(\left. e^{x} = r = \left| w \right|\Leftrightarrow x = \ln\left| w \right| \right.\),
\(y = \phi = \arg\left( w \right)\), tehát
\[\left. e^{z} = w\Leftrightarrow z = x + iy = \ln\left| w \right| + i\arg\left( w \right). \right.\]

\end{bizonyitas}

Az előbbiek alapján szeretnénk definiálni a természetes alapú
logaritmust a komplex számokon.

\begin{definicio}

Definíció:\\
\(w \in {\mathbb{C}}\text{\textbackslash}\left\{ 0 \right\}\) esetén
legyen \(\log w: = \ln\left| w \right| + i\arg w\). (A logaritmus
függvény értelmezhető minden olyan tartományon, ahol az argumentum
egyértelműen értelmezhető. Mivel \(\log\left( z \right) = \ln z\), ha
\(z\) tisztán valós, olykor \(\log z\) helyett \(\ln z\) jelölést
használjuk, még ha \(z\) nem is tisztán valós.)

\end{definicio}

\hypertarget{konform-lekepezesek}{%
\subsection{Konform leképezések}\label{konform-lekepezesek}}

\begin{definicio}

Definíció:\\
Legyen \(\Omega \subset {\mathbb{C}}\) tartomány. Ha
\(\left. f:\Omega\rightarrow{\mathbb{C}} \right.\) holomorf és
\(f'\left( z \right) \neq 0,\forall z \in \Omega\), akkor \(f\)-t
konform leképezésnek nevezzük.

\end{definicio}

\begin{tetel}

Konform leképezések alaptétele:\\
Legyen \(\Omega \subset {\mathbb{C}}\) egyszeresen összefüggő tartomány,
melynek legalább 2 határpontja van. Ekkor létezik egyetlen \(f\) konform
leképezés, amely \(\Omega\) -t a \(\mathbb{C}\) egységkörére képezi
injektív módon úgy, hogy egy adott \(z_{0} \in \Omega\) pontra
\(f\left( z_{0} \right) = 0\), \(\arg f'\left( z_{0} \right) = \phi\)
adott.

\end{tetel}

\begin{pelda}

Példa:\\
Félsík konform leképezése az egységkörre.
\(\Omega: = \left\{ {z \in {\mathbb{C}}:\Im\left( z \right) > 0} \right\}\),
\(K_{1}: = \left\{ {w \in {\mathbb{C}}:\left| w \right| < 1} \right\}\).
\(f\left( z \right): = \frac{z - z_{0}}{z - \overline{z_{0}}}e^{i\phi}\),
ahol a felülvonás a komplex konjugálás. Láthatjuk, hogy \(f\) holomorf
\(\Omega\) -n, és \(\Im\left( z \right) > 0\) miatt
\(\left. \frac{\left| {z - z_{0}} \right|}{\left| {z - \overline{z_{0}}} \right|} < 1\Leftrightarrow\frac{z - z_{0}}{z - \overline{z_{0}}} \in K_{1}\Rightarrow w \in K_{1} \right.\).
Ha \(\Im\left( z \right) = 0\), akkor
\(\left| \frac{z - z_{0}}{z - \overline{z_{0}}} \right| = 1\), ha
\(\left. \Im\left( z \right) < 0\Rightarrow\left| \frac{z - z_{0}}{z - \overline{z_{0}}} \right| > 1 \right.\),
vagyis ekkor \(w\) a \(K_{1}\) -en kívül van.

\end{pelda}

\hypertarget{alkalmazas-aramlastani-feladatokra}{%
\subsubsection{Alkalmazás áramlástani
feladatokra}\label{alkalmazas-aramlastani-feladatokra}}

Síkbeli az áramlás, ha az áramlás sebessége egy
\(\left( {x,y} \right) \in {\mathbb{R}}^{2}\) pontban
\[ \tilde w\left( {x,y} \right): = \left( {\tilde u\left( {x,y} \right),\tilde v\left( {x,y} \right)} \right),\]
ahol az alábbi jelölések érvényesek: \[\begin{aligned}
  \tilde u\left( {x,y} \right): =  & u\left( {x + iy} \right) \\ 
  \tilde v\left( {x,y} \right): =  & v\left( {x + iy} \right) \\ 
  w\left( {x + iy} \right): =  & u\left( {x + iy} \right) + iv\left( {x + iy} \right) \\ 
  \bar w\left( {x + iy} \right) =  & u\left( {x + iy} \right) - i \cdot v\left( {x + iy} \right). \\ 
\end{aligned} \]

Bizonyos fizikai feltételek teljesülése esetén az áramlás divergencia-
és rotációmentes, vagyis
\(0 = {\text{div}}\left( {\tilde w} \right): = \frac{{\partial \tilde u}}{{\partial x}} + \frac{{\partial \tilde v}}{{dy}}\)
és
\(0 = {\text{rot}}\left( {\tilde w} \right) = \frac{{\partial \tilde v}}{{\partial x}} - \frac{{\partial \tilde u}}{{\partial y}}.\)
A \(\overline{w}\) függvényre teljesülnek a Cauchy-Reimann parciális
differenciálegyenletek \(\left. \Rightarrow\overline{w} \right.\)
holomorf függvény \(\widetilde{u}\) és \(\widetilde{v}\) folytonosan
differenciálható). A \(\overline{w}\) függvénynek létezik primitív
függvénye, \(f'\left( z \right): = \overline{w}\), \(F: = f + ig\).
Ekkor \[\left. {\begin{array}{*{20}{l}}
  {F' = \bar w = u - iv} \\ 
  {F' = \frac{{\partial f}}{{\partial x}} + i\frac{{\partial g}}{{\partial x}}} 
\end{array}} \right\}u = \frac{{\partial f}}{{\partial x}},v =  - \frac{{\partial g}}{{\partial x}}.\]
Mivel \(F\) holomorf, ezért \(F\)-re is teljesülnek a C-R egyenletek:
\(\frac{\partial f}{\partial x} = \frac{\partial g}{\partial y}\),
\(\frac{\partial f}{\partial y} = - \frac{\partial g}{\partial x}\)
\(\left. \Rightarrow u = \frac{\partial f}{\partial x} = \frac{\partial g}{\partial y},\, v = - \frac{\partial g}{\partial x} = \frac{\partial f}{\partial y} \right.\).
A fizikában \(f\)-et a sebesség potenciáljaként definiáljuk. Belátjuk,
hogy \(g\) az áramvonalak mentén állandó. Áramvonal: olyan
\(\left. \left( {\phi,\psi} \right):\left\lbrack {\alpha,\beta} \right\rbrack\rightarrow{\mathbb{R}}^{2} \right.\)
folytonosan differenciálható görbe, melynél
\[\left( {\overset{.}{\phi}\left( t \right),\overset{.}{\psi}\left( t \right)} \right) \parallel \left( {\widetilde{u}\left( {\phi\left( t \right),\psi\left( t \right)} \right),\widetilde{v}\left( {\phi\left( t \right),\psi\left( t \right)} \right)} \right),\, t \in {\left\lbrack {\alpha,\beta} \right\rbrack,}\]vagyis
melynél a görbe érintővektora párhuzamos a helyi sebességvektorral.
Ekkor \[\begin{array}{l}
\begin{array}{l}
\begin{array}{ll}
{\widetilde{u}\left( {\phi\left( t \right),\psi\left( t \right)} \right) = \lambda\left( t \right) \cdot \overset{.}{\phi}\left( t \right)} & {/ \cdot \overset{.}{\psi}\left( t \right)} \\
{\widetilde{v}\left( {\phi\left( t \right),\psi\left( t \right)} \right) = \lambda\left( t \right) \cdot \overset{.}{\psi}\left( t \right)} & {/ \cdot \overset{.}{\phi}\left( t \right)} \\
\end{array} \\
\end{array} \\
{\widetilde{u}\left( {\phi\left( t \right),\psi\left( t \right)} \right)\overset{.}{\psi}\left( t \right) - \widetilde{v}\left( {\phi\left( t \right),\psi\left( t \right)} \right)\overset{.}{\phi}\left( t \right) = 0.} \\
\end{array}\] A C-R egyenletekből következőket felhasználva, majd a
közvetett függvény deriválására vonatkozó összefüggésből
\[\left. \frac{\partial\widetilde{g}}{\partial y}\left( {\phi\left( t \right),\psi\left( t \right)} \right)\overset{.}{\psi}\left( t \right) + \frac{\partial\widetilde{g}}{\partial x}\left( {\phi\left( t \right),\psi\left( t \right)} \right)\overset{.}{\phi}\left( t \right) = 0\Leftrightarrow\frac{d}{dt}\left\lbrack {\widetilde{g}\left( {\phi\left( t \right),\psi\left( t \right)} \right)} \right\rbrack = 0, \right.\]
tehát a
\(\left. t\mapsto\widetilde{g}\left( {\phi\left( t \right),\psi\left( t \right)} \right) \right.\)
függvény állandó.

Ábra: ide kell betenni a konform.eps filet.

\hypertarget{lebesgue-integral}{%
\section{Lebesgue-integrál}\label{lebesgue-integral}}

A Reimann integrál hátrányai:

\begin{itemize}
\tightlist
\item
  Csak véges intervallumon és korlátos függvények esetén értelmezhető
  közvetlenül.
\item
  Az integrál és a limesz felcserélhetősége csak egyenletes konvergencia
  esetén lehetséges.
\item
  Nevezetes függvényterek nem vezethetők be.
\end{itemize}

\begin{definicio}

Definíció:\\
Legyen \(I\) valamilyen \({\mathbb{R}}^{n}\) -beli intervallum, azaz
\(I: = I_{1} \times I_{2} \times ... \times I_{n}\), ahol
\(I_{j} = \left\lbrack {a_{j},b_{j}} \right\rbrack \in {\mathbb{R}}\)
egydimenziós intervallumok. Ekkor az \(I\) Lebesgue mértéke:
\(\lambda\left( I \right): = \lambda\left( I_{1} \right) \cdot \lambda\left( I_{2} \right) \cdot ... \cdot \lambda\left( I_{n} \right)\),
ahol \(\lambda\left( I_{j} \right) = b_{j} - a_{j}\).

\end{definicio}

\begin{definicio}

Definíció:\\
Egy \(A \subset {\mathbb{R}}^{n}\) halmazt nullmértékűnek nevezünk, ha
\(\forall\varepsilon > 0\) szám esetén az \(A\) halmaz lefedhető
megszámlálhatóan (véges vagy végtelen) sok intervallummal úgy, hogy azok
mértékének összege \(\leq \varepsilon\), vagyis
\(A \subset {\bigcup\limits_{k = 1}^{\infty}I^{k}},\, I^{k} \subset {\mathbb{R}}^{n}\),
\({\sum\limits_{k = 1}^{\infty}{\lambda\left( I^{k} \right)}} \leq \varepsilon\).

\end{definicio}

\begin{pelda}

Példák:

\begin{enumerate}
\def\labelenumi{\arabic{enumi}.}
\tightlist
\item
  Minden megszámlálhatóan (véges vagy végtelen) sok pontból álló halmaz
  \({\mathbb{R}}^{n}\) -ben nullmértékű. Legyen \(\varepsilon > 0\)
  tetszőleges. Az \(i\)-edik pontot lefedjük egy \(I_{i}\) kis
  intervallummal úgy, hogy mértéke \(\frac{\varepsilon}{2^{i}}\) legyen,
  tehát az első pontot például egy \(\frac{\varepsilon}{2}\) mértékű
  intervallummal. Mivel
  \(\sum\limits_{k = 1}^{\infty}{\frac{\varepsilon}{2^{k}} = \varepsilon}\),
  így az említett halmaz valóban nullmértékű
\item
  \({\mathbb{R}}^{2}\) -ben egy egyenes szakasz nullmértékű, ugyanis
  lefedhető tetszőlegesen kis magasságú téglalappal
\item
  \({\mathbb{R}}^{2}\) -ben minden egyenes is nullmértékű.

  Ábra: ide kell betenni a lebesgue\_A1\_A2\_A3\_A4\_A5.eps filet.

  Ehhez belátjuk, hogy megszámlálhatóan végtelen sok nullmértékű halmaz
  uniója is nullmértékű:
  \(A: = {\bigcup\limits_{j = 1}^{\infty}A_{j}}\), ekkor az első ponthoz
  hasonlóan \(A_{j}\) -t befedjük egy legfeljebb
  \(\frac{\varepsilon}{2^{j}}\) mértékűvel, vagyis \(A_{1}\) -t egy
  legfeljebb \(\frac{\varepsilon}{2}\) -vel, \(A_{2}\) -t egy legfeljebb
  \(\frac{\varepsilon}{2^{2}}\) -tel... Így felhasználva ismét a
  \(\sum\limits_{k = 1}^{\infty}{\frac{\varepsilon}{2^{k}} = \varepsilon}\)
  összefüggést, láthatjuk, hogy az említett halmaz valóban nullmértékű.
\end{enumerate}

\end{pelda}

\begin{ajanlo}

\begin{ajanlofig}

\href{https://xkcd.com}{\includegraphics[width=5.20833in,height=2.82292in]{wikipedian_protester.png}}

\end{ajanlofig}

Text

\end{ajanlo}

Tehát láttuk, hogy \({\mathbb{R}}^{2}\) -ben megszámlálhatóan sok
nullmértékű halmaz unója is nullmértékű. De ez nem csak
\({\mathbb{R}}^{2}\) -ben igaz, az érvelés hasonló általános esetben.
Legyen \(A_{j}\) nullmértékű, belátjuk, hogy
\(\bigcup\limits_{j = 1}^{\infty}A_{j}\) nullmértékű.
\[\begin{array}{*{20}{c}}
  {{A_1} \subset \bigcup\limits_{k = 1}^\infty  {{I_{1,k}}} ,\,\mathop \sum \limits_{k = 1}^\infty  \lambda \left( {{I_{1,k}}} \right) \leqslant \frac{\varepsilon }{2}} \\ 
  {{A_2} \subset \bigcup\limits_{k = 1}^\infty  {{I_{2,k}}} ,\,\mathop \sum \limits_{k = 1}^\infty  \lambda \left( {{I_{2,k}}} \right) \leqslant \frac{\varepsilon }{{{2^2}}}} \\ 
   \vdots  \\ 
  {{A_j} \subset \bigcup\limits_{k = 1}^\infty  {{I_{j,k}}} ,\,\mathop \sum \limits_{k = 1}^\infty  \lambda \left( {{I_{j,k}}} \right) \leqslant \frac{\varepsilon }{{{2^j}}}} \\ 
   \vdots  
\end{array}\] Ezek az \(I_{j',k}\) intervallumok megszámlálhatóan
végtelen sokan vannak, mert sorba rendezhetjük őket.\\
(Táblázatba rendezve őket, az átlók mentén a sorrend:
\(I_{1,1},\, I_{1,2},\, I_{2,1},\, I_{1,3},\, I_{2,2},\, I_{3,1}\ldots\))
Ekkor pedig
\(\sum\limits_{j' = 1}^{\infty}{\frac{\varepsilon}{2^{j'}} = \varepsilon}\)
miatt az unió is nullmértékű.

Előző órán láttuk, hogy \({\mathbb{R}}^{2}\) -ben egy egyenes
nullmértékű, ugyanis megszámlálhatóan végtelen sok nullmértékű halmaz
uniója. Hasonlóan, \({\mathbb{R}}^{3}\) -ben egy sík nullmértékű...

\hypertarget{lepcsos-fuggvenyek-integralja}{%
\subsection{Lépcsős függvények
integrálja}\label{lepcsos-fuggvenyek-integralja}}

\begin{definicio}

Definíció:\\
Legyen \(\left. f:{\mathbb{R}}^{n}\rightarrow{\mathbb{R}} \right.\)
olyan függvény, hogy véges sok intervallumban nem 0 állandó, máshol 0.
Ekkor \(f\)-t lépcsős függvénynek nevezzük.

\end{definicio}

\begin{definicio}

Definíció:\\
Legyen \(f\) lépcsős függvény, mely az \(I_{k}\) intervallumon \(c_{k}\)
-val egyenlő. Ekkor
\({\int f}: = {\sum\limits_{k = 1}^{\infty}{c_{k}\lambda\left( I_{k} \right)}}\).
Belátható, hogy a definíció egyértelmű.

\end{definicio}

\begin{tetel}

A lemma:\\
Legyen \(\left( f_{j} \right)\) lépcsős függvények monoton csökkenő
sorozata, amelyre
\(\lim\limits_{j\rightarrow\infty}f_{j}\left( x \right) = 0,\forall x \in \left\{ {{\mathbb{R}}\backslash A} \right\}\),
ahol \(A\) nullmértékű halmaz. Ezt úgy mondjuk, hogy a limesz majdnem
\(x\)-re vagy majdnem mindenütt 0. Ekkor az integrálok
\(\lim\limits_{j\rightarrow\infty}{\int f_{j}} = 0\) sorozata .
(Bizonyítás nélkül.)

\end{tetel}

\begin{tetel}

B lemma:\\
Legyen \(\left( f_{j} \right)\) lépcsős függvények egy monoton növő
sorozata, amelyre az integrálok sorozata \(\int f_{j}\) felülről
korlátos, \({\int f_{j}} \leq C \in {\mathbb{R}},\forall j\)
\(\left. \Leftrightarrow\lim{\int f_{j}} \right.\) véges. Ekkor majdnem
minden \(x \in {\mathbb{R}}^{n}\) pontra
\(\lim\limits_{j\rightarrow\infty}f_{j}\left( x \right)\) is véges.

\end{tetel}

\begin{bizonyitas}

Bizonyítás:\\
Legyen
\(M_{0}: = \left\{ {x \in {\mathbb{R}}^{n}:\lim\limits_{j\rightarrow\infty}f_{j}\left( x \right) = \infty} \right\}\)!
Ekkor belátandó, hogy \(M_{0}\) nullmértékű. Tetszőleges rögzített
\(\varepsilon > 0\) esetén legyen
\(M_{\varepsilon,j}: = \left\{ {x \in {\mathbb{R}}^{n}:f_{j}\left( x \right) > \frac{C}{\varepsilon}} \right\}\)!
Mivel \(\left( f_{j} \right)\) monoton növő, ezért
\(\left. M_{\varepsilon,1} \subset M_{\varepsilon,2} \subset M_{\varepsilon,3} \subset \ldots\Rightarrow M_{\varepsilon,N} = {\bigcup\limits_{j = 1}^{N}M_{\varepsilon,j}} \right.\).
Jelöljük:
\[M_{\varepsilon}: = {\left\{ {x \in {\mathbb{R}}^{n}:\exists j \in {\mathbb{N}}:f_{j}\left( x \right) > \frac{C}{\varepsilon}} \right\}.}\]
Ekkor
\(M_{\varepsilon} = {\bigcup\limits_{j = 1}^{\infty}M_{\varepsilon,j}}\),
\(M_{0} \subset M_{\varepsilon},\forall\varepsilon > 0\).
\(M_{\varepsilon,j}\) véges sok diszjunkt intervallum egyesítése, melyek
mértékének összege \(\leq \varepsilon,\forall j\), mert ha nem így
lenne, akkor az \({\int{f_{j} > \varepsilon\frac{C}{\varepsilon}}} = C\)
ellentmondásra vezetne. Tehát
\(M_{\varepsilon} = {\bigcup\limits_{j = 1}^{\infty}M_{\varepsilon,j}}\)
megszámlálhatóan végtelen sok intervallum uniója, \[\begin{gathered}
  {M_{\varepsilon ,1}} \subset {M_{\varepsilon ,2}} \subset ... \\ 
   \Downarrow  \\ 
  \lambda \left( {{M_{\varepsilon ,1}}} \right) \leqslant \lambda \left( {{M_{\varepsilon ,2}}} \right) \leqslant ... \leqslant \varepsilon  \\ 
   \Downarrow  \\ 
  \lambda \left( {{M_\varepsilon }} \right) = \lambda \left( {\bigcup\limits_{j = 1}^\infty  {{M_{\varepsilon ,j}}} } \right) \leqslant \varepsilon . \\ 
\end{gathered} \] Tehát
\(\left. M_{0} \subset M_{\varepsilon},\forall\varepsilon > 0\Rightarrow M_{0} \right.\)
nullmértékű.

\end{bizonyitas}

\begin{definicio}

Definíció:\\
Jelölje \(C_{1}\) az olyan
\(\left. f:{\mathbb{R}}^{n}\rightarrow{\mathbb{R}} \right.\) függvények
összességét, amelyekhez léteznek lépcsős függvények monoton növekedő
olyan \(\left( f_{j} \right)\) sorozata, hogy
\(f\left( x \right) = \lim\limits_{j\rightarrow\infty}f_{j}\left( x \right)\)
majdnem minden \(x\)-re és \(\left( {\int f_{j}} \right)\) sorozat
felülről korlátos
\(\left. \Leftrightarrow\exists\lim{\int f_{j}} < \infty \right.\).

\end{definicio}

\begin{megjegyzes}

Megjegyzés:\\
Tfh \(f \in C_{1}\). Ekkor az \(\int f\) -t így szeretnénk értelmezni:
\({\int f}: = \lim\limits_{j\rightarrow\infty}{\int f_{j}}\).

\end{megjegyzes}

Kérdés:

\begin{enumerate}
\def\labelenumi{\arabic{enumi}.}
\tightlist
\item
  Egy ilyen definíció egyértelmű lenne-e, vagyis függ-e az
  \(\left( f_{j} \right)\) sorozat megválasztásától?
\item
  Ha spec. \(f\) lépcsős függvény, akkor a régi és az új definíció
  azonos-e?
\end{enumerate}

\begin{tetel}

Tétel:\\
Legyenek \(f,g \in C_{1}\), \(f \leq g\), \(\left( f_{j} \right)\) és
\(\left( g_{j} \right)\) lépcsős függvények monoton sorozata úgy, hogy
\(\lim\left( f_{j} \right) = f\), \(\lim\left( g_{j} \right) = g\),
továbbá \(\int f_{j}\) és \(\int g_{j}\) korlátos. Ekkor
\(\lim{\int f_{j}} \leq \lim{\int g_{j}}\).

\end{tetel}

\begin{bizonyitas}

Bizonyítás:\\
Jelöljük egy
\(\left. h:{\mathbb{R}}^{n}\rightarrow{\mathbb{R}} \right.\) függvény
pozitív illetve negatív részét az alábbiak szerint.
\[h^{+}\left( x \right): = \left\{ \begin{matrix}
{h\left( x \right)} & {\text{ha~}h\left( x \right) > 0} \\
0 & {\text{ha~}h\left( x \right) \leq 0} \\
\end{matrix} \right.\] \[h^{-}\left( x \right): = \left\{ \begin{matrix}
{h\left( x \right)} & {\text{ha~}h\left( x \right) < 0} \\
0 & {\text{ha~}h\left( x \right) \geq 0} \\
\end{matrix} \right.\] Tekintsük rögzített \(j \in {\mathbb{N}}\) esetén
a következő függvénysorozatot:
\(\left( {f_{j} - g_{k}} \right)_{k \in {\mathbb{N}}}\). Mivel
\(\left( g_{k} \right)\) monoton növő,
\(\left( {f_{j} - g_{k}} \right)_{k \in {\mathbb{N}}}\) monoton
csökkentő függvénysorozat.
\(\lim\limits_{k\rightarrow\infty}\left( {f_{j} - g_{k}} \right) = f_{j} - g\)
majdnem mindenütt. Mivel \(\left( f_{j} \right)\) monoton növő és
\(\lim\left( f_{j} \right) = f\) majdnem mindenütt
\(\left. \Rightarrow f_{j} \leq f \leq g \right.\)
\(\left. \Rightarrow f_{j} \leq g \right.\), vagyis \(f_{j} - g \leq 0\)
majdnem mindenütt.
\(\lim\limits_{k\rightarrow\infty}\left( {f_{j} - g_{k}} \right) = f_{j} - g \leq 0\).
Tekintsük \(\left( {f_{j} - g_{k}} \right)_{k \in {\mathbb{N}}}^{+}\)
-t, ez is lépcsős függvénysorozat, ez is monoton csökkenő,
\(\lim\limits_{k\rightarrow\infty}\left( {f_{j} - g_{k}} \right)^{+} = \left( {f_{j} - g} \right)^{+} = 0\)
majdnem mindenütt. Alkalmazzuk az \(A\) lemmát az
\(\left( {f_{j} - g_{k}} \right)_{k \in {\mathbb{N}}}^{+}\) sorozatra
\(\left. \Rightarrow\lim\limits_{k\rightarrow\infty}{\int\left( {f_{j} - g_{k}} \right)^{+}} = 0 \right.\).
Nyilván \[\begin{gathered}
  {h^ - } \leqslant h \leqslant {h^ + } \\ 
   \Downarrow  \\ 
  {f_j} - {g_k} \leqslant {\left( {{f_j} - {g_k}} \right)^ + } \\ 
   \Downarrow  \\ 
  \int {\left( {{f_j} - {g_k}} \right)}  \leqslant \int {{{\left( {{f_j} - {g_k}} \right)}^ + }}  \\ 
   \Downarrow  \\ 
  \int {{f_j}}  - \int {{g_k}}  \leqslant \int {{{\left( {{f_j} - {g_k}} \right)}^ + }}.  \\ 
\end{gathered} \]Ekkor \(\left. k\rightarrow\infty \right.\) esetre
\[\begin{gathered}
  \int {{f_j}}  - \mathop {\lim }\limits_{k \to 0} \int {{g_k}}  \leqslant 0 \Rightarrow \int {{f_j}}  \leqslant \mathop {\lim }\limits_{k \to \infty } \int {{g_k}} ,\forall j \\ 
   \Downarrow  \\ 
  \mathop {\lim }\limits_{j \to \infty } \int {{f_j}}  \leqslant \mathop {\lim }\limits_{k \to \infty } \int {{g_k}}.  \\ 
\end{gathered} \]

\end{bizonyitas}

\textbf{Következmények}:

\begin{enumerate}
\def\labelenumi{\arabic{enumi}.}
\tightlist
\item
  Ha \(f = g \in C_{1}\) és \(\left( f_{j} \right)\) és
  \(\left( g_{j} \right)\) lépcsős függvények monoton növekedő sorozata,
  amelyekre \(\lim\left( f_{j} \right) = f\) majdnem mindenütt és
  \(\lim\left( g_{j} \right) = g\) majdnem mindenütt, akkor
  \(\left. \Rightarrow\lim{\int f_{j}} = \lim{\int g_{j}} \right.\).
  Most már lehet definiálni: \({\int f}: = \lim{\int f_{j}}\), ahol
  \(f \in C_{1}\).
\item
  Ha \(f\) spec. lépcsős függvény, akkor a régi és az új integrál
  definíciója azonos, ugyanis választható \(f_{j} = f\)-nek.
\item
  \(f,g \in C_{1}\) és
  \(\left. f \leq g\Rightarrow{\int f} \leq {\int g} \right.\).
\end{enumerate}

\hypertarget{az-integral-tulajdonsagai-c_1--ben}{%
\subsubsection{\texorpdfstring{Az integrál tulajdonságai \(C_{1}\)
-ben}{Az integrál tulajdonságai C\_\{1\} -ben}}\label{az-integral-tulajdonsagai-c_1--ben}}

\begin{enumerate}
\def\labelenumi{\arabic{enumi}.}
\tightlist
\item
  Ha
  \(\left. f,g \in C_{1}\Rightarrow\left( {f + g} \right) \in C_{1} \right.\)
  és \({\int\left( {f + g} \right)} = {\int f} + {\int g}\).
\item
  Tfh \(f \in C_{1},\lambda \geq 0\) állandó
  \(\lambda \cdot f \in C_{1}\) és
  \({\int{\lambda f}} = \lambda{\int f}\).
\item
  Ha
  \(\left. f \in C_{1}\Rightarrow f^{+} \in C_{1},f^{-} \in C_{1} \right.\).
\end{enumerate}

\begin{bizonyitas}

Bizonyítás:

\begin{enumerate}
\def\labelenumi{\arabic{enumi}.}
\tightlist
\item
  Definíció szerint \(\exists\left( f_{j} \right)\) és
  \(\exists\left( g_{j} \right)\) monoton növekedő lépcsős
  függvénysorozatok, melyekre \(\lim\left( f_{j} \right) = f\) majdnem
  mindenütt, \(\lim\left( g_{j} \right) = g\) majdnem mindenütt és
  \({\int f} = \lim{\int f_{j}}\) valamint
  \({\int g} = \lim{\int g_{j}}\). \(\left( {f_{j} + g_{j}} \right)\)
  lépcsős függvények monoton növő sorozata,
  \(\lim\left( {f_{j} + g_{j}} \right) = f + g\). \[\begin{gathered}
    \lim \int {\left( {{f_j} + {g_j}} \right)}  = \lim \left( {\int {{f_j}}  + \int {{g_j}} } \right) = \lim \int {{f_j}}  + \lim \int {{g_j}}  \\ 
     \Downarrow  \\ 
    \int {\left( {f + g} \right)}  = \lim \int {{f_j}}  + \lim \int {{g_j}}  = \int f  + \int g  \\ 
  \end{gathered} \]
\item
  \(\left. f \in C_{1}\Rightarrow\exists\left( f_{j} \right) \right.\)
  lépcsős függvények monoton növő sorozata, hogy
  \(\lim\left( f_{j} \right) = f\) majdnem mindenütt. Ekkor
  \(\left. \lim{\int f_{j}} = {\int f}\Rightarrow\left( {\lambda f_{j}} \right) \right.\)
  lépcsős függvények monoton sorozata,
  \(\left. \lim{\int{\lambda f_{j}}} = \lambda\lim{\int f_{j}} = \lambda{\int f}\Rightarrow{\int{\lambda f}} = \lambda{\int f} \right.\).
\item
  \(\left. \left( f_{j} \right)\rightarrow f \right.\) monoton növekvő,
  \(\left. \left( f_{j}^{+} \right)\rightarrow f^{+} \right.\) monoton
  növekvő, és
  \(\left. \left( f_{j}^{-} \right)\rightarrow f^{-} \right.\) szintén
\end{enumerate}

\end{bizonyitas}

\begin{definicio}

Definíció:\\
\(\left( {f \cup g} \right)\left( x \right): = \max\left\{ {f\left( x \right),g\left( x \right)} \right\}\),
\(\left( {f \cap g} \right)\left( x \right): = \min\left\{ {f\left( x \right),g\left( x \right)} \right\}\).

\end{definicio}

\begin{allitas}

Állítás:\\
Ha
\(\left. f,g \in C_{1}\Rightarrow\left( {f \cup g} \right) \in C_{1},\left( {f \cap g} \right) \in C_{1} \right.\).

\end{allitas}

\hypertarget{integralas-a-c_2-osztalyaban}{%
\subsection{\texorpdfstring{Integrálás a \(C_{2}\)
osztályában}{Integrálás a C\_\{2\} osztályában}}\label{integralas-a-c_2-osztalyaban}}

\begin{definicio}

Definíció:\\
Ha \(f = f_{1} - f_{2}\), ahol \(f_{1},f_{2} \in C_{1}\), akkor
\(f \in C_{2}\). Ekkor legyen
\({\int f}: = {\int f_{1}} - {\int f_{2}}\).

\end{definicio}

\begin{allitas}

Állítás:\\
Az integrál definíciója egyértelmű.

\end{allitas}

\begin{bizonyitas}

Bizonyítás:\\
Legyen \(f = f_{1} - f_{2} = g_{1} - g_{2}\), ahol
\(f_{1},f_{2},g_{1},g_{2} \in C_{1}\).
\({\int f_{1}} - {\int f_{2}} = {\int g_{1}} - {\int g_{2}}\) ugyanis
\({\int f_{1}} + {\int g_{2}} = {\int g_{1}} + {\int f_{2}}\), mert
\(\left. f_{1} + g_{2} = g_{1} + f_{2}\Rightarrow{\int f_{1}} + {\int g_{2}} = {\int g_{1}} + {\int f_{2}} \right.\).

\end{bizonyitas}

\hypertarget{a-c_2--beli-integral-tulajdonsagai}{%
\subsubsection{\texorpdfstring{A \(C_{2}\) -beli integrál
tulajdonságai:}{A C\_\{2\} -beli integrál tulajdonságai:}}\label{a-c_2--beli-integral-tulajdonsagai}}

\begin{enumerate}
\def\labelenumi{\arabic{enumi}.}
\tightlist
\item
  Ha \(\left. f,g \in C_{2}\Rightarrow f + g \in C_{2} \right.\) és
  \({\int{f + g}} = {\int f} + {\int g}\).
\item
  \(\left. f \in C_{2},\lambda \in {\mathbb{R}}\Rightarrow\lambda \cdot f \in C_{2} \right.\)
  és \({\int{\lambda f}} = \lambda{\int f}\)
\item
  \(f,g \in C_{2}\),
  \(\left. f \leq g\Rightarrow{\int f} \leq {\int g} \right.\)
\item
  Ha \(f \in C_{2}\), akkor egy nullmértékű halmazon megváltoztatva
  szintén továbbra is \(\in C_{2}\) marad, és az integrál értéke nem
  változik.
\item
  Ha
  \(\left. f \in C_{2}\Rightarrow f^{+},f^{-},\left| f \right| \in C_{2} \right.\).
\item
  Ha
  \(\left. f \in C_{2}\Rightarrow\left| {\int f} \right| \leq {\int\left| f \right|} \right.\).
\item
  Legyen \(f \in C_{2}\), ekkor \(\exists\left( f_{j} \right)\) lépcső
  függvényekből álló sorozat (nem feltétlen monoton), hogy
  \(\left. \left( f_{j} \right)\rightarrow f \right.\) majdnem mindenhol
  és \(\left. {\int f_{j}}\rightarrow{\int f} \right.\)
\end{enumerate}

\begin{bizonyitas}

Bizonyítás:

\begin{enumerate}
\def\labelenumi{\arabic{enumi}.}
\tightlist
\item
  \(f = f_{1} - f_{2}\), \(g = g_{1} - g_{2}\), ahol
  \[{f_1},{f_2},{g_1},{g_2} \in {C_1} \Rightarrow f + g = \underbrace {\left( {{f_1} + {g_1}} \right)}_{ \in {C_1}} - \underbrace {\left( {{f_2} + {g_2}} \right)}_{ \in {C_1}},\]
  így ekkor \[\begin{aligned}
    \int {\left( {f + g} \right)}  &  = \int {\left( {{f_1} + {g_1}} \right)}  - \int {\left( {{f_2} + {g_2}} \right)}  \\ 
     &  = \left[ {\int {\left( {{f_1}} \right)}  + \int {\left( {{g_1}} \right)} } \right] - \left[ {\int {\left( {{f_2}} \right)}  + \int {\left( {{g_2}} \right)} } \right] \\ 
     &  = \left( {\int {{f_1}}  - \int {{f_2}} } \right) + \left( {\int {{g_1}}  - \int {{g_2}} } \right) \\ 
     &  = \int f  + \int g . \\ 
  \end{aligned} \]
\item
  \(f = f_{1} - f_{2},f_{1} \in C_{1}\), ekkor
  \(\lambda f = \lambda f_{1} - \lambda f_{2}\). Ha \(\lambda \geq 0\),
  akkor \({\underbrace {\lambda {f_1}}_{ \in {C_1}}}\). Ha
  \(\lambda <0 \Rightarrow \lambda f = \underbrace {\left( { - \lambda } \right)}_{ > 0}{f_2} - \underbrace {\left( { - \lambda } \right)}_{ > 0}{f_1}\)
  .
\item
  \(\left. f \leq g\Rightarrow g - f \geq 0 \right.\). Azt kellene
  igazolni, hogy ekkor
  \(\int {\underbrace {\left( {g - f} \right)}_{ = h}} \geqslant 0\).
  \(h \geq 0\), \(h: = h_{1} - h_{2}\),
  \[\left. h_{j} \in C_{1}\Rightarrow h_{1} \geq h_{2}\Rightarrow{\int h_{1}} \geq {\int h_{2}}\Rightarrow{\int h_{1}} - {\int h_{2}} \geq 0\Rightarrow{\int h} \geq 0 \right.\].
\item
  Házi feladat.
\item
  \(f = f_{1} - f_{2},f_{j} \in C_{1}\).
  \(\left| f \right| = \underbrace {\left( {{f_1} \cup {f_2}} \right)}_{ \in {C_1}} - \underbrace {\left( {{f_1} \cap {f_2}} \right)}_{ \in {C_1}}\).
  \(f^{+} = \left( {f_{1} \cup f_{2}} \right) - f_{2}\),
  \({f^ - } = \underbrace {\left( {{f_1} \cup {f_2}} \right)}_{ \in {C_1}} - \underbrace {{f_1}}_{ \in {C_1}}.\)
\item
  \(\left. - \left| f \right| \leq f \leq \left| f \right|\Rightarrow - {\int{\left| f \right| \leq {\int f}}} \leq {\int\left| f \right|} \right.\).
\item
  \(\left. f \in C_{2}\Rightarrow\exists g,h:f = g - h \right.\), ahol
  \(\left. g,h \in C_{1}\Rightarrow\exists\left( g_{j} \right) \right.\)
  (monoton növő) lépcsős függvénysorozat, hogy
  \(\left. \left( g_{j} \right)\rightarrow g \right.\) majdnem mindenütt
  és \(\lim{\int g_{j}} = {\int g}\) továbbá
  \(\exists\left( h_{j} \right)\) (monoton növő) lépcsős
  függvénysorozat, hogy
  \(\left. \left( h_{j} \right)\rightarrow h \right.\) majdnem mindenütt
  és
  \(\left. \lim{\int h_{j}} = {\int h}\Rightarrow\lim\left( {g_{j} - h_{j}} \right): = \lim\left( f_{j} \right) = g - h = f \right.\)
  majdnem mindenütt,
  \({\int{\lim\left( {g_{j} - h_{j}} \right)}} = {\int\left( {\lim g_{j} - \lim h_{j}} \right)} = \lim{\int g_{j}} - \lim{\int h_{j}} = {\int g} - {\int h} = {\int f}\).
\end{enumerate}

\end{bizonyitas}

\begin{allitas}

Állítás:\\
Ha
\(\left. f,g \in C_{2}\Rightarrow\left( {f \cup g} \right) \in C_{2},\left( {f \cap g} \right) \in C_{2} \right.\).

\end{allitas}

\begin{bizonyitas}

Bizonyítás:\\
\(f \cup g = \underbrace {{{\left( {f - g} \right)}^ + }}_{ \in {C_2}} + \underbrace g_{ \in {C_2}}\),
\(f \cap g = \underbrace f_{ \in {C_2}} - \underbrace {{{\left( {f - g} \right)}^ + }}_{ \in {C_2}}\).

\end{bizonyitas}

\begin{tetel}

Beppo Levi tétele (monoton sorozatokból, illetve nemnegatív tagú
sorokról):

\begin{enumerate}
\def\labelenumi{\arabic{enumi}.}
\tightlist
\item
  Tfh \(f_{j} \in C_{2}\) (integrálható), \(\left( f_{j} \right)\)
  monoton nő és \(\lim{\int f_{j}}\) véges
  \(\left. \Leftrightarrow{\int f_{j}} \right.\) felülről korlátos).
  Ekkor
  \(f\left( x \right): = \lim\left( {f_{j}\left( x \right)} \right)\)
  véges majdnem minden \(x\)-re, továbbá \(f\) is integrálható,
  \({\int f} = \lim{\int f_{j}}\).
\item
  Sorokra: tfh \(g_{j} \in C_{2}\), \(g_{j} \geq 0\) és
  \({\sum\limits_{j = 1}^{\infty}{\int g_{j}}} < \infty\). Ekkor majdnem
  minden \(x\)-re
  \(C_{2} \ni f\left( x \right): = {\sum\limits_{j = 1}^{\infty}{g_{j}\left( x \right)}} < \infty\)
  (a sor konvergens) és
  \({\int f} = {\sum\limits_{j = 1}^{\infty}{\int g_{j}}}\).
\end{enumerate}

\end{tetel}

A két állítás egymással ekvivalens, ugyanis legyen
\(f_{k}: = {\sum\limits_{k = 1}^{k}g_{j}}\). Az \(f_{k}\) monoton nő
\(\left. \Leftrightarrow g_{j} \geq 0 \right.\),
\(\lim{\int f_{k}} = \lim{\int{\sum\limits_{j = 1}^{k}g_{j}}} = \lim{\sum\limits_{j = 1}^{k}{\int g_{j}}}\).
A sorokra vonatkozó formáját fogjuk bizonyítani.

\begin{ajanlo}

\begin{ajanlofig}

\href{https://xkcd.com}{\includegraphics[width=5.20833in,height=2.82292in]{wikipedian_protester.png}}

\end{ajanlofig}

Text

\end{ajanlo}

\begin{bizonyitas}

Beppo Levi tételének bizonyítása

Két részre bontjuk a bizonyítást, első részben \(g_{j} \in C_{1}\).\\
Ez azt jelenti, hogy
\(\exists h_{j_{k}}:\lim\limits_{k\rightarrow\infty}\left( h_{j_{k}} \right) = g_{j}\)
majdnem mindenütt, ahol \(h_{j_{k}}\) lépcsős függvények, monoton nőnek,
továbbá
\({\int{g_{j} =}}\lim\limits_{k\rightarrow\infty}{\int h_{j_{k}}}\).
Mivel \(g_{j} \geq 0\), ezért feltehető, hogy \(h_{j_{k}} \geq 0\),
ugyanis \(h_{j_{k}}\) helyett választhatnánk a \(h_{{}_{j_{k}}}^{+}\)
függvényeket is. 5let: jelöljük
\(H_{k}: = {\sum\limits_{j = 1}^{k}h_{j_{k}}}\), ekkor \(H_{k}\) is
lépcsős függvény és \(\left( H_{k} \right)\) monoton növő sorozat,
ugyanis
\(H_{k + 1} = {\sum\limits_{j = 1}^{k + 1}h_{j_{k + 1}}} \geq {\sum\limits_{j = 1}^{k}h_{j_{k + 1}}} \geq {\sum\limits_{j = 1}^{k}h_{j_{k}}} = H_{k}\),
továbbá \(h_{j_{k}} \leq g_{j}\), mert
\(\left( h_{j_{k}} \right)_{k \in {\mathbb{N}}}\) monoton növőleg tart
\(g_{j}\) -hez.
\[{H_k} = \sum\limits_{j = 1}^k {{h_{{j_k}}}}  \leqslant {G_k}: = \sum\limits_{j = 1}^k {{g_j}}  \Rightarrow \int {{H_k}}  \leqslant \int {{G_k}}  = \int {\sum\limits_{j = 1}^k {{g_j}} }  = \sum\limits_{j = 1}^k {\int {{g_j}} }  \leqslant \sum\limits_{j = 1}^\infty  {\int {{g_j}} }  < \infty \]
Alkalmazzuk a B lemmát a \(\left( H_{k} \right)\) sorozatra. Eszerint
\(\left. \exists H:\lim\left( H_{k} \right) = H\Rightarrow H \in C_{1} \right.\)
majdnem mindenütt, és \({\int H} = \lim{\int H_{k}}\). Legyen \(m > k\),
ekkor
\(H_{m} = {\sum\limits_{j = 1}^{m}h_{j_{m}}} \geq {\sum\limits_{j = 1}^{k}h_{j_{m}}}\).
Ha most \(\left. m\rightarrow\infty \right.\), akkor
\(\left. H_{m}\rightarrow H,h_{j_{m}}\rightarrow g_{j} \right.\) majdnem
mindenütt, így \(H \geq {\sum\limits_{j = 1}^{k}g_{j}} = G_{k}\). Tehát
\(H_{k} \leq G_{k} \leq H\). Ha most
\(\left. k\rightarrow\infty \right.\), akkor
\(\left. H_{k}\rightarrow H \right.\), így
\(\left. G_{k}\rightarrow H \right.\).
\(\lim\limits_{k\rightarrow\infty}{\sum\limits_{j = 1}^{k}{g_{j}\left( x \right)}} = H\left( x \right) = :f\left( x \right)\),
vagyis
\(\sum\limits_{j = 1}^{\infty}{g_{j}\left( x \right) = f\left( x \right)}\).
\(\left. H \in C_{1}\Leftrightarrow f \in C_{1} \right.\).
\(\left. H_{k} \leq G_{k} \leq H\Rightarrow{\int H_{k}} \leq {\int G_{k}} \leq {\int H} \right.\),
ahol \(\left. {\int H_{k}}\rightarrow{\int H} \right.\), így
\(\left. {\int G_{k}}\rightarrow{\int H} = {\int f} \right.\).
\(\int{{\sum\limits_{j = 1}^{k}g_{j}} = {\sum\limits_{j = 1}^{k}{\int\left. g_{j}\rightarrow{\int f} \right.}}}\).

\textbf{Most a második része a bizonyításnak:} általános estben
vizsgálódunk, mikor \(g_{j} \in C_{2}\). Észrevétel:
\(\forall\varepsilon > 0\forall\phi \in C_{2}\exists\phi_{1},\phi_{2} \in C_{1}:\phi = \phi_{1} - \phi_{2},\phi_{2} \geq 0,{\int{\phi_{2} \leq \varepsilon}}\).
Ugyanis tetszőleges \(\phi \in C_{2}\) esetén \(\phi\) előállítható
\(\phi = \phi_{1} - \phi_{2}\) formában, ahol
\(\phi_{1},\phi_{2} \in C_{1}\). Mivel \(\phi_{2} \in C_{1}\), ezért
\(\exists\left( \psi_{k} \right)\) monoton növő lépcsős függvény
sorozat, amelyre \(\lim\left( \psi_{k} \right) = \phi_{2}\) majdnem
mindenütt , \(\lim{\int{\left( \psi_{k} \right) = {\int\phi_{2}}}}\).
\(\left. \phi_{2} \geq \psi_{k}\Rightarrow\phi_{2} - \psi_{k} \geq 0,\forall k \right.\).
\(\forall\varepsilon > 0\exists k_{0}:{\int\left( {\phi_{2} - \psi_{k_{0}}} \right)} \leq \varepsilon\).
Így
\(\phi = \underbrace {\left( {{\phi _1} - {\psi _{{k_0}}}} \right)}_{ \in {C_1}} - \underbrace {\left( {{\phi _2} - {\psi _{{k_0}}}} \right)}_{ \geqslant 0, \in {C_1},\smallint \left( {{\phi _2} - {\psi _{{k_0}}}} \right) \leqslant \varepsilon }\).

Alkalmazzuk tehát az észrevételt \(g_{j} \in C_{2}\) függvényekre:
\(g_{j} = g_{j,1} - g_{j,2}\), ahol \(g_{j,1},g_{j,2} \in C_{1}\) és
\(g_{j,2} \geq 0\), \({\int g_{j,2}} \leq \frac{1}{2^{j}}\). Tehát
\(g_{j,2} \geq 0\), \(g_{j,2} \in C_{1}\),
\(\sum\limits_{j = 1}^{\infty}{{\int{g_{j,2} \leq {\sum\limits_{j = 1}^{\infty}\frac{1}{2^{j}}}}} < \infty}\).
A \(g_{j,2}\) tagokból álló sorra alkalmazható a bizonyítás első része,
így
\({\sum\limits_{j = 1}^{\infty}{g_{j,2}\left( x \right)}}: = g_{2}^{*}\left( x \right)\)
konvergens majdnem minden \(x\)-re, \(g_{2}^{*} \in C_{1}\),
\({\int g_{2}^{*}} = \sum\limits_{j = 1}^{\infty}{\int g_{j,2}}\).
Másrészt \(g_{j,1} = g_{j} + g_{j,2} \geq 0\), \(g_{j,1} \in C_{1}\).
\(\int{g_{j,1} = {\int g_{j}} + {\int g_{j,2}}}\), így
\({\sum\limits_{j = 1}^{\infty}{{\int g_{j,1}} = {\sum\limits_{j = 1}^{\infty}{\int g_{j}}}}} + {\sum\limits_{j = 1}^{\infty}{\int g_{j,2}}}\),
ezért a \(g_{j,1}\) tagokból álló sorra is alkalmazható a bizonyítás
első része
\(\left. \Rightarrow\sum\limits_{j = 1}^{\infty}g_{j,1}\left( x \right): = g_{1}^{*}\left( x \right) \right.\)
konvergens majdnem minden \(x\)-re, \(g_{1}^{*} \in C_{1}\),
\({\int g_{1}^{*}} = \sum\limits_{j = 1}^{\infty}{\int g_{j,1}}\), így
\(\sum\limits_{j = 1}^{\infty}g_{j} = \sum\limits_{j = 1}^{\infty}\left( {g_{j,1} - g_{j,2}} \right) = g_{1}^{*} - g_{2}^{*} = :f\)
konvergens majdnem minden \(x\)-re és \(f \in C_{2}\), mert
\(g_{1}^{*} \in C_{1}\) és \(g_{2}^{*} \in C_{1}\); továbbá
\({\int f} = {\sum\limits_{j = 1}^{\infty}{\int g_{j}}}\).

\end{bizonyitas}

\textbf{Következmények} (a vizsgán a tételek következményei legalább oly
fontosak, mint a bizonyítások):

\begin{enumerate}
\def\labelenumi{\arabic{enumi}.}
\tightlist
\item
  Tfh \(f_{j} \in C_{2}\), \(\left( f_{j} \right)\) monoton nő,
  \(\lim\left( f_{j} \right) = f\) majdnem mindenütt, ahol
  \(f \in C_{2}\), ekkor \({\int f} = \lim{\int f_{j}}\). Ugyanis a
  feltevésekből következik, hogy
  \(\left. f_{j} \leq f\Rightarrow{\int{f_{j} \leq {\int f}}} \right.\),
  így a Beppo-Levi tétel miatt \({\int f} = \lim{\int f_{j}}\).
\item
  Tfh \(g_{j} \in C_{2}\), \(g_{j} \geq 0\),
  \({\sum\limits_{j = 1}^{\infty}g_{j}} = f\) majdnem mindenütt, vagyis
  konvergens, ahol \(f \in C_{2}\). Ekkor
  \({\int f} = {\sum\limits_{j = 1}^{\infty}{\int g_{j}}}\). (A
  részletösszegekre alkalmazzuk az 1-t.)
\item
  Ha \(g_{j} \in C_{2}\), de nem teszem fel róluk, hogy nemnegatívak, de
  \(\left. {\sum\limits_{j = 1}^{\infty}{\int\left| g_{j} \right|}} < \infty\Rightarrow \right.\)
  majdnem minden \(x\)-re
  \({\sum\limits_{j = 1}^{\infty}{g_{j}\left( x \right)}} = f\left( x \right)\),
  vagyis a sor majdnem mindenütt konvergens, ahol \(f \in C_{2}\),
  \({\int f} = {\sum\limits_{j = 1}^{\infty}{\int g_{j}}}\).

  \begin{bizonyitas}

  Bizonyítás: \[
     - \left| {{g_j}} \right| \leqslant g_j^ -  \leqslant {g_j} \leqslant g_j^ +  \leqslant \left| {{g_j}} \right| \Rightarrow  \int {g_j^ + }  \leqslant \int {\left| {{g_j}} \right|}  \Rightarrow \mathop \sum \limits_{j = 1}^\infty  \int {g_j^ + }  < \infty  \]
  \[ - g_j^ -  \leqslant \left| {{g_j}} \right| \Rightarrow  \int {\left( { - g_j^ - } \right)}  \leqslant \int {\left| {{g_j}} \right|}  \Rightarrow \mathop \sum \limits_{j = 1}^\infty  \int {\left( { - g_j^ - } \right)}  \leqslant \mathop \sum \limits_{j = 1}^\infty  \int {\left| {{g_j}} \right|}  < \infty \]
  A Beppo Levi tétel alkalmazható \(g_{j}^{+}\), illetve a
  \(\left( {- g_{j}^{-}} \right)\) tagokból álló sorra. Tehát
  \(\exists\lim\limits_{k\rightarrow\infty}{\sum\limits_{j = 1}^{k}\left( {- g_{j}^{-}} \right)}: = - f_{-}\)
  és
  \(\exists\lim\limits_{k\rightarrow\infty}{\sum\limits_{j = 1}^{k}g_{j}^{+}}: = f_{+}\),
  így
  \[\left. g_{j} = g_{j}^{+} + g_{j}^{-}\Rightarrow{\sum\limits_{j = 1}^{\infty}g_{j}} = {\sum\limits_{j = 1}^{\infty}\left( {g_{j}^{+} + g_{j}^{-}} \right)} = {\sum\limits_{j = 1}^{\infty}g_{j}^{+}} + {\sum\limits_{j = 1}^{\infty}g_{j}^{-}} = f_{+} + f_{-}: = f \right.\],
  valamint
  \({\int f} = {\int\left( {f_{-} + f_{+}} \right)} = {\int\left( {{\sum\limits_{j = 1}^{\infty}g_{j}^{-}} + {\sum\limits_{j = 1}^{\infty}g_{j}^{+}}} \right)} = {\sum\limits_{j = 1}^{\infty}{\int g_{j}^{-}}} + {\sum\limits_{j = 1}^{\infty}{\int g_{j}^{+}}}\).

  \end{bizonyitas}
\item
  Ha \(f \in C_{2}\), \(f \geq 0\),
  \(\left. {\int f} = 0\Rightarrow f = 0 \right.\) majdnem mindenütt.

  \begin{bizonyitas}

  Bizonyítás:\\
  Alkalmazzuk a Beppo Levi tételt \(g_{j}: = f\) \(\forall j\) -re:
  \(g_{j} \geq 0\), \(g_{j} \in C_{2}\),
  \[\int {{g_j}}  = \int f  = 0 \Rightarrow \mathop \sum \limits_{j = 1}^\infty  \int {{g_j}}  = 0 \Rightarrow \mathop \sum \limits_{j = 1}^\infty  \underbrace {{g_j}\left( x \right)}_{f\left( x \right)}\]
  konvergens majdnem minden \(x\)-re
  \(\left. \Rightarrow f\left( x \right) = 0 \right.\) majdnem minden
  \(x\)-re.

  \end{bizonyitas}
\end{enumerate}

\hypertarget{lebesgue-tetel}{%
\subsubsection{Lebesgue tétel}\label{lebesgue-tetel}}

Kérdés: ha \(f_{j} \in C_{2}\), \(\lim\left( f_{j} \right) = f\) majdnem
mindenütt, akkor igaz-e, hogy
\(\left. \Rightarrow f \in C_{2} \right.\),
\({\int f} = \lim{\int f_{j}}\).\\
Válasz: általában nem, de más megszorítást alkalmazva már igen.

\begin{pelda}

Példák:

\begin{enumerate}
\def\labelenumi{\arabic{enumi}.}
\tightlist
\item
  \(f_{j}\left( t \right) = \left( {j + 1} \right)t^{j},t \in \left\lbrack 0,1 \right\rbrack,j \in {\mathbb{N}}\).
  Ekkor
  \(\lim\limits_{j\rightarrow\infty}f_{j}\left( t \right) = \left\{ \begin{matrix} 0 & {\text{ha~}t \in \left\lbrack 0,1 \right)} \\ \infty & {\text{ha~}t = 1} \\ \end{matrix} \right.\),
  vagyis \(\lim f_{j} = 0\) majdnem mindenütt a
  \(\left\lbrack 0,1 \right\rbrack\) intervallumon.
  \(\lim\limits_{j\rightarrow\infty}{\int\limits_{0}^{1}{f_{j}\left( t \right)}} = \left\lbrack t^{j + 1} \right\rbrack_{0}^{1} = 1 \neq {\int\limits_{0}^{1}{\lim\limits_{j\rightarrow\infty}f_{j}\left( t \right)}} = 0\).
\item
  \(f_{j}\left( t \right) = \left( {j + 1} \right)^{2}t^{j},t \in \left\lbrack 0,1 \right\rbrack,j \in {\mathbb{N}}\),
  ekkor megint \(\lim f_{j} = 0\) majdnem mindenütt, de
  \(\left. {\int\limits_{0}^{1}{f_{j}\left( t \right)}} = \left( {j + 1} \right)\rightarrow\infty \right.\).
\end{enumerate}

\end{pelda}

\begin{tetel}

Tétel (Lebesgue tétel):\\
Tfh \(f_{j} \in C_{2}\), \(\lim f_{j} = f\) majdnem mindenütt,
\(\exists g \in C_{2}:\left| {f_{j}\left( x \right)} \right| \leq g\left( x \right)\)
majdnem minden \(x\)-re, \(\forall j\). Ekkor \(f \in C_{2}\) és
\({\int f} = \lim{\int f_{j}}\).

\end{tetel}

\begin{bizonyitas}

Bizonyítás:\\
Jelöljük:
\(h_{j}\left( x \right): = \sup\left\{ {f_{j}\left( x \right),f_{j + 1}\left( x \right),...} \right\} = {\bigcup\limits_{k = j}^{\infty}{f_{k}\left( x \right)}}\).
Mivel
\(\lim\limits_{j\rightarrow\infty}f_{j}\left( x \right) = f\left( x \right)\)
majdnem mindenütt, ezért
\(\lim\limits_{j\rightarrow\infty}h_{j}\left( x \right) = f\left( x \right)\)
majdnem minden \(x\)-re, \(\left( h_{j} \right)\) monoton csökkenő
sorozat. Belátandó először, hogy \(h_{j} \in C_{2},\forall j\). Pl.:
\(h_{1}\left( x \right) = \sup\left\{ {f_{1}\left( x \right),f_{2}\left( x \right),...} \right\}\),
\(h_{1,k}: = \sup\left\{ {f_{1}\left( x \right),f_{2}\left( x \right),\ldots,f_{k}\left( x \right)} \right\} = {\bigcup\limits_{j = 1}^{k}f_{j}}\).
\(h_{1,k} \in C_{2}\), \(\left( h_{1,k} \right)\) növő,
\(\left. {\int h_{1,k}} \leq {\int g}\Rightarrow\lim\limits_{k\rightarrow\infty}h_{1,k} \right.\)
véges majdnem mindenütt (Beppo Levi monoton (növő) sorozatokra), így
\(\left. h_{1} = \lim\limits_{k\rightarrow\infty}h_{1,k}\Rightarrow h_{1} \in C_{2} \right.\).
\(\left( h_{j} \right) \in C_{2}\), monoton csökkenő sorozat,
\(\left. {\int{h_{j} \geq - {\int g}}}\Rightarrow\lim h_{j} = f \right.\)
integrálható (Beppo Levi monoton (csökkenő) sorozatokra), továbbá
\({\int f} = \lim{\int h_{j}}\). Észrevétel:
\(\left. f_{j} \leq h_{j}\Rightarrow{\int f_{j}} \leq {\int h_{j}} \right.\).
Most fordítva:
\(\phi_{j}\left( x \right): = \inf\left\{ {f_{j}\left( x \right),f_{j + 1}\left( x \right),...} \right\}\).
Ekkor \(\lim\phi_{j} = f\) majdnem mindenütt,
\(\left( \phi_{j} \right)\) monoton növő. Ekkor az előbbiekhez hasonló
módon belátható, hogy \(\phi_{j} \in C_{2},\forall j\).
\(\left. \phi_{j} \leq f_{j} \leq g\Rightarrow{\int{\phi_{j} \leq {\int g}}} \right.\).
Ekkor a Beppo Levi tételét alkalmazva monoton (növő) sorozatokra,
\(\lim\left( \phi_{j} \right) = f\) integrálható,
\({\int f} = \lim{\int\phi_{j}}\), \(\phi_{j} \leq f_{j} \leq h_{j}\)
\(\left. \Rightarrow{\int\phi_{j}} \leq {\int f_{j}} \leq {\int h_{j}}\Rightarrow\lim{\int f_{j}} = f \right.\).

\end{bizonyitas}

\begin{tetel}

Spec eset (kis Lebesgue tétel):\\
Tfh \(\left. \left( f_{j} \right)\rightarrow f \right.\) majdnem
mindenütt, \(\exists a > 0:\left| x \right| > a\) esetén
\(f_{j}\left( x \right) = 0\) és
\(\exists K > 0:\left| x \right| \leq a\) esetén
\(\left| {f_{j}\left( x \right)} \right| \leq K\), \(\forall j\). Ekkor
\(f \in C_{2}\) és \({\int f} = \lim{\int f_{j}}\).

\end{tetel}

\begin{bizonyitas}

Bizonyítás:\\
A
\(g\left( x \right): = \left\{ \begin{matrix} K & {\text{ha~}\left| x_{k} \right| \leq a,\forall k} \\ 0 & \text{egyébként} \\ \end{matrix} \right.\)
függvényt bevezetve az állítást visszavezettük az előző tételre.

\end{bizonyitas}

\begin{tetel}

Tétel (Fatou lemma):\\
Tfh \(f_{j} \in C_{2}\), \(\lim\left( f_{j} \right) = f\) majdnem
mindenütt, továbbá \(0 \leq f_{j}\) majdnem mindenütt, \(\int f_{j}\)
viszont felülről korlátos. Ekkor \(f \in C_{2}\),
\({\int f} \leq \liminf{\int f_{j}}\). (Bizonyítás lehetséges a Lebesgue
tétel bizonyításának gondolatmenetével.)

\end{tetel}

\begin{tetel}

Tétel:\\
Tfh \(f_{j} \in C_{2}\), \(\lim\left( f_{j} \right) = f\) majdnem
mindenütt és \(\exists g \in C_{2}:\left| f \right| < g\). Ekkor
\(f \in C_{2}\).

\end{tetel}

\begin{bizonyitas}

Bizonyítás:\\
Visszavezetjük a Lebesgue tételre.
\[g_{j}\left( x \right): = \left\{ \begin{matrix}
{f_{j}\left( x \right)} & {\text{ha~}\left| {f_{j}\left( x \right)} \right| \leq g\left( x \right)} \\
{g\left( x \right)} & {\text{ha~}f_{j}\left( x \right) > g\left( x \right)} \\
{- g\left( x \right)} & {\text{ha~}f_{j}\left( x \right) < - g\left( x \right)} \\
\end{matrix} \right.\] Ekkor \(\left| g_{j} \right| \leq g \in C_{2}\).
Mivel \(\left| f \right| \leq g\) és \(\lim\left( f_{j} \right) = f\)
majdnem mindenütt
\(\left. \Rightarrow\lim\left( g_{j} \right) = f \right.\) majdnem
mindenütt a definícióból következően. Alkalmazzuk a
\(\left( g_{j} \right)\) sorozatra a Lebesgue tételt, melyből
következik, hogy \(f \in C_{2}\).

\end{bizonyitas}

\hypertarget{merheto-fuggvenyek}{%
\subsection{Mérhető függvények}\label{merheto-fuggvenyek}}

\begin{definicio}

Definíció:\\
Egy \(\left. f:{\mathbb{R}}^{n}\rightarrow{\mathbb{R}} \right.\)
függvényt (Lebesgue szerint) mérhetőnek nevezünk, ha előállítható
lépcsős függvények konvergens sorozatának határértékeként majdnem
mindenütt.

\end{definicio}

\begin{allitas}

Állítás:\\
Ha \(\left. f \in C_{2}\Rightarrow f \right.\) mérhető. Ezt láttuk
korábbról már. \(C_{2}\) osztály tárgyalása során.)

\end{allitas}

\begin{allitas}

Állítás:\\
Ha \(f\) mérhető és
\(\left. \exists g \in C_{2}:\left| f \right| \leq g\Rightarrow f \in C_{2} \right.\).
Ez következik az előző tételből.

\end{allitas}

\begin{pelda}

Példa:\\
Egy mérhető, de nem integrálható függvény az alábbi:
\(f\left( x \right) = 1,\forall x \in {\mathbb{R}}^{n}\),
\(f_{j}\left( x \right) = \left\{ \begin{matrix} 1 & {\text{ha~}\left| x_{k} \right| \leq j,\forall k = 1,2,...,n} \\ 0 & \text{egyébként} \\ \end{matrix} \right.\).
Ekkor \(f_{j} \in C_{2}\) lépcsős függvény,
\(\left. \lim f_{j}\left( x \right) = f\left( x \right),\forall x\Rightarrow f \right.\)
mérhető, de \(\left. {\int f_{j}}\rightarrow\infty \right.\), vagyis
\(f\) nem integrálható, ugyanis \(\left( f_{j} \right)\) monoton nő,
ezért ha \(f\) integrálható lenne, akkor Beppo Levi miatt
\({\int f} = \lim{\int f_{j}} = \infty\) lenne.

\end{pelda}

\begin{allitas}

Állítás:\\
Ha \(f\), \(g\) mérhetőek, akkor

\begin{enumerate}
\def\labelenumi{\arabic{enumi}.}
\tightlist
\item
  \(f + g\) is mérhető,
\item
  \(f \cdot g\) is mérhető,
\item
  \(\frac{f}{g}\) is mérhető, ha \(g \neq 0\) majdnem mindenütt.
\end{enumerate}

\end{allitas}

\begin{bizonyitas}

Bizonyítás:

\begin{enumerate}
\def\labelenumi{\arabic{enumi}.}
\tightlist
\item
  \(f = \lim\left( f_{j} \right)\) majdnem mindenütt, \(f_{j}\) lépcsős
  függvény, \(g = \lim\left( g_{j} \right)\) majdnem mindenütt,
  \(g_{j}\) lépcsős függvény,
  \(f + g = \lim\left( {f_{j} + g_{j}} \right)\) majdnem mindenütt,
  \(f_{j} + g_{j}\) is lépcsős függvény.
\item
  \(\frac{1}{g}\) -re látjuk be, mellyel a 3. állítás 2-ból igazolható:
  \(g = \lim\left( g_{j} \right)\) majdnem mindenütt,
  \(h_{j}: = \left\{ \begin{matrix} \frac{1}{g\left( x \right)} & {\text{ha~}g_{j}\left( x \right) \neq 0} \\ 0 & {\text{ha~}g_{j}\left( x \right) = 0} \\ \end{matrix} \right.\),
  melyből következik, hogy \(\lim\left( h_{j} \right) = \frac{1}{g}\)
  majdnem mindenütt.
\end{enumerate}

\end{bizonyitas}

\begin{ajanlo}

\begin{ajanlofig}

\href{https://xkcd.com}{\includegraphics[width=5.20833in,height=2.82292in]{wikipedian_protester.png}}

\end{ajanlofig}

Text

\end{ajanlo}

\begin{tetel}

Tétel:\\
Tfh \(f_{j}\) mérhető, \(\lim\left( f_{j} \right) = f\) majdnem
mindenütt \(\left. \Rightarrow f \right.\) mérhető.

\end{tetel}

\begin{bizonyitas}

Bizonyítás:\\
Legyen \(\left. g:{\mathbb{R}}^{n}\rightarrow{\mathbb{R}} \right.\)
olyan függvény, hogy \(g\left( x \right) > 0,\forall x\) és
\(g \in C_{2}\). Értelmezzük a \(h_{j}\) függvényeket az alábbiak
szerint: \(h_{j}: = \frac{f_{j}g}{\left| f_{j} \right| + g}\), \(h_{j}\)
mérhető és
\(\left. \left| h_{j} \right| \leq g\Rightarrow h_{j} \in C_{2} \right.\).
\(\lim\left( h_{j} \right) = \frac{fg}{\left| f \right| + g}: = h\)
majdnem mindenütt. Alkalmazzuk a Lebesgue tételt a
\(\left( h_{j} \right)\) sorozatra!
\(h = \frac{fg}{\left| f \right| + g} \in C_{2}\), így a fenti
összefüggés szerint
\(\left. \frac{fg}{\left| f \right| + g} = h\Rightarrow fg - \left| f \right|h = gh\Leftrightarrow fg - f\left| h \right| = gh \right.\)
mert \({sgn}h = {sgn}f\), így \(f = \frac{gh}{g - \left| h \right|}\),
ez pedig mérhető, mert a számláló mérhető, ugyanis \(g\) és \(h\) is
mérhetők, és mert a nevező is mérhető, továbbá
\(g - \left| h \right| > 0\), ugyanis
\(\left| h \right| = \frac{\left| f \right|g}{\left| f \right| + g} < g\).

\end{bizonyitas}

\begin{tetel}

Tétel:\\
Tfh
\(\left. f_{1},f_{2},...,f_{r}:{\mathbb{R}}^{n}\rightarrow{\mathbb{R}} \right.\)
mérhető, \(\left. g:{\mathbb{R}}^{r}\rightarrow{\mathbb{R}} \right.\)
folytonos! Ekkor mérhető a
\(\left. h: = g \circ \left( {f_{1},f_{2},...,f_{r}} \right):{\mathbb{R}}^{n}\rightarrow{\mathbb{R}} \right.\).

\end{tetel}

\begin{bizonyitas}

Bizonyítás:\\
\(f_{k}\) mérhető
\(\left. \Rightarrow\exists\left( \phi_{k,j} \right)_{j \in {\mathbb{N}}} \right.\)
lépcsős függvénysorozat, hogy
\(\left. \left( \phi_{k,j} \right)_{j \in {\mathbb{N}}}\rightarrow f_{k} \right.\)
majdnem mindenütt.
\(h_{j}: = g \circ \left( {\phi_{1,j},\phi_{2,j},...,\phi_{r,j}} \right)\)
véges sok intervallumon állandó. \(g\left( 0,0,...,0 \right) \neq 0\)
esetén \(h_{j}\) helyett vesszük a
\(h_{j}^{\ast}\left( x \right) = \left\{ \begin{matrix} {h_{j}\left( x \right)} & {\text{ha~}\left| x \right| \leq j} \\ 0 & {\text{ha~}\left| x \right| > j} \\ \end{matrix} \right.\)
függvényt. Mivel
\(\left. h_{j}\left( x \right)\rightarrow h\left( x \right) \right.\)
minden \(x\)-re
\(\left. \Rightarrow h_{j}^{\ast}\left( x \right)\rightarrow h\left( x \right) \right.\)
majdnem minden \(x\)-re, \(h_{j}^{*}\) lépcsős függvény.

\end{bizonyitas}

\hypertarget{merheto-halmazok-mertek}{%
\subsection{Mérhető halmazok, mérték}\label{merheto-halmazok-mertek}}

\begin{definicio}

Definíció:\\
Legyen \(A \subset {\mathbb{R}}^{n}\) halmaz. Az \(A\) halmaz
karakterisztikus függvényének nevezzük:
\(\chi_{A}\left( x \right): = \left\{ \begin{matrix} 1 & {x \in A} \\ 0 & {x \in {\mathbb{R}}^{n}\backslash A} \\ \end{matrix} \right.\).
Látható, hogy ekkor \(\chi_{A}\left( x \right) \geq 0\).

\end{definicio}

\begin{definicio}

Definíció:\\
Egy \(A \subset {\mathbb{R}}^{n}\) halmazt mérhetőnek nevezünk, ha
\(\chi_{A}\left( x \right)\) mérhető függvény. Ekkor az \(A\) halmaz
mértékét így értelmezzük:
\(\lambda\left( A \right): = \left\{ \begin{matrix} {\int_{{\mathbb{R}}^{n}}\chi_{A}} & {\chi_{A} \in C_{2}} \\ \infty & {\chi_{A} \notin C_{2}} \\ \end{matrix} \right.\)
(korábban lehagytuk, hogy milyen halmazon integrálunk, mert egyértelmű
volt). Láthatjuk, hogy \(\lambda\left( A \right) \geq 0\).

\end{definicio}

\begin{allitas}

Állítás:\\
Két mérhető halmaz különbsége, véges és megszámlálhatóan végtelen sok
mérhető halmaz uniója és metszete is mérhető.

\end{allitas}

\begin{bizonyitas}

Bizonyítás:\\
\(\chi_{A\backslash B} = \chi_{A} - \chi_{A}\chi_{B}\) mérhető függvény.
\(\chi_{\underset{j = 1}{\overset{k}{\cup}}A_{j}} = {\bigcup\limits_{j = 1}^{k}\chi_{A_{j}}}\),
ahol \(\chi_{A_{j}}\) mérhető függvények (felső burkoló).
\(\chi_{\underset{j = 1}{\overset{\infty}{\cup}}A_{j}} = \lim\limits_{k\rightarrow\infty}{\bigcup\limits_{j = 1}^{k}\chi_{A_{j}}}\)

\end{bizonyitas}

\begin{allitas}

Állítás:\\
Egy \(A \subset {\mathbb{R}}^{n}\) nullmértékű
\(\left. \Leftrightarrow\lambda\left( A \right) = 0 \right.\), azaz
\({\int_{{\mathbb{R}}^{n}}\chi_{A}} = 0\).

\end{allitas}

\begin{bizonyitas}

Bizonyítás:\\
\(\Rightarrow\) irányba: ha \(A\) nullmértékű,
\(\lambda\left( A \right) = {\int_{{\mathbb{R}}^{n}}\chi_{A}} = 0\) mert
\(\chi_{A} = 0\) majdnem mindenütt, ha \(A\) nullmértékű.\\
\(\Leftarrow\) irányba: ha
\(\lambda\left( A \right) = {\int_{{\mathbb{R}}^{n}}\chi_{A}} = 0\),
\(\left. \chi_{A} \geq 0\Rightarrow\chi_{A} = 0 \right.\) majdnem
mindenütt \(\left. \Rightarrow A \right.\) nullmértékű.

\end{bizonyitas}

\begin{tetel}

Tétel:\\
Ha \(A = {\bigcup\limits_{j = 1}^{k}A_{j}}\) és \(A_{j}\) mérhetők,
páronként diszjunktak, akkor
\(\lambda\left( A \right) = \lambda\left( {\underset{j = 1}{\overset{k}{\cup}}A_{j}} \right) = {\sum\limits_{j = 1}^{k}{\lambda\left( A_{j} \right)}}\).
Ezt úgy mondjuk, hogy a mérték additív halmazfüggvény.

\end{tetel}

\begin{bizonyitas}

Bizonyítás:\\
Ugyanis ha a fentiek teljesülnek, akkor
\[{\chi _A} = {\chi _{\bigcup\limits_{j = 1}^k {{A_j}} }} = \sum\limits_{j = 1}^k {{\chi _{{A_j}}}}  \Rightarrow \underbrace {\int_{{\mathbb{R}^n}} {{\chi _A}} }_{\lambda \left( A \right)} = \sum\limits_{j = 1}^k {\underbrace {\int_{{\mathbb{R}^n}} {{\chi _{{A_j}}}} }_{\lambda \left( {{A_j}} \right)}}. \]

\end{bizonyitas}

\begin{tetel}

Tétel:\\
Ha \(A = {\bigcup\limits_{j = 1}^{\infty}A_{j}}\), \(A_{j}\) mérhetők,
páronként diszjunktak, akkor
\(\lambda\left( A \right) = \lambda\left( {\underset{j = 1}{\overset{k}{\cup}}A_{j}} \right) = {\sum\limits_{j = 1}^{\infty}{\lambda\left( A_{j} \right)}}\).
Ezt úgy mondjuk, hogy a mérték \(\sigma\) additív.

\end{tetel}

\begin{bizonyitas}

Bizonyítás:\\
Csak vázolva:
\(\chi_{\underset{j = 1}{\overset{\infty}{\cup}}A_{j}} = {\sum\limits_{j = 1}^{\infty}\chi_{A_{j}}}\),
most pedig a Beppo Levi tételt alkalmazzuk.

\end{bizonyitas}

\hypertarget{integralas-merheto-halmazokon}{%
\subsubsection{Integrálás mérhető
halmazokon}\label{integralas-merheto-halmazokon}}

Eddig \(\left. f:{\mathbb{R}}^{n}\rightarrow{\mathbb{R}} \right.\)
függvények (Lebesgue) integrálját értelmeztük.

\begin{definicio}

Definíció:\\
Legyen \(A \subset {\mathbb{R}}^{n}\) mérhető halmaz,
\(\left. f:A\rightarrow{\mathbb{R}} \right.\) függvény. Legyen
\[\widetilde{f}\left( x \right): = \left\{ \begin{matrix}
{f\left( x \right)} & {x \in A} \\
0 & {x \in {\mathbb{R}}^{n}\backslash A.} \\
\end{matrix} \right.\] Ha
\(\left. \widetilde{f}:{\mathbb{R}}^{n}\rightarrow{\mathbb{R}} \right.\)
függvény integrálható, akkor azt mondjuk, hogy az
\(\left. f:A\rightarrow{\mathbb{R}} \right.\) függvény integrálható és
\({\int_{A}f} = {\int_{{\mathbb{R}}^{n}}\widetilde{f}}\).

\end{definicio}

\begin{megjegyzes}

Megjegyzés:

\begin{itemize}
\tightlist
\item
  Ha \(\left. g:{\mathbb{R}}^{n}\rightarrow{\mathbb{R}} \right.\)
  függvény integrálható és \(A \subset {\mathbb{R}}^{n}\) mérhető, akkor
  a
  \(\left. h: = \left. g \right|_{A}:A\rightarrow{\mathbb{R}} \right.\)
  integrálható, ugyanis \[\widetilde{h}: = \left\{ \begin{matrix}
  {h\left( x \right)} & {x \in A} \\
  0 & {x \in {\mathbb{R}}^{n}\backslash A,} \\
  \end{matrix} \right.\] ekkor \(\widetilde{h}: = g \cdot \chi_{A}\)
  mérhető, továbbá
  \(\left| \widetilde{h} \right| \leq \left| g \right|\) integrálható
  \(\left. \Rightarrow\widetilde{h} \right.\) is integrálható.
\item
  Ha \(\left. f:A\rightarrow{\mathbb{R}}^{n} \right.\) integrálható,
  \(B\) mérhető,
  \(\left. B \subset A\Rightarrow\left. f \right|_{B} \right.\) is
  integrálható.
\end{itemize}

\end{megjegyzes}

\hypertarget{a-lebesgue-es-riemann-integral-kapcsolata}{%
\paragraph{A Lebesgue és Riemann integrál
kapcsolata}\label{a-lebesgue-es-riemann-integral-kapcsolata}}

Legyen
\(\left. f:\left\lbrack {a,b} \right\rbrack\rightarrow{\mathbb{R}} \right.\)
korlátos függvény.

\begin{tetel}

Tétel:\\
Ha \(f\) egy Lebesgue szerint nullmértékű halmaz kivételével folytonos,
akkor \(f\) függvény Riemann és Lebesgue szerint is integrálható, és a
kétféle integrál egyenlő.

Ábra: ide kell betenni a a\_d\_b1\_b2.eps filet.

\end{tetel}

\begin{bizonyitas}

Bizonyítás:\\
Először belátjuk, hogy az \(f\) Lebesgue integrálható (sőt,
\(f \in C_{1}\)).

\[\phi_{1}: = \left\{ \begin{matrix}
{\inf\left\{ {f\left( x \right):a \leq x \leq \frac{a + b}{2}} \right\}} & {\text{ha~}x \in \left\lbrack {a,\frac{a + b}{2}} \right\rbrack} \\
{\inf\left\{ {f\left( x \right):\frac{a + b}{2} \leq x \leq b} \right\}} & {\text{ha~}x \in \left( {\frac{a + b}{2},b} \right\rbrack} \\
\end{matrix} \right.\] \[\phi_{2}: = \begin{cases}
{\inf\left\{ {f\left( x \right):a \leq x \leq d_{1}} \right\}} & {\text{ha~}x \in \left\lbrack {a,d_{1}} \right\rbrack} \\
{\inf\left\{ {f\left( x \right):d_{1} < x \leq c_{1}} \right\}} & {\text{ha~}x \in \left( {d_{1},c_{1}} \right\rbrack} \\
 \vdots & \vdots \\
{\inf\left\{ {f\left( x \right):d_{2} < x \leq b} \right\}} & {\text{ha~}x \in \left( {d_{2},b} \right\rbrack} \\
\end{cases}\]\\
... és így tovább (vagyis az egyes intervallumokat mindig felezzük),
valamint \(\phi_{k}\left( x \right): = 0\) ha
\(x \notin \left\lbrack {a,b} \right\rbrack,\forall k\). Ekkor
\(\phi_{k}\) -k lépcsős függvények, \(\left( \phi_{k} \right)\) monoton
növő. Mivel \(f\) folytonos egy nullmértékű halmaz kivételével, ezért
\(\left. \phi_{k}\left( x \right)\rightarrow f\left( x \right) \right.\)
majdnem mindenütt (ahol \(f\) folytonos).
\({\int_{\text{L}}\phi_{k}} \leq \left( {b - a} \right)M\), ahol \(M\)
olyan szám, amelyre \(\left| {f\left( x \right)} \right| \leq M\), ezért
\(f \in C_{1}\), \({\int_{\text{L}}f} = \lim{\int_{\text{L}}\phi_{k}}\)
(Lebesgue integrál).

Az \(f\) függvény egy Riemann féle felső összege \(\int_{L}\psi_{k}\),
ahol \[\psi_{1}\left( x \right): = \left\{ \begin{matrix}
{\sup\left\{ {f\left( x \right):a \leq x \leq \frac{a + b}{2}} \right\}} & {\text{ha}~x \in \left\lbrack {a,\frac{a + b}{2}} \right\rbrack} \\
{\sup\left\{ {f\left( x \right):\frac{a + b}{2} < x \leq b} \right\}} & {\text{ha}~x \in {\left( {\frac{a + b}{2},b} \right\rbrack.}} \\
\end{matrix} \right.\]\\
\(\psi_{k}\) -k lépcsős függvények, monoton csökkenők,
\(\left. \left( \psi_{k} \right)\rightarrow f \right.\) majdnem
mindenütt.
\(\left. {\int_{\text{R}}\psi_{k}} \geq - \left( {b - a} \right)M\Rightarrow - f \in C_{1} \right.\),
\({\int_{\text{L}}f} = \lim{\int_{\text{R}}\psi_{k}}\) (ahol előbbi a
Lebesgue integrál, utóbbi a Riemann féle integrál felső összege),
\(\lim{\int_{\text{L}}\phi_{k}} = {\int_{\text{L}}f}\),
\(\left. \lim{\int_{\text{R}}\psi_{k}} = {\int_{\text{R}}f}\Rightarrow f \right.\)
Reimann és Lebesgue integrálható, és az integrálok értéke megegyezik.

\end{bizonyitas}

\begin{tetel}

Tétel:\\
Ha
\(\left. f:\left\lbrack {a,b} \right\rbrack\rightarrow{\mathbb{R}} \right.\)
korlátos függvény Riemann szerint integrálható
\(\left. \Rightarrow f \right.\) folytonos majdnem mindenütt.
(bizonyítás nélkül)

\end{tetel}

\begin{megjegyzes}

Megjegyzés:\\
Ha egy \(f\) függvény Riemann szerint improprius integrálja konvergens
\(\Rightarrow f\) Lebesgue integrálható.

\end{megjegyzes}

\begin{pelda}

Példa:\\
A \(\left\lbrack {0,\infty} \right)\) intervallumon értelmezzük az \(f\)
függvényt: \(f\left( x \right): = \left( {- 1} \right)^{j}\frac{1}{j}\)
ha \(j - 1 \leq x < j\), \(j \in {\mathbb{N}}\).
\(\int\limits_{0}^{\infty}{f\left( x \right)dx}\) improprius integrálja
konvergens, mert
\(\sum\limits_{j = 1}^{\infty}{\left( {- 1} \right)^{j}\frac{1}{j}}\)
konvergens.
\(\int\limits_{0}^{\infty}{\left| {f\left( x \right)} \right|dx}\)
divergál, mert \(\sum\limits_{j = 1}^{\infty}\frac{1}{j}\) divergál. Ha
\(f\) Lebesgue szerint integrálható
\(\left. \Rightarrow\left| f \right| \right.\) is integrálható Lebesgue
szerint. Tehát a fenti \(f\) függvény improprius integrálja konvergens,
de nem Lebesgue-integrálható.

\textbf{Másik példa}:
\(f\left( x \right): = \left\{ \begin{matrix} 1 & {x \in {\mathbb{R}}\backslash{\mathbb{Q}}} \\ 0 & {x \in {\mathbb{Q}}} \\ \end{matrix} \right.\).
Ekkor \(f\) Lebesgue szerint integrálható \(\mathbb{Q}\) egy nullmértékű
halmaz), de Riemann szerint nem integrálható.

\end{pelda}

\begin{tetel}

Tétel (Fubini tétel) (bizonyítás nélkül):\\
Tfh \(\left. f:{\mathbb{R}}^{2}\rightarrow{\mathbb{R}} \right.\) képező,
integrálható függvény. Ekkor majdnem minden \(x \in {\mathbb{R}}\)
esetén \(\left. y\mapsto f\left( {x,y} \right) \right.\) integrálható
\(\mathbb{R}\) -en, továbbá
\(\left. x\mapsto{\int{f\left( {x,y} \right)dy}} \right.\) is
integrálható \(\mathbb{R}\) -en és
\({\int_{{\mathbb{R}}^{2}}f} = {\int_{\mathbb{R}}{\left\lbrack {\int_{\mathbb{R}}{f\left( {x,y} \right)dy}} \right\rbrack dx}} = {\int_{\mathbb{R}}{\left\lbrack {\int_{\mathbb{R}}{f\left( {x,y} \right)dx}} \right\rbrack dy}}\).
Ha \(f\) nemnegatív és mérhető, akkor a Fubini tétel mindig érvényes
(ilyenkor az integrál \(\infty\) is lehet).

\end{tetel}

\hypertarget{az-l2left-m-right-fuggvenyter}{%
\subsubsection{\texorpdfstring{Az \(L^{2}\left( M \right)\)
függvénytér}{Az L\^{}\{2\}\textbackslash{}left( M \textbackslash{}right) függvénytér}}\label{az-l2left-m-right-fuggvenyter}}

Jelöljük: legyen \(M \subset {\mathbb{R}}^{n}\) -beli mérhető halmaz,
ekkor \(L^{2}\left( M \right)\) jelölje az
\(\left. f:M\rightarrow{\mathbb{R}} \right.\) olyan mérhető függvények
összességét, melyekre a függvénynek az abszolútérték négyzete
integrálható (ez a témakör nagyon fontos a fizikában is).

\begin{allitas}

Állítás:\\
Ez az \(L^{2}\left( M \right)\) vektortér a szokásos műveletekkel.

\end{allitas}

\begin{bizonyitas}

Bizonyítás:\\
Tfh \(\left. f,g \in L^{2}\left( M \right)\Rightarrow f,g \right.\)
mérhető!

\begin{itemize}
\tightlist
\item
  \(\left. \Rightarrow f + g \in L^{2}\left( M \right) \right.\), azaz
  az összeadás nem visz ki a halmazból. Láttuk már, hogy ekkor \(f + g\)
  is mérhető, ha \(f\) és \(g\) mérhetőek. Továbbá
  \(\underbrace {{{\left| {f + g} \right|}^2}}_{{\text{mérhető}}} \leqslant {\left( {\left| f \right| + \left| g \right|} \right)^2} \leqslant 2\left( {{{\left| f \right|}^2} + {{\left| g \right|}^2}} \right)\).
  Ez viszont integrálható, így \(\left| {f + g} \right|^{2}\) is
  integrálható.
\item
  Skalárral való szorzás:
  \(\left. f \in L^{2}\left( M \right)\Rightarrow\lambda f \right.\)
  mérhető, továbbá
  \(\left| {\lambda f} \right|^{2} = \left| \lambda \right|^{2}\left| f \right|^{2}\)
  integrálható, így \(\lambda f \in L^{2}\left( M \right)\).
\end{itemize}

\end{bizonyitas}

\begin{allitas}

Állítás:\\
\(f,g \in L^{2}\left( M \right)\) integrálható.

\end{allitas}

\begin{bizonyitas}

Bizonyítás:\\
\(f \cdot g\) mérhető (mérhető függvények szorzata mérhető, korábbról
láttuk),
\(\left| {f \cdot g} \right| = \left| f \right| \cdot \left| g \right| \leqslant \frac{1}{2}\underbrace {\left( {{{\left| f \right|}^2} + {{\left| g \right|}^2}} \right)}_{{\text{integrálható}}}\).

\end{bizonyitas}

\begin{definicio}

Definíció:\\
Legyen \(f,g \in L^{2}\left( M \right)\)! Értelmezzük a két függvényen
az alábbi művelet:
\(\left\langle {f,g} \right\rangle: = {\int_{M}{f \cdot g}}\).

\end{definicio}

\begin{allitas}

Állítás\\
\(L^{2}\left( M \right)\) a fenti művelettel valós euklideszi tér, ahol
a skalárszorzat a fent jelölt művelet.

\end{allitas}

\begin{bizonyitas}

Bizonyítás:\\
\(L^{2}\left( M \right)\) valós vektortér, a fent jelölt szorzás művelet
skalárszorzás, ugyanis teljesíti:

\begin{itemize}
\tightlist
\item
  \(\left\langle {f_{1} + f_{2},g} \right\rangle = \left\langle {f_{1},g} \right\rangle + \left\langle {f_{2},g} \right\rangle\)
\item
  \(\left\langle {f,g} \right\rangle = \left\langle {g,f} \right\rangle\)
\item
  \(\left\langle {\lambda f,g} \right\rangle = \lambda\left\langle {f,g} \right\rangle\)
\item
  \(\left\langle {f,f} \right\rangle = {\int_{M}\left| f \right|^{2}} \geq 0\)
  és
  \(\left. \left\langle {f,f} \right\rangle = 0\Leftrightarrow f = 0 \right.\)
  majdnem mindenütt.
\end{itemize}

\end{bizonyitas}

Az \(L^{2}\left( M \right)\) térben a norma:
\(\left\| f \right\| = \sqrt{\left\langle {f,f} \right\rangle} = \sqrt{\int_{M}\left| f \right|^{2}}\).

\begin{megjegyzes}

Megjegyzés:\\
Itt is igaz a Cauchy-Schwarz egyenlőtlenség, vagyis
\(\left| {\int_{M}{f \cdot g}} \right| = \left| \left\langle {f,g} \right\rangle \right| \leq \left\| f \right\| \cdot \left\| g \right\| = \sqrt{\int_{M}\left| f^{2} \right|} \cdot \sqrt{\int_{M}\left| g^{2} \right|}\).

\end{megjegyzes}

\begin{definicio}

Definíció:\\
Hilbert térnek a teljes euklideszi teret nevezzük.

\end{definicio}

\begin{tetel}

Riesz-Fischer-tétel (bizonyítás nélkül):\\
Az \(L^{2}\left( M \right)\) tér teljes, vagyis
\(L^{2}\left( M \right)\) tér Hilbert tér.

\end{tetel}

\hypertarget{az-lpleft-m-right-fuggvenyter}{%
\subsubsection{\texorpdfstring{Az \(L^{p}\left( M \right)\)
függvénytér}{Az L\^{}\{p\}\textbackslash{}left( M \textbackslash{}right) függvénytér}}\label{az-lpleft-m-right-fuggvenyter}}

Jelölés: legyen \(1 \leq p < \infty\), \(M \subset {\mathbb{R}}^{n}\)
mérhető halmaz. Jelölje \(L^{p}\left( M \right)\) az olyan
\(\left. f:M\rightarrow{\mathbb{R}} \right.\) mérhető függvények
összességét, amelyekre \(\left| f \right|^{p}\) integrálható \(M\)-n.

\begin{allitas}

Állítás:\\
Az \(L^{p}\left( M \right)\) vektortér a szokásos műveletekkel.

\end{allitas}

\begin{bizonyitas}

Bizonyítás:\\
\(\left. f,g \in L^{p}\left( M \right)\Rightarrow f + g \right.\) is
mérhető, az abszolút érték \(p\)-dik hatványa is mérhető (a folytonos
p-edik hatványfüggvény és mérhető függvény kompozíciója).
\(\left| {f + g} \right|^{p} \leq \left( {\left| f \right| + \left| g \right|} \right)^{p} \leq 2^{p - 1}\left( {\left| f \right|^{p} + \left| g \right|^{p}} \right)\)
integrálható, tehát \(f + g \in L^{p}\left( M \right)\). Ha
\(\left. f \in L^{p}\left( M \right)\Rightarrow\lambda f \in L^{p}\left( M \right) \right.\)
nyilvánvaló.

\end{bizonyitas}

\begin{definicio}

Definíció:\\
Vezessük be az \(L^{p}\left( M \right)\) vektortérben a következő
normát:
\(\left\| f \right\|: = \left\{ {\int_{M}\left| f \right|^{p}} \right\}^{1/p}\).

\end{definicio}

\begin{allitas}

Állítás:\\
Az \(L^{p}\left( M \right)\) tér a fenti művelettel, mint normával,
normált tér.

\end{allitas}

\begin{bizonyitas}

Bizonyítás:

\begin{itemize}
\tightlist
\item
  \(\left\| f \right\| \geq 0\),
  \(\left. \left\| f \right\| = 0\Leftrightarrow f = 0 \right.\) majdnem
  mindenütt.
\item
  \(\left\| {\lambda f} \right\| = \left| \lambda \right| \cdot \left\| f \right\|\).
\item
  A háromszög egyenlőtlenség bizonyításához szükséges a Hölder
  egyenlőtlenség és a Young egyenlőtlenség.
\end{itemize}

\end{bizonyitas}

\begin{allitas}

Állítás (Young):\\
Legyen \(1 < p < \infty\), \(\frac{1}{p} + \frac{1}{q} = 1\)
\(q = \frac{p}{p - 1}\)). Ekkor \(\forall a,b \geq 0\) számokra:
\(ab \leq \frac{a^{p}}{p} + \frac{b^{q}}{q}\).

\end{allitas}

\begin{bizonyitas}

Bizonyítás:\\
A bizonyítandó egyenlőtlenség ekvivalens:
\(ab^{1 - q} \leq \frac{a^{p}b^{- q}}{p} + \frac{1}{q}\), feltéve, hogy
\(b \neq 0\). (A \(b = 0\) eset triviális.) \(c: = ab^{1 - q}\)
jelöléssel \(c^{p} = a^{p}b^{{({1 - q})}p} = a^{p}b^{- q}\). Vagyis az
állítás: \(c \leq \frac{c^{p}}{p} + \frac{1}{q}\).
\(g\left( c \right): = \frac{c^{p}}{p} - c + \frac{1}{q}\),
\(g\left( 1 \right) = 0\), \(g'\left( c \right) = c^{p - 1} - 1\). Ez
kisebb 0-nál, ha \(c < 1\), és nagyobb nullánál, ha \(c > 1\) (tehát
\(c = 1\) -ben minimuma van), tehát \(g\left( c \right) \geq 0\).

\end{bizonyitas}

\begin{tetel}

Tétel (Hölder-egyenlőtlenség):\\
Tfh \(f \in L^{p}\left( M \right),g \in L^{q}\left( M \right)\),
\(1 < p < \infty\), \(1 < q < \infty\),
\(\left. \frac{1}{p} + \frac{1}{q} = 1\Rightarrow f \cdot g \right.\)
integrálható, és
\(\left| {\int_{M}{f \cdot g}} \right| \leq {\int_{M}{\left| f \right| \cdot \left| g \right|}} \leq \left\| f \right\|_{L^{p}{(M)}} \cdot \left\| g \right\|_{L^{q}{(M)}}\).

\end{tetel}

\begin{bizonyitas}

Bizonyítás:\\
Alkalmazzuk a Young egyenlőséget:
\(a: = \frac{\left| {f\left( x \right)} \right|}{\left\| f \right\|}\),
\(b: = \frac{\left| {g\left( x \right)} \right|}{\left\| g \right\|}\).
\(\frac{\left| {f\left( x \right)} \right|}{\left\| f \right\|_{L^{p}{(M)}}} \cdot \frac{\left| {g\left( x \right)} \right|}{\left\| g \right\|_{L^{q}{(M)}}} \leq \frac{1}{p}\frac{\left| {f\left( x \right)} \right|^{p}}{\left\| f \right\|^{p}} + \frac{1}{q}\frac{\left| {g\left( x \right)} \right|^{q}}{\left\| g \right\|^{q}}\).
Integrálva mindekét oldalt \(M\)-re:
\(\frac{\int_{M}{\left| f \right| \cdot \left| g \right|}}{\left\| f \right\| \cdot \left\| g \right\|} \leq \frac{1}{p}1 + \frac{1}{q}1 = 1\).

\end{bizonyitas}

\begin{tetel}

Tétel (Minkowski-egyenlőtlenség):\\
Ha
\(\left. f,g \in L^{p}\left( M \right)\Rightarrow\left\| {f + g} \right\|_{L^{p}{(M)}} \leq \left\| f \right\|_{L^{p}{(M)}} + \left\| g \right\|_{L^{p}{(M)}} \right.\).

\end{tetel}

\begin{bizonyitas}

Bizonyítás:\\
\(p = 1\) esetére triviális.

\(p > 1\) esetén: \[\begin{aligned}
  {\left\| {f + g} \right\|^p} =  & \int_M {{{\left| {f + g} \right|}^p}}  \\ 
   =  & \int_M {{{\left| {f + g} \right|}^{p - 1}}}  \cdot \left| {f + g} \right| \\ 
   \leqslant  & \int_M {{{\left| {f + g} \right|}^{p - 1}}}  \cdot \left| f \right| + \int_M {{{\left| {f + g} \right|}^{p - 1}}}  \cdot \left| g \right| \\ 
  {\text{Hölder}}\quad  \leqslant  & {\left\{ {\int_M {{{\left| {f + g} \right|}^{\left( {p - 1} \right)q}}} } \right\}^{1/q}} \cdot {\left\{ {\int_M {{{\left| f \right|}^p}} } \right\}^{1/p}} +  \\ 
   &  + {\left\{ {\int_M {{{\left| {f + g} \right|}^{\left( {p - 1} \right)q}}} } \right\}^{1/q}} \cdot {\left\{ {\int_M {{{\left| g \right|}^p}} } \right\}^{1/p}} \\ 
   =  & {\left( {{{\left\| {f + g} \right\|}_{_{{L^p}\left( M \right)}}}} \right)^{p/q}} \cdot \left[ {{{\left\| f \right\|}_{{L^p}\left( M \right)}} + {{\left\| g \right\|}_{{L^p}\left( M \right)}}} \right], \\ 
\end{aligned} \] amelyben
\(p - \frac{p}{q} = p\left( {1 - \frac{1}{q}} \right) = p\frac{1}{p} = 1\).
Így az előbbi egyenlőtlenségből
\(\left\| {f + g} \right\|_{L^{p}{(M)}} = \left\| {f + g} \right\|_{L^{p}{(M)}}^{p - p/q} \leq \left\| f \right\|_{L^{p}{(M)}} + \left\| g \right\|_{L^{p}{(M)}}\).
Tehát \(L^{p}\left( M \right)\) tér normáltságának utolsó feltételét is
igazoltuk, azaz
\(\left\| {f + g} \right\| \leq \left\| f \right\| + \left\| g \right\|\),
tehát \(L^{p}\left( M \right)\) normált tér.

\end{bizonyitas}

\begin{tetel}

Tétel (bizonyítás nélkül):\\
\(L^{p}\left( M \right)\) teljes normált tér, azaz Banach
\(1 \leq p < \infty\)).

\end{tetel}

\end{document}
