\documentclass[]{scrartcl}
\usepackage{lmodern}
\usepackage{amssymb,amsmath}
\usepackage{ifxetex,ifluatex}
\ifnum 0\ifxetex 1\fi\ifluatex 1\fi=0 % if pdftex
  \usepackage[T1]{fontenc}
  \usepackage[utf8]{inputenc}
\else % if luatex or xelatex
  \ifxetex
    \usepackage{mathspec}
  \else
    \usepackage{fontspec}
  \fi
  \defaultfontfeatures{Ligatures=TeX,Scale=MatchLowercase}
\fi
% use upquote if available, for straight quotes in verbatim environments
\IfFileExists{upquote.sty}{\usepackage{upquote}}{}
% use microtype if available
\IfFileExists{microtype.sty}{%
\usepackage[]{microtype}
\UseMicrotypeSet[protrusion]{basicmath} % disable protrusion for tt fonts
}{}
\PassOptionsToPackage{hyphens}{url} % url is loaded by hyperref
\usepackage[unicode=true]{hyperref}
\hypersetup{
            pdftitle={Analízis I},
            pdfauthor={Simon László; Tüzes Dániel; Izsák Ferenc},
            pdfborder={0 0 0},
            breaklinks=true}
\urlstyle{same}  % don't use monospace font for urls
\usepackage{graphicx,grffile}
\makeatletter
\def\maxwidth{\ifdim\Gin@nat@width>\linewidth\linewidth\else\Gin@nat@width\fi}
\def\maxheight{\ifdim\Gin@nat@height>\textheight\textheight\else\Gin@nat@height\fi}
\makeatother
% Scale images if necessary, so that they will not overflow the page
% margins by default, and it is still possible to overwrite the defaults
% using explicit options in \includegraphics[width, height, ...]{}
\setkeys{Gin}{width=\maxwidth,height=\maxheight,keepaspectratio}
\setlength{\emergencystretch}{3em}  % prevent overfull lines
\providecommand{\tightlist}{%
  \setlength{\itemsep}{0pt}\setlength{\parskip}{0pt}}
\setcounter{secnumdepth}{0}
% Redefines (sub)paragraphs to behave more like sections
\ifx\paragraph\undefined\else
\let\oldparagraph\paragraph
\renewcommand{\paragraph}[1]{\oldparagraph{#1}\mbox{}}
\fi
\ifx\subparagraph\undefined\else
\let\oldsubparagraph\subparagraph
\renewcommand{\subparagraph}[1]{\oldsubparagraph{#1}\mbox{}}
\fi

% set default figure placement to htbp
\makeatletter
\def\fps@figure{htbp}
\makeatother

\usepackage{amsthm}
\usepackage{newunicodechar}
\newunicodechar{■}{\qed}
\newunicodechar{∎}{\qed}

\subtitle{Előadásjegyzet fizikusoknak matematikusoktól}

\renewcommand{\contentsname}{Tartalomjegyzék}

\usepackage{xcolor}
\definecolor{sotetvoros}{HTML}{900000}
\definecolor{sotetzold}{HTML}{009900}
\usepackage{mdframed}
\usepackage{tcolorbox}
\tcbuselibrary{most}

\newenvironment{definicio}{}{}
\tcolorboxenvironment{definicio}{blanker,breakable,left=5mm, right=5mm,
before skip=10pt,after skip=10pt,borderline west={1.5mm}{0pt}{black},borderline east={1.5mm}{0pt}{black}}

\newenvironment{tetel}{}{}
\tcolorboxenvironment{tetel}{blanker,breakable,left=5mm,right=5mm,
before skip=10pt,after skip=10pt,borderline west={1.5mm}{0pt}{sotetvoros},borderline east={1.5mm}{0pt}{sotetvoros}}

\newenvironment{bizonyitas}{}{}
\tcolorboxenvironment{bizonyitas}{blanker,breakable,left=5mm,right=5mm,
before skip=10pt,after skip=10pt,borderline west={1.5mm}{0pt}{sotetvoros},borderline east={1.5mm}{0pt}{sotetvoros}}

\newenvironment{allitas}{}{}
\tcolorboxenvironment{allitas}{blanker,breakable,left=5mm,right=5mm,
before skip=10pt,after skip=10pt,borderline west={0.75mm}{0pt}{sotetvoros},borderline east={0.75mm}{0pt}{sotetvoros}}

\newenvironment{pelda}{}{}
\tcolorboxenvironment{pelda}{blanker,breakable,left=5mm,right=5mm,
before skip=10pt,after skip=10pt,borderline west={1.5mm}{	0pt}{sotetzold},borderline east={1.5mm}{0pt}{sotetzold}}

\newenvironment{megjegyzes}{}{}
\tcolorboxenvironment{megjegyzes}{blanker,breakable,left=5mm,right=5mm,
before skip=10pt,after skip=10pt,borderline west={0.75mm}{0pt}{lightgray},borderline east={0.75mm}{0pt}{lightgray}}

\newenvironment{ajanlofig}{\begin{figure}\begin{center}}{
\end{center}\end{figure}}

\newenvironment{ajanlo}{\begin{samepage}}{\end{samepage}}

\title{Analízis I}
\author{Simon László \and Tüzes Dániel \and Izsák Ferenc}
\date{}

\begin{document}
\maketitle

{
\setcounter{tocdepth}{6}
\tableofcontents
}
Ajánlott irodalom:

\begin{enumerate}
\def\labelenumi{\arabic{enumi}.}
\tightlist
\item
  \href{http://www.typotex.hu/index.php?page=konyvek\&book_id=240}{Komornik
  Vilmos: Valós analízis előadások}
\item
  \href{https://notendur.hi.is/vae11/\%C3\%9Eekking/principles_of_mathematical_analysis_walter_rudin.pdf}{W.
  Rudin: Principles of Mathematical Analysis, McGraw-Hill, 1976. (angol
  nyelvű)}
\item
  \href{http://www.cs.elte.hu/\%7Esimonp/jegyzet1.pdf}{Mezei István,
  Faragó István, Simon Péter: Bevezetés az analízisbe}
\end{enumerate}

Ez a jegyzet \textbf{nem} szakirodalom és nem garantált, hogy az órai
anyagot teljesen lefedi, az előadásokra bejárni ajánlott. Az eredeti
jegyzet Simon László előadásai alapján Tüzes Dániel készítette, és
lektorálta 2009-ben Simon László, majd frissítette 2016-2017-ben Izsák
Ferenc.

Ha a jegyzetben helyesírási, tartalmi, vagy formai hibát találsz, kérlek
jelezd az előadónak vagy a
\href{mailto:tuzesdaniel@gmail.com}{\nolinkurl{tuzesdaniel@gmail.com}}
e-mail címen! Ha a jegyzet nem jelenik meg helyesen, olvasd el az
\href{http://fizweb.extra.hu/index.php?dir=kalkulus,analizis/analizisI/\&file=utmutato.html}{útmutatót},
vagy egyszerűen használd a
\href{http://www.mozilla.com/en-US/firefox/all.html}{Firefox legújabb
böngészőjét!}

A jegyzet korábbi, nem következetes jelölésétől eltérően a következőkben
törekszünk arra, hogy egy függvényt \(\left. f:X\rightarrow Y \right.\)
alakban adunk meg, akkor az azt jelenti, hogy az értelmezési tartománya
\(X\), nem pedig annak csak egy része. Ez utóbbira használjuk majd az
\(\left. f:X\rightarrowtail Y \right.\) jelölést.

\section{Metrikus tér}\label{metrikus-ter}

A korábban (középiskolában) tanultakból általánosítunk.
\({\mathbb{R}}^{n}\)-ben éltünk eddig, ahol vektor alatt ezt értettük:
\(\mathbf{v} = \left( {v_{1},v_{2}...v_{n}} \right)\) ahol
\(v_{j} \in {\mathbb{R}}\) és \(\mathbf{v} \in {\mathbb{R}}^{n}\). Ezen
vektorfogalmat fogjuk általánosítani úgy, hogy a már korábban tanult
vektorok némely tulajdonságait kiválasztjuk, s egy halmaz
(\(\mathbb{V}\)) elemeit (\(a\), \(b\) és \(c\)) akkor fogjuk
vektoroknak nevezni, ha az alább kiválaszott - és korábban
(középiskolában) már tanult - tulajdonságokat (a műveletekkel)
teljesítik.

\begin{itemize}
\tightlist
\item
  összeadás +\\
  \({\mathbb{R}}^{n}\)-ben azt mondtuk, hogy
  \[\mathbf{v} + \mathbf{u} = \left( {v_{1},v_{2}...v_{n}} \right) + \left( {u_{1},u_{2}...u_{n}} \right) = \left( {v_{1} + u_{1},v_{2} + u_{2}...v_{n} + u_{n}} \right)\]
  Ezek tulajdonságaiból az alábbiakat általánosítjuk:

  \begin{enumerate}
  \def\labelenumi{\arabic{enumi}.}
  \tightlist
  \item
    \(a + \left( {b + c} \right) = \left( {a + b} \right) + c\)
    (asszociativitás)
  \item
    \(\exists!0 \in {\mathbb{V}}:a + 0 = 0 + a = a\) (egység, semleges
    elem létezése)
  \item
    \(\forall a \in {\mathbb{V}}\exists!( - a) \in {\mathbb{V}}:a + ( - a) = 0\)
    (inverz elem létezése)
  \item
    \(a + b = b + a\) (kommutativitás)
  \end{enumerate}

  Az első három tulajdonsággal rendelkező struktúrát csoportnak, a
  negyedikkel is rendelkezőt Abel-csoportnak vagy kommutatív csoportnak
  nevezzük.
\item
  Skalárral való szorzás ·\\
  Legyen \(\lambda,\beta \in {\mathbb{R}}\)! \({\mathbb{R}}^{n}\)-ben
  azt mondtuk, hogy
  \(\lambda\mathbf{v} = \lambda\left( {v_{1},v_{2}...v_{n}} \right) = \left( {\lambda v_{1},\lambda v_{2}...\lambda v_{n}} \right)\),
  ezek tulajdonságaiból az alábbiakat általánosítjuk:

  \begin{enumerate}
  \def\labelenumi{\arabic{enumi}.}
  \tightlist
  \item
    \(\lambda\left( {a + b} \right) = \lambda a + \lambda b,\left( {\lambda + \mu} \right)a = \lambda a + \mu a\)
    (disztributivitás)
  \item
    \(\lambda\left( {\beta a} \right) = \left( {\lambda\beta} \right)a\)
  \item
    \(1a = a\)
  \end{enumerate}
\end{itemize}

\begin{definicio}

Definíció:\\
Ha egy halmazon értelmezve van az összeadás és a skalárral való szorzás
a fentiek szerint, akkor azt vektortérnek (avagy lineáris térnek)
nevezzük.

\end{definicio}

Ismert művelet volt \({\mathbb{R}}^{n}\)-ben a skaláris szorzás, ezt
értettük alatta:
\({\mathbf{v},\mathbf{u}} = {\sum\limits_{j = 1}^{n}{v_{j}u_{j}}}\).
Erre érvényesek az alábbi tulajdonságok:

\begin{itemize}
\tightlist
\item
  \({a,b + c} = {a,b} + {a,c}\)
\item
  \({a,b} = {b,a}\)
\item
  \(\lambda{a,b} = {\lambda a,b}\)
\item
  \({a,a} \geq 0\) és \(\left. {a,a} = 0\Leftrightarrow a = 0 \right.\)
\end{itemize}

\begin{definicio}

Definíció:\\
Legyen \(X\) vektortér, amelynek elemei között értelmezve van a skaláris
szorzat (két elem skaláris szorzata egy \(\mathbb{R}\)-beli szám) a
fenti tulajdonságokkal. Ekkor \(X\)-t valós euklideszi (eukleidészi)
térnek nevezzük.

\end{definicio}

\begin{pelda}

P\textbf{élda}\\
A \(\left\lbrack 0,1 \right\rbrack\) intervallumon értelmezett folytonos
függvények összessége (röviden \(C\left\lbrack 0,1 \right\rbrack\) ) a
szokásos összeadással, számmal való szorzással, ha a skaláris szorzat
definíciója: \({f,g}: = {\int_{0}^{1}{f \cdot g}}\).

\end{pelda}

\begin{definicio}

Definíció:\\
Legyen \(X\) valós euklideszi tér! Ekkor egy \(a \in X\) elem normáját
így határozhatjuk meg: \(\left\| a \parallel \right.: = \sqrt{a,a}\)

\end{definicio}

A norma tulajdonságai:

\begin{enumerate}
\def\labelenumi{\arabic{enumi}.}
\tightlist
\item
  \(\left\| a \parallel \right. \geq 0\) és
  \(\left. \left\| a \parallel \right. = 0\Leftrightarrow a = 0 \right.\)
\item
  \(\left\| {\lambda a} \parallel \right. = \left| \lambda \right| \cdot \left\| a \parallel \right.\)
\item
  \(\left\| {a + b} \parallel \right. \leq \left\| a \parallel \right. + \left\| b \parallel \right.\)
  (háromszög egyenlőtlenség), mert
  \({a + b,a + b} = {a,a} + {b,a} + {a,b} + {b,b} =\)
  \(= \left\| a \parallel \right.^{2} + \left\| b \parallel \right.^{2} + 2{a,b} \leq \left\| a \parallel \right.^{2} + \left\| b \parallel \right.^{2} + 2\left\| a \parallel \right. \cdot \left\| b \parallel \right. = \left( {\left\| a \parallel \right. + \left\| b \parallel \right.} \right)^{2}\).
  Itt felhasználtuk az ún Cauchy-Schwarz-egyenlőtlenséget, mely szerint:
\end{enumerate}

\hypertarget{cauchy-schwarz}{}
\begin{tetel}

Tétel (Cauchy-Schwarz egyenlőtlenség, CS):\\
Legyen \(X\) valós euklideszi tér! Ekkor \(\forall a,b \in X\) esetén
\(\left| {a,b} \right| \leq \left\| a \parallel \right. \cdot \left\| b \parallel \right.\).

\end{tetel}

\begin{bizonyitas}

Bizonyítás:\\
\(0 \leq {a + \lambda b,a + \lambda b} = {a,a} + {\lambda b,a} + {a,\lambda b} + {\lambda b,\lambda b} = {a,a} + 2\lambda{a,b} + \lambda^{2}{b,b}\),
ez teljesül minden \(\lambda\) értékre, így
\(4{a,b}^{2} - 4{a,a}{b,b} \leq 0\), vagyis
\[\left. \left\langle {a,b} \right\rangle^{2} \leq \left\langle {a,a} \right\rangle\left\langle {b,b} \right\rangle\Rightarrow\left| \left\langle {a,b} \right\rangle \right| \leq \sqrt{\left\langle {a,a} \right\rangle}\sqrt{\left\langle {b,b} \right\rangle} = \left\| a \right\| \cdot \left\| b \right\|. \right.\]
■

\end{bizonyitas}

\begin{definicio}

Definíció:\\
Legyen \(X\) vektortér, amelyen értelmezve van egy norma a fenti
tulajdonságokkal, ekkor \(X\)-t normált térnek nevezzük.

\end{definicio}

\begin{pelda}

Példa:\\
\(X = C\left\lbrack 0,1 \right\rbrack\), a függvény normája pedig
\(\left\| f \parallel \right.: = \sup\left| f \right|\).

\end{pelda}

Egy normált térben mindig értelmezhető az elemek \(\rho\) távolsága,
\(\rho\left( {a,b} \right): = \left\| {a - b} \parallel \right.\). A
távolság (metrika) tulajdonságai:

\begin{enumerate}
\def\labelenumi{\arabic{enumi}.}
\tightlist
\item
  \(\rho\left( {a,b} \right) \geq 0\) és
  \(\left. \rho\left( {a,b} \right) = 0\Leftrightarrow a = b \right.\)
\item
  \(\rho\left( {a,b} \right) = \rho\left( {b,a} \right)\)
\item
  \(\rho\left( {a,c} \right) \leq \rho\left( {a,b} \right) + \rho\left( {b,c} \right)\)
  (háromszög egyenlőtlenség)
\end{enumerate}

\begin{definicio}

Definíció:\\
Legyen \(X\) valamilyen halmaz és tfh értelmezve van
\(\left. \rho:X \times X\rightarrow{\mathbb{R}} \right.\) függvény
(metrika, távolság) a fenti tulajdonságokkal! Ekkor \(X\)-t metrikus
térnek nevezzük.

\end{definicio}

\subsection{Topológiai alapfogalmak a metrikus
térben}\label{topologiai-alapfogalmak-a-metrikus-terben}

\begin{definicio}

Definíció:\\
Legyen \(X\) metrikus tér! Egy \(a \in X\) pont \(r\) sugarú környezete
azon pontok összessége, amelyek \(a\)-tól \(r\)-nél kisebb távolságra
vannak:
\({B_{r}\left( a \right): = \left\{ {x \in X:\rho\left( {x,a} \right) < r} \right\}}.\)

\end{definicio}

\subsubsection{Pont és halmaz viszonya}\label{pont-es-halmaz-viszonya}

Legyen \(a \in X,M \subset X\)!

\begin{definicio}

Definíció:\\
Azt mondjuk, hogy az \(a\) pont az \(M\) halmaznak belső pontja, ha
létezik \(a\)-nak olyan \(r\) sugarú környezete, hogy
\(B_{r}\left( a \right) \subset M\). Jele:
\({a \in {int}\left( M \right)}.\)

\end{definicio}

\begin{definicio}

Definíció:\\
Az \(a\) pont az \(M\) halmaznak külső pontja, ha létezik \(a\)-nak
olyan \(r\) sugarú környezete, hogy
\(B_{r}\left( a \right) \cap M = \varnothing\). Jele:
\({a \in {ext}\left( M \right)}.\)

\end{definicio}

\begin{definicio}

Definíció:\\
Az \(a\) pont \(M\)-nek határpontja, ha \(a\) minden \(r\) sugarú
környezete esetén \(B_{r}\left( a \right) \cap M \neq \varnothing\) és
\(B_{r}\left( a \right) \cap M^{C} \neq \varnothing\). Jele:
\({a \in \partial\left( M \right) = {front}\left( M \right)}.\)

\end{definicio}

\begin{allitas}

Állítás:\\
\(\partial\left( M \right),{ext}\left( M \right),{int}\left( M \right)\)
halmazok diszjunktak, uniójuk kiadja \(X\)-et.

\end{allitas}

\begin{definicio}

Definíció:\\
Egy \(a \in X\) pontot az \(M\) halmaz torlódási pontjának nevezünk, ha
az \(a\) pont minden környezetében van \(M\)-beli, de \(a\)-tól
különböző pont, formailag: \(a\) torlódási pont, ha
\(\left\{ {B_{r}\left( a \right)\backslash\left\{ a \right\}} \right\} \cap M \neq \varnothing\).
Az \(M\) halmaz torlódási pontjainak halmazát \(M\) '-vel jelöljük.

\end{definicio}

\begin{megjegyzes}

Megjegyzés:\\
Ha az \(a\) pont \(M\)-nek torlódási pontja, akkor \(a\)-nak minden
környezete végtelen sok pontot tartalmaz az \(M\) halmazból.

\end{megjegyzes}

\begin{definicio}

Definíció:\\
Az \(M\) halmaz belső és határpontjainak összességét az \(M\) halmaz
lezárásának nevezzük, \(\overline{M} = {int}M \cup \partial M\).

\end{definicio}

\begin{pelda}

Példák:

\begin{itemize}
\tightlist
\item
  \(\left. X = {\mathbb{R}},M = \left( 0,1 \right)\Rightarrow M' = \left\lbrack 0,1 \right\rbrack \right.\),
  \(\partial M = \left\{ 0,1 \right\},{int}M = \left( 0,1 \right),\overline{M} = \left\lbrack 0,1 \right\rbrack\)
\item
  \(\left. X = {\mathbb{R}},M = {\mathbb{Z}}\Rightarrow M' = \varnothing \right.\)
  ,
  \(\partial M = {\mathbb{Z}},{int}M = \varnothing,\overline{M} = {\mathbb{Z}}\)
\item
  \(\left. X = {\mathbb{R}},M = \left\lbrack 0,1 \right\rbrack\Rightarrow M' = \left\lbrack 0,1 \right\rbrack \right.\)
  ,
  \(\partial M = \left\{ 0,1 \right\},{int}M = \left( 0,1 \right),\overline{M} = \left\lbrack 0,1 \right\rbrack\)
\end{itemize}

\end{pelda}

\subsubsection{Nyílt és zárt halmazok}\label{nyilt-es-zart-halmazok}

\begin{definicio}

Definíció:\\
Egy \(M \subset X\) halmazt nyíltnak nevezünk, ha \(\forall x \in M\)
esetén
\[\left. x \in {int}\left( M \right)\Leftrightarrow M \subset {int}\left( M \right)\Leftrightarrow M \cap \partial M = \varnothing. \right.\]

\end{definicio}

\begin{definicio}

Definíció:\\
Egy \(M\) halmazt zártnak nevezünk, ha tartalmazza az összes
határpontját \(\left. \Leftrightarrow\partial M \subset M \right.\).

\end{definicio}

\begin{pelda}

Példák:\\
Legyen \(X: = {\mathbb{R}}\) , ekkor:

\begin{itemize}
\tightlist
\item
  \(M = \left\lbrack 0,1 \right\rbrack\) zárt halmaz
\item
  \(M = \left( 0,1 \right)\) nyílt halmaz
\item
  \(M = \left( 0,1 \right\rbrack\) se nem nyílt, se nem zárt halmaz
\item
  \(M = {\mathbb{Z}}\) zárt halmaz
\end{itemize}

\end{pelda}

\begin{allitas}

Állítás:\\
Egy \(M \subset X\) halmaz zárt
\(\left. \Leftrightarrow M = \overline{M}\Leftrightarrow M' \subset M \right.\).

\end{allitas}

\begin{tetel}

Tétel:\\
Tetszőleges \(M\) halmaz esetén \({int}\left( M \right)\) és
\({ext}\left( M \right)\) nyílt halmaz.

\end{tetel}

\begin{bizonyitas}

Bizonyítás\\
(\({int}\left( M \right)\) nyílt halmaz): legyen \(a \in {int}M\). Azt
kellene megmutatni, hogy
\(\exists B_{r}\left( a \right) \subset {int}M\).
\(\left. a \in {int}\left( M \right)\Rightarrow\exists B_{R}\left( a \right) \subset M \right.\).
Legyen \(r: = R/2\), ekkor
\(B_{r}\left( a \right) \subset {int}\left( M \right)\), ugyanis ha
\(b \in B_{r}\left( a \right)\), akkor a háromszög egyenlőtlenség miatt
\(\left. B_{r}\left( b \right) \subset B_{R}\left( a \right) \subset M,b \in {int}\left( M \right)\Rightarrow B_{r}\left( a \right) \subset {int}\left( M \right) \right.\)
.■

\end{bizonyitas}

\begin{allitas}

Állítás:\\
\(\partial M,\overline{M},M'\) zárt halmazok.

\end{allitas}

\begin{tetel}

Tétel:\\
Ha \(M \subset X\) nyílt, akkor \(M^{C} = X\text{\textbackslash}M\) zárt
halmaz.

\end{tetel}

\begin{bizonyitas}

Bizonyítás:\\
Tfh \(M\) nyílt halmaz, ekkor \(\partial M \cap M = \varnothing\),
\(\partial M = \partial\left( M^{c} \right)\), ezért
\[\left. \partial M^{C} \cap M = \varnothing\Rightarrow\partial M^{C} \subset M^{C} \right.,\]
vagyis \(M^{C}\) zárt. ■

\end{bizonyitas}

\begin{tetel}

Tétel:\\
Akárhány nyílt halmaz uniója nyílt halmaz, és véges sok nyílt halmaz
metszete is nyílt.

\end{tetel}

\begin{bizonyitas}

Bizonyítás:\\
Legyenek \(M_{\gamma \in I}\) nyílt halmazok (\(I\) indexhalmaz)!
Belátjuk, hogy \(M: = {\underset{\gamma \in I}{\cup}M_{\gamma}}\) nyílt.
Legyen
\(\left. a \in M\Rightarrow\exists\gamma:a \in M_{\gamma} \right.\).
Mivel \(M_{\gamma}\) nyílt, ezért
\[\left. \exists B_{r}\left( a \right) \subset M_{\gamma}\Rightarrow B_{r}\left( a \right) \subset M \right..\]\\
Legyenek \(M_{j \in I}\) nyílt halmazok (\(I\) indexhalmaz)! Belátjuk,
hogy \(M: = {\underset{j = 1}{\overset{p}{\cap}}M_{j}}\) nyílt halmaz.
Legyen
\(\left. a \in M\Rightarrow a \in M_{j},\forall j = 1,2...p \right.\).
Mivel \(M_{j}\) nyílt, ezért
\(\exists r_{j}:B_{r_{j}}\left( a \right) \subset M_{j}\). Legyen
\(\left. r = \min\left\{ {r_{1},r_{2},..,r_{p}} \right\}\Rightarrow B_{r}\left( a \right) \subset {\underset{j = 1}{\overset{p}{\cap}}M_{j}} \right.\)
. ■

\end{bizonyitas}

\begin{tetel}

Tétel:\\
Akárhány zárt halmaz metszete zárt halmaz, és véges sok zárt halmaz
uniója is zárt.

\end{tetel}

\begin{bizonyitas}

Bizonyítás:\\
(Belátjuk, hogy metszetük zárt.) Tfh \(M_{\gamma}\) zárt! Ekkor
\(M_{\gamma}^{C}\) nyílt halmaz. Ezért
\[{\bigcap\limits_{\gamma \in I}M_{\gamma}} = \left( {\underset{\gamma \in I}{\cup}M_{\gamma}^{C}} \right)^{C}\]
zárt. Az unió esete hasonlóan bizonyítható. ■\\

\end{bizonyitas}

\begin{megjegyzes}

Megjegyzés: végtelen sok nyílt halmaz metszete általában nem nyílt, az
alaphalmaz és az üres halmaz nyílt és zárt egyszerre.

\end{megjegyzes}

\begin{ajanlo}

\begin{ajanlofig}

\href{https://xkcd.com}{\includegraphics[width=5.20833in,height=2.82292in]{wikipedian_protester.png}}

\end{ajanlofig}

\textbf{\emph{\href{https://xkcd.com/}{xkcd}}}, sometimes styled
\textbf{\emph{XKCD}},\href{https://en.wikipedia.org/wiki/Xkcd\#cite_note-aboutxkcd-3}{{[}‡
1{]}} is a \href{https://en.wikipedia.org/wiki/Webcomic}{webcomic}
created by \href{https://en.wikipedia.org/wiki/Randall_Munroe}{Randall
Munroe}. The comic's
\href{https://en.wikipedia.org/wiki/Tagline}{tagline} describes it as
``A webcomic of romance, sarcasm, math, and
language''.\href{https://en.wikipedia.org/wiki/Xkcd\#cite_note-4}{{[}‡
2{]}}\href{https://en.wikipedia.org/wiki/Xkcd\#cite_note-Boston.com-5}{{[}3{]}}
Munroe states on the comic's website that the name of the comic is not
an \href{https://en.wikipedia.org/wiki/Acronym}{acronym} but ``just a
word with no phonetic pronunciation''.

The subject matter of the comic varies from statements on life and love
to \href{https://en.wikipedia.org/wiki/Mathematical_joke}{mathematical}
and \href{https://en.wikipedia.org/wiki/Science}{scientific}
\href{https://en.wikipedia.org/wiki/In-joke}{in-jokes}. Some strips
feature simple humor or
\href{https://en.wikipedia.org/wiki/Pop-culture}{pop-culture}
references. Although it has a cast of
\href{https://en.wikipedia.org/wiki/Stick_figures}{stick
figures},\href{https://en.wikipedia.org/wiki/Xkcd\#cite_note-Guzman-6}{{[}4{]}}\href{https://en.wikipedia.org/wiki/Xkcd\#cite_note-7}{{[}5{]}}
the comic occasionally features landscapes and intricate mathematical
patterns such as \href{https://en.wikipedia.org/wiki/Fractal}{fractals},
graphs and
\href{https://en.wikipedia.org/wiki/Chart}{charts}.\href{https://en.wikipedia.org/wiki/Xkcd\#cite_note-8}{{[}6{]}}
New comics are added three times a week, on Mondays, Wednesdays, and
Fridays,\href{https://en.wikipedia.org/wiki/Xkcd\#cite_note-aboutxkcd-3}{{[}‡
1{]}}\href{https://en.wikipedia.org/wiki/Xkcd\#cite_note-redhat-9}{{[}7{]}}
although on some occasions they have been added every weekday.

\end{ajanlo}

\subsection{Sorozatok határértéke a metrikus
térben}\label{sorozatok-hatarerteke-a-metrikus-terben}

\begin{definicio}

Definíció:\\
Egy \(\left. f:{\mathbb{N}}\rightarrow X \right.\) (\(X\) metrikus tér)
függvényt \(X\)-beli sorozatnak nevezünk. Jelölés: a sorozat \(k\)-adik
tagja \(a_{k}: = f\left( k \right)\)-nek, a sorozat
\(\left( a_{k} \right)_{k \in {\mathbb{N}}}: = f\)
\(\left( a_{k} \right) = f\).

\end{definicio}

\begin{definicio}

Definíció:\\
Azt mondjuk, hogy az \(\left( a_{k} \right)\) sorozat határértéke
(limesze) \(a \in X\), ha az \(a\) pont tetszőleges \(\varepsilon\)
sugarú környezetéhez létezik olyan \(k_{0} \in {\mathbb{N}}\)
küszöbszám, hogy \(k > k_{0},k \in {\mathbb{N}}\) esetén
\(a_{k} \in B_{\varepsilon}\left( a \right)\). Másképp:
\(\left. \forall\varepsilon > 0\exists k_{0}:k > k_{0}\Rightarrow\rho\left( {a_{k},a} \right) < \varepsilon \right.\),
ezt így jelöljük:
\({\lim\left( a_{k} \right) \equiv \underset{k\rightarrow\infty}{\lim}a_{k} = a}.\)

\end{definicio}

\subsubsection{A limesz tulajdonságai}\label{a-limesz-tulajdonsagai}

\begin{enumerate}
\def\labelenumi{\arabic{enumi}.}
\item
  Ha \(a_{k} = a\) (minden \(k\)-ra), akkor
  \(\lim\left( a_{k} \right) = a\)
\item
  Tfh \(\lim\left( a_{k} \right) = a\), akkor \(\left( a_{k} \right)\)
  minden részsorozatának határértéke létezik és értékük \(a\).\\
  Részsorozat: \(\left( a_{k} \right)\) véges vagy végtelen sok elemét
  elhagyom úgy, hogy még mindig végtelen sok maradjon, és a sorrenden
  nem változtatok. Másképpen: \(\left( a_{k} \right)\) részsorozata
  \(\left( a_{g_{k}} \right)\), ahol
  \(\left. g:{\mathbb{N}}\rightarrow{\mathbb{N}} \right.\) szigorúan
  monoton növő.

  \begin{bizonyitas}

  Bizonyítás:\\
  \(\left. \lim\left( a_{k} \right): = a\Rightarrow\forall\varepsilon > 0\exists k_{0}:k > k_{0}\Rightarrow\rho\left( {a_{k},a} \right) < \varepsilon \right.\).
  Mivel \(\left. g_{k} \geq k\Rightarrow k > k_{0} \right.\)-ra
  \(\rho\left( {a_{g_{k}},a} \right) < \varepsilon\), hisz ekkor
  \(g_{k} > k_{0}\).

  \end{bizonyitas}
\item
  A határérték egyértelmű.

  \begin{bizonyitas}

  Bizonyítás:\\
  Tfh \(\left( a_{k} \right)\) határértékei \(a\) és \(b\) (\(X\)
  elemei). Belátandó, hogy \(a = b\) . Ekkor egyrészt:
  \[\left. \forall\varepsilon > 0\exists k_{0}:k > k_{0}\Rightarrow\rho\left( {a_{k},a} \right) < \varepsilon \right.,\]
  másrészt
  \(\forall\varepsilon > 0\exists k_{1}:k > k_{1},\rho\left( {a_{k},b} \right) < \varepsilon\)
  \(\left. \Rightarrow k > \max\left\{ {k_{0},k_{1}} \right\} \right.\)
  esetén
  \(\rho\left( {a_{k},a} \right) < \varepsilon,\rho\left( {a_{k},b} \right) < \varepsilon\),
  így a háromszög egyenlőtlenség alapján
  \(\left. \rho\left( {a,b} \right) \leq \rho\left( {a,a_{k}} \right) + \rho\left( {a_{k},b} \right) < 2\varepsilon,\forall\varepsilon > 0\Rightarrow\rho\left( {a,b} \right) = 0\Leftrightarrow a = b \right.\)

  \end{bizonyitas}
\item
  Ha
  \(\left. \lim\left( a_{k} \right) = a\Rightarrow\left( a_{k} \right) \right.\)
  minden átrendezésének a határértéke szintén \(a\)\\
  Egy \(\left( a_{k} \right)\) átrendezése: veszek egy
  \(\left. g:{\mathbb{N}}\rightarrow{\mathbb{N}} \right.\) bijekciót, az
  átrendezett sorozat: \(\left( a_{g_{k}} \right)\).
\item
  Sorozatok összefésülése:\\
  \(\left( a_{k} \right),\left( b_{k} \right)\) \(X\)-beli sorozatok
  összefésülése olyan \(\left( c_{k} \right)\) \(X\)-beli sorozat,
  melynek elemei \(a_{1},b_{1},a_{2},b_{2}...\). Ha
  \(\left. \lim\left( a_{k} \right) = a = \lim\left( b_{k} \right)\Rightarrow\lim\left( c_{k} \right) = a \right.\)
\item
  Ha egy sorozatnak létezik a limesze, akkor korlátos is. (Korlátos:
  létezik olyan \(n\) dimenziós gömb, mely tartalmazza a sorozat összes
  elemét.)

  \begin{bizonyitas}

  Bizonyítás:\\
  \(\left. \lim\left( a_{k} \right) = a\Rightarrow\varepsilon = 1\exists k_{0}:k > k_{0}\Rightarrow\rho\left( {a_{k},a} \right) < 1 \right.\),
  így
  \[r: = \max\left\{ {\rho\left( {a,a_{1}} \right),\rho\left( {a,a_{2}} \right),...,\rho\left( {a,a_{k_{0}}} \right)} \right\}\]
  esetén \(a_{k} \in B_{r + 1}\left( a \right)\forall k\).

  \end{bizonyitas}
\end{enumerate}

\subsubsection{A limesz műveleti tulajdonságai normált
terekben}\label{a-limesz-muveleti-tulajdonsagai-normalt-terekben}

A következőkben \(X\) mindig egy normált teret jelöl,
\(\left( a_{k} \right)\) , illetve \(\left( b_{k} \right)\) pedig
egy-egy \(X\)-beli sorozatot.

A bizonyítások során az egész félévben külön hivatkozás nélkül
használjuk azt a tényt, hogy ha egy \(x_{n} \subset X\) sorozatra
\(\lim\left\| x_{n} \right\| \leq 0\) , akkor
\(\left. x_{n}\rightarrow 0 \right.\) .

\paragraph{Összeadás}\label{osszeadas}

\begin{tetel}

Tétel:\\
Ha
\(\left. \lim\left( a_{k} \right) = a,\lim\left( b_{k} \right) = b\Rightarrow\lim\left( {a_{k} + b_{k}} \right) = a + b \right.\).

\end{tetel}

\begin{bizonyitas}

Bizonyítás:\\
mivel \(\lim\left( a_{k} \right) = a\), ezért
\(\left. \forall\varepsilon > 0\exists k_{0}:k > k_{0}\Rightarrow\rho\left( {a,a_{k}} \right) = \left\| {a_{k} - a} \parallel \right. < \varepsilon \right.\)
és mivel \(\lim\left( b_{k} \right) = b\), ezért
\(\left. \forall\varepsilon > 0\exists k_{1}:k > k_{1}\Rightarrow\rho\left( {b,b_{k}} \right) = \left\| {b_{k} - b} \parallel \right. < \varepsilon \right.,\)
így \[\begin{array}{l}
{\rho\left( {a_{k} + b_{k},a + b} \right) = \left\| {\left( {a_{k} + b_{k}} \right) - \left( {a + b} \right)} \right\| =} \\
{\left\| {\left( {a_{k} - a} \right) + \left( {b_{k} - b} \right)} \right\| \leq \left\| {a_{k} - a} \right\| + \left\| {b_{k} - b} \right\| < 2\varepsilon,} \\
\end{array}\] ha \(k > \max\left\{ {k_{0},k_{1}} \right\}\) . ■

\end{bizonyitas}

\paragraph{Szorzás}\label{szorzas}

\begin{allitas}

Állítás:\\
Legyen \(X\) normált tér! Ha \(\lim\left( \lambda_{k} \right) = 0\) és
\(\left( a_{k} \right)\) korlátos,
\(\left. \Rightarrow\lim\left( {\lambda_{k}a_{k}} \right) = 0 \right..\)

\end{allitas}

\begin{tetel}

Tétel:\\
Tfh \(\lim\left( a_{k} \right) = a\) és
\(\lim\left( \lambda_{k} \right) = \lambda\) (
\(\lambda_{k} \in {\mathbb{R}}\) ). Ekkor
\(\lim\left( {\lambda_{k}a_{k}} \right) = \lambda a\).

\end{tetel}

\begin{bizonyitas}

Bizonyítás:\\
Mivel \(\lim\left( a_{k} \right) = a\) ezért
\(\left. \forall\varepsilon > 0\exists k_{0}:k > k_{0}\Rightarrow\left\| {a_{k} - a} \parallel \right. < \varepsilon \right.\).
Mivel \(\lim\left( \lambda_{k} \right) = k\) ezért
\(\left. \forall\varepsilon > 0\exists k_{1}:k > k_{1}\Rightarrow\left| {\lambda_{k} - \lambda} \right| < \varepsilon \right.\).
Tehát \(k > \max\left\{ {k_{0},k_{1}} \right\}\) esetén
\(\left\| {\lambda_{k}a_{k} - \lambda a} \parallel \right. = \left\| {\left( {\lambda_{k}a_{k} - \lambda a_{k}} \right) + \left( {\lambda a_{k} - \lambda a} \right)} \parallel \right. \leq \left\| {\lambda_{k}a_{k} - \lambda a_{k}} \parallel \right. + \left\| {\lambda a_{k} - \lambda a} \parallel \right. =\)
\(= \left\| {\left( {\lambda_{k} - \lambda} \right)a_{k}} \parallel \right. + \left\| {\lambda\left( {a_{k} - a} \right)} \parallel \right. = \underset{< \varepsilon}{\underbrace{\left| {\lambda_{k} - \lambda} \right|}}\left\| a_{k} \parallel \right. + \underset{\text{rögz.}}{\underbrace{\left| \lambda \right|}}\underset{< \varepsilon}{\underbrace{\left\| {a_{k} - a} \parallel \right.}}\).
Mivel \(\left( a_{k} \right)\) korlátos,
\(\exists M > 0:\left\| a_{k} \parallel \right. < M\forall k \in {\mathbb{N}}\)-re,
tehát \(k > \max\left\{ {k_{0},k_{1}} \right\}\) esetén
\(\left\| {\lambda_{k}a_{k} - \lambda a} \parallel \right. < \varepsilon M + \left| \lambda \right|\varepsilon = \left( {M + \left| \lambda \right|} \right)\varepsilon\)
. ■

\end{bizonyitas}

\paragraph{Osztás}\label{osztas}

\begin{tetel}

Tétel:\\
Legyen \(\left( a_{k} \right)\) egy valós vagy komplex sorozat. Ha
\(\left. a = \lim\left( a_{k} \right) \neq 0\Rightarrow\lim\left( \frac{1}{a_{k}} \right) = \frac{1}{a} \right.\).

\end{tetel}

\begin{bizonyitas}

Bizonyítás:\\
Mivel
\(\left. \lim\left( a_{k} \right) = a\Rightarrow\forall\varepsilon > 0\exists k_{0}:k > k_{0}\Rightarrow\left| {a_{k} - a} \right| < \varepsilon \right.\),
így
\[\left. \exists k_{1}:k > k_{1}\Rightarrow\left| {a_{k} - a} \right| < \varepsilon\left| a \right|^{2}/2 \right..\]
Legyen \(\varepsilon: = \frac{\left| a \right|}{2}\), ekkor
\(\left. \exists k_{2}:k > k_{2}\Rightarrow\left| a_{k} \right| > \frac{\left| a \right|}{2} \right.\).
Legyen \(k > \max\left\{ {k_{1},k_{2}} \right\}\), ekkor
\[{\left| {\frac{1}{a_{k}} - \frac{1}{a}} \right| = \frac{\left| {a - a_{k}} \right|}{\left| {a_{k}a} \right|} < \frac{\varepsilon\left| a \right|^{2}/2}{\left| a_{k} \right|\left| a \right|} = \frac{\varepsilon\left| a \right|/2}{\left| a \right|/2} = \varepsilon}.\]
■

\end{bizonyitas}

\subsubsection{Zárt halmazok jellemzése
sorozatokkal}\label{zart-halmazok-jellemzese-sorozatokkal}

Emlékeztető: \(X\) metrikus térben egy \(M\) halmazt zártnak neveztünk,
ha
\[\left. \partial M \subset M\Leftrightarrow\overline{M} \subset M\Leftrightarrow\overline{M} = M \right.,\]
ahol \(\overline{M} = {int}\left( M \right) \cup \partial M\) , továbbá
\(\left. a \in \overline{M}\Leftrightarrow \right.\) ha \(a\) bármely
környezete tartalmaz \(M\) beli pontot is. Ezek szerint \(M\) zárt
halmaz pontosan akkor, ha minden olyan pont, amelynek bármely
környezetében van \(M\) beli pont, az \(M\)-hez tartozik.

\begin{tetel}

Tétel:\\
Egy \(M \subset X\) halmaz zárt pontosan akkor, ha tetszőleges
konvergens sorozatot nézve, melynek tagjai \(a_{k} \in M\)
\(\lim\left( a_{k} \right) \in M\).

\end{tetel}

\begin{bizonyitas}

Bizonyítás:\\
Az előbbiek szerint \(M\) halmaz zárt pontosan akkor, ha minden olyan
pont, amelynek bármely környezetében van \(M\) beli pont, az \(M\)-hez
tartozik.\\
\(\Rightarrow\) irányban: tfh \(M\) zárt! Ha \(a_{k} \in M\) és
\(\lim\left( a_{k} \right) = a\), akkor \(a \in M\), mert a minden
környezetében van \(M\) beli pont is (nevezetesen \(a_{k}\) ).\\
\(\Leftarrow\) irányban: fordítva is igaz, ha \(a\) minden környezete
tartalmaz \(M\) beli pontot, akkor
\(\exists\left( a_{k} \right) \in M:\lim\left( a_{k} \right) = a\).
Vagyis minden olyan pont (\(a\)), amelynek minden környezetében van
\(M\)-beli pont (az \(a_{k}\)-k), az \(M\)-nek eleme, és a fentiek
szerint ebből következik, hogy \(M\) zárt. ■

\end{bizonyitas}

\paragraph{Korlátos és zárt halmazok, illetve sorozatkompakt
halmazok}\label{korluxe1tos-uxe9s-zuxe1rt-halmazok-illetve-sorozatkompakt-halmazok}

\begin{tetel}

Tétel (Bolzano-Weierstrass-féle kiválasztási tétel \({\mathbb{R}}^{n}\)
-ben) :\\
Legyen \(\left( a_{k} \right)\) korlátos sorozat
\({\mathbb{R}}^{n}\)-ben! Ekkor \(\left( a_{k} \right)\) sorozatnak
létezik konvergens részsorozata.

\end{tetel}

\begin{bizonyitas}

Bizonyítás:\\
Először \(n = 1\) esetre, ekkor
\(\left( {a_{k} \in {\mathbb{R}}} \right)\) korlátos
\(\left. \Rightarrow\exists\left\lbrack {c,d} \right\rbrack a_{k},\forall k \right.\).
Felezzük \(\left\lbrack {c,d} \right\rbrack\) intervallumot! Ekkor a két
zárt fél intervallum közül legalább az egyik végtelen sok tagot
tartalmaz a sorozatból. Ez legyen
\(\left\lbrack {c_{1},d_{1}} \right\rbrack\). Ezt megint felezzük,
melyek közül legalább az egyik végtelen sok tagot tartalmaz a
sorozatból, ez legyen \(\left\lbrack {c_{2},d_{2}} \right\rbrack\)
\ldots{}Így \(a_{k}\)-ból kiválasztható egy \(a_{k_{l}}\) részsorozat
úgy, hogy \(a_{k_{l}} \in \left\lbrack {c_{l},d_{l}} \right\rbrack\).
Belátjuk, hogy \(a_{k_{l}}\) részsorozat konvergens.\\
\(\left\lbrack {c,d} \right\rbrack \supset \left\lbrack {c_{1},d_{1}} \right\rbrack \supset \left\lbrack {c_{2},d_{2}} \right\rbrack \supset ... \supset \left\lbrack {c_{l},d_{l}} \right\rbrack\),
\(\underset{l\rightarrow\infty}{\lim}\left| {c_{l} - d_{l}} \right| = \underset{l\rightarrow\infty}{\lim}\frac{c - d}{2^{l}} = 0\).
Tudjuk, hogy \(\left\{ {c_{l}:l \in {\mathbb{N}}} \right\}\) felülről
korlátos
\(\left. \Rightarrow\exists\sup\left\{ {c_{l}:l \in {\mathbb{N}}} \right\} \right.\)
és azt is, hogy \(\left\{ {d_{l}:l \in {\mathbb{N}}} \right\}\) alulról
korlátos
\(\left. \Rightarrow\exists\inf\left\{ {d_{l}:l \in {\mathbb{N}}} \right\} \right.\).
Mivel
\[\left. \underset{l\rightarrow\infty}{\lim}\left| {c_{l} - d_{l}} \right| = 0\Rightarrow\sup\left\{ {c_{l}:l \in {\mathbb{N}}} \right\} = \inf\left\{ {d_{l}:l \in {\mathbb{N}}} \right\}: = \alpha \right.,\]
továbbá
\(\left. a_{k_{l}} \in \left\lbrack {c_{l},d_{l}} \right\rbrack\Rightarrow\lim\left( a_{k_{l}} \right) = \alpha \right.\)
(„rendőr-elv'').\\
\(n = 2\) esetre, ekkor
\(a_{k} = \left( {a_{k}^{(1)},a_{k}^{(2)}} \right)\). Mivel \(a_{k}\)
korlátos sorozat \({\mathbb{R}}^{2}\)-ben, így
\(a_{k}^{(1)},a_{k}^{(2)}\) korlátos sorozatok \(\mathbb{R}\)-ben. Az
előzőek szerint az előbbiből kiválasztható ebből egy konvergens
részsorozat, \(\left( a_{k_{l}}^{(1)} \right)_{l \in {\mathbb{N}}}\).
Tekintsük az \(a_{k}^{(2)}\) ugyanilyen indexű elemekből álló
\(\left( a_{k_{l}}^{(2)} \right)\) részsorozatát (mely korlátos
\(\mathbb{R}\)-ben). Az előzőek szerint ennek létezik konvergens
részsorozata,
\(\left( a_{k_{l_{m}}}^{(2)} \right)_{m \in {\mathbb{N}}}\).
\(\left( a_{k_{l}}^{(1)} \right)_{l \in {\mathbb{N}}}\) konvergens, így
\(\left( a_{k_{l_{m}}}^{(1)} \right)_{m \in {\mathbb{N}}}\) is az, így
\(\left( a_{k_{l_{m}}} \right): = \left( {a_{k_{l_{m}}}^{(1)},a_{k_{l_{m}}}^{(2)}} \right)\)
részsorozat konvergens.\\
\(n = 3\) esetén hasonló módon, mint \(n = 1\)-ről váltottunk \(n = 2\)
-re, itt is igazolható (tkp teljes indukció). ■

\end{bizonyitas}

\begin{megjegyzes}

Megjegyzés:\\
Hasonló jellegű állítások általában nem igazak tetszőleges normált
terekben, csak véges dimenzióban!

\end{megjegyzes}

\begin{ajanlo}

\begin{ajanlofig}

\href{https://www.youtube.com/watch?v=e8o6rt6o4LA}{\includegraphics[width=3.80208in,height=2.80208in]{tomandjerry.jpg}}

\end{ajanlofig}

Ne feledjétek, ebből a tárgyól vizsga lesz! Ha hétről hétre tanulsz, és
a kérdéseket időben felteszed a tanárnak, sokkal könnyebb felkészülni a
vizsgára.

\end{ajanlo}

\end{document}
